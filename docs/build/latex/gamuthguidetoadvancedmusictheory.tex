%% Generated by Sphinx.
\def\sphinxdocclass{report}
\documentclass[letterpaper,10pt,english]{sphinxmanual}
\ifdefined\pdfpxdimen
   \let\sphinxpxdimen\pdfpxdimen\else\newdimen\sphinxpxdimen
\fi \sphinxpxdimen=.75bp\relax
\ifdefined\pdfimageresolution
    \pdfimageresolution= \numexpr \dimexpr1in\relax/\sphinxpxdimen\relax
\fi
%% let collapsible pdf bookmarks panel have high depth per default
\PassOptionsToPackage{bookmarksdepth=5}{hyperref}

\PassOptionsToPackage{warn}{textcomp}
\usepackage[utf8]{inputenc}
\ifdefined\DeclareUnicodeCharacter
% support both utf8 and utf8x syntaxes
  \ifdefined\DeclareUnicodeCharacterAsOptional
    \def\sphinxDUC#1{\DeclareUnicodeCharacter{"#1}}
  \else
    \let\sphinxDUC\DeclareUnicodeCharacter
  \fi
  \sphinxDUC{00A0}{\nobreakspace}
  \sphinxDUC{2500}{\sphinxunichar{2500}}
  \sphinxDUC{2502}{\sphinxunichar{2502}}
  \sphinxDUC{2514}{\sphinxunichar{2514}}
  \sphinxDUC{251C}{\sphinxunichar{251C}}
  \sphinxDUC{2572}{\textbackslash}
\fi
\usepackage{cmap}
\usepackage[T1]{fontenc}
\usepackage{amsmath,amssymb,amstext}
\usepackage{babel}



\usepackage{tgtermes}
\usepackage{tgheros}
\renewcommand{\ttdefault}{txtt}



\usepackage[Bjarne]{fncychap}
\usepackage[,numfigreset=1,mathnumfig]{sphinx}

\fvset{fontsize=auto}
\usepackage{geometry}

\usepackage{sphinxcontribtikz}

% Include hyperref last.
\usepackage{hyperref}
% Fix anchor placement for figures with captions.
\usepackage{hypcap}% it must be loaded after hyperref.
% Set up styles of URL: it should be placed after hyperref.
\urlstyle{same}


\usepackage{sphinxmessages}
\setcounter{tocdepth}{1}



\title{GAMuTh: Guide to Advanced Music Theory}
\date{Mar 14, 2022}
\release{0.0.1}
\author{Fabian C.\@{} Moss}
\newcommand{\sphinxlogo}{\vbox{}}
\renewcommand{\releasename}{Release}
\makeindex
\begin{document}

\pagestyle{empty}
\sphinxmaketitle
\pagestyle{plain}
\sphinxtableofcontents
\pagestyle{normal}
\phantomsection\label{\detokenize{index::doc}}


\begin{sphinxadmonition}{warning}{Warning:}
\sphinxAtStartPar
The content on these pages is very much under construction!
Since this is ongoing work, I can give no guarantee for completeness or accuracy.
Feel free to \sphinxhref{mailto:fabianmoss@gmail.com}{contact me} with your questions and suggestions!
\end{sphinxadmonition}

\sphinxAtStartPar
Welcome!

\sphinxAtStartPar
This is not a pedagogical resource for basic music theory concepts
but an in\sphinxhyphen{}depth introduction into the structures of Western music,
built axiomatically from tones and their relations.
The logo, a \sphinxhref{https://en.wikipedia.org/wiki/Maxima\_(music)}{maxima}, the longest note value in medieval music,
symbolically reflects this level of difficulty.

\sphinxAtStartPar
The content presented on these pages is inspired by a number of great books, e.g.
Aldwell \sphinxstyleemphasis{et al.} {[}\hyperlink{cite.8_bibliography:id12}{1}{]}, Lewin {[}\hyperlink{cite.8_bibliography:id2}{26}{]}, Straus {[}\hyperlink{cite.8_bibliography:id3}{57}{]},
Cadwallader and Gagné {[}\hyperlink{cite.8_bibliography:id18}{5}{]}, and Müller {[}\hyperlink{cite.8_bibliography:id24}{42}{]}.
What is new and unique about the approach taken here is that we take
a computational perspective and implement all introduced concepts.
This does not only provide us with sharp and unequivocal definitions,
but also allows us to scale music theory up from the analysis of individual
bars, sections, or pieces to that of entire repertoires and corpora!

\sphinxAtStartPar
I recently also discovered \sphinxhref{https://pedrokroger.net/mfgan/}{Music for Geeks and Nerds} by Pedro Kroger
which looks very interesting.
The Python project \sphinxhref{https://github.com/gciruelos/musthe}{musthe} also seems to pursue a similar goal.

\begin{sphinxadmonition}{note}{Note:}
\sphinxAtStartPar
Some content is only available in the online HTML version and not in the PDF,
e.g. scores rendered with Lilypond. Please visit the \sphinxstylestrong{website} to see the full version.
\end{sphinxadmonition}

\begin{sphinxadmonition}{note}{Note:}
\sphinxAtStartPar
TODO: add link to website
\end{sphinxadmonition}
\subsubsection*{Quickstart}

\begin{sphinxadmonition}{warning}{Warning:}
\sphinxAtStartPar
These instructions may not work yet.
\end{sphinxadmonition}

\sphinxAtStartPar
Installation of the \sphinxtitleref{gamuth} Python library is as easy as it can be. Just
type the following in your terminal:

\begin{sphinxVerbatim}[commandchars=\\\{\}]
\PYG{n}{pip} \PYG{n}{install} \PYG{n}{gamuth}
\end{sphinxVerbatim}

\sphinxAtStartPar
Then, in a Python script or Jupyter notebook, import the library or individual classes:

\begin{sphinxVerbatim}[commandchars=\\\{\}]
\PYG{g+gp}{\PYGZgt{}\PYGZgt{}\PYGZgt{} }\PYG{k+kn}{from} \PYG{n+nn}{gamuth} \PYG{k+kn}{import} \PYG{n}{Tone}\PYG{p}{,} \PYG{n}{Interval}

\PYG{g+gp}{\PYGZgt{}\PYGZgt{}\PYGZgt{} }\PYG{n}{t} \PYG{o}{=} \PYG{n}{Tone}\PYG{p}{(}\PYG{n}{octave}\PYG{o}{=}\PYG{l+m+mi}{0}\PYG{p}{,} \PYG{n}{fifth}\PYG{o}{=}\PYG{l+m+mi}{0}\PYG{p}{,} \PYG{n}{third}\PYG{o}{=}\PYG{l+m+mi}{0}\PYG{p}{)} \PYG{c+c1}{\PYGZsh{} C}
\PYG{g+gp}{\PYGZgt{}\PYGZgt{}\PYGZgt{} }\PYG{n}{s} \PYG{o}{=} \PYG{n}{Tone}\PYG{p}{(}\PYG{n}{octave}\PYG{o}{=}\PYG{l+m+mi}{0}\PYG{p}{,} \PYG{n}{fifth}\PYG{o}{=}\PYG{l+m+mi}{0}\PYG{p}{,} \PYG{n}{third}\PYG{o}{=}\PYG{l+m+mi}{1}\PYG{p}{)} \PYG{c+c1}{\PYGZsh{} E}

\PYG{g+gp}{\PYGZgt{}\PYGZgt{}\PYGZgt{} }\PYG{n}{i} \PYG{o}{=} \PYG{n}{Interval}\PYG{p}{(}\PYG{n}{t}\PYG{p}{,} \PYG{n}{s}\PYG{p}{)}

\PYG{g+gp}{\PYGZgt{}\PYGZgt{}\PYGZgt{} }\PYG{n+nb}{print}\PYG{p}{(}\PYG{n}{i}\PYG{o}{.}\PYG{n}{specific} \PYG{n}{interval}\PYG{p}{)} \PYG{c+c1}{\PYGZsh{} directed interval in semitones (major third)}
\PYG{g+go}{4}
\end{sphinxVerbatim}

\sphinxAtStartPar
See the documentation {\hyperref[\detokenize{api:api}]{\sphinxcrossref{\DUrole{std,std-ref}{API \sphinxhyphen{} gamuth}}}} for examples how to use the library, its classes and their methods.


\chapter{Fundamentals}
\label{\detokenize{1_fundamentals:fundamentals}}\label{\detokenize{1_fundamentals::doc}}
\sphinxAtStartPar
The theory presented in here can be described as a \sphinxstyleemphasis{tonal theory} in the sense
that its most fundamental objects are \sphinxstyleemphasis{tones}, discrete musical entities that have
a certain location in tonal space.
A tonal space is then a metrical space describing all possible tone locations,
and the metric is given by an \sphinxstyleemphasis{interval function} between the tones. Note that by this definition,
there are as many different tonal spaces as there are interval functions.

\sphinxAtStartPar
While many aspects and examples will be taken
from Western (classical) music, the theory is in principle not restricted to this
tradition but extends well to virtually all musical cultures where a tone is a meaningful concept.

\sphinxAtStartPar
Perhaps the most simple description of music is \sphinxstyleemphasis{sound organized in time} (attributed to
Edgar Varèse, see also {[}\hyperlink{cite.8_bibliography:id39}{70}{]}).
Later we will see that this description falls short of encompassing many central aspects to music
but it provides a good starting point for our considerations. Taking this definition for granted
means that we can conceptualize music within a two\sphinxhyphen{}dimensional framework, where the axes
represent sound and time, respectively (see \hyperref[\detokenize{1_fundamentals:fig-soundtime}]{Fig.\@ \ref{\detokenize{1_fundamentals:fig-soundtime}}}). Note that only the x\sphinxhyphen{}axis (“Time”)
is not represented as an arrow to indicate that in music (as in life) we can only move forward.
\begin{figure}[htbp]\centering\capstart\begin{tikzpicture}
\draw[thick,->,>=stealth] (-0.3,0) -- (8,0) node[midway,below] {Time};
\draw[thick] (0,-0.3) -- (0,3) node[midway,above,sloped] {Sound};
\end{tikzpicture}\caption{Two\sphinxhyphen{}dimensional depiction of music: music as sound organized in time.}\label{\detokenize{1_fundamentals:id7}}\label{\detokenize{1_fundamentals:fig-soundtime}}\end{figure}
\sphinxAtStartPar
This is also the way music is usually displayed in \sphinxstyleemphasis{Digital Audio Workstations} (DAW) that feature
a master window where music blocks can be arranged along a fixed timeline. Producing music in
these environments thus quite literally consists of stacking blocks on top of one another.


\section{Tones}
\label{\detokenize{1_fundamentals:tones}}
\sphinxAtStartPar
Let’s start with a mental exercise: imagine a tone.
Contemplate for a while what this means.
Does this tone have a pitch? A duration? A velocity (volume)?
\begin{itemize}
\item {} 
\sphinxAtStartPar
Riemann (1916). \sphinxstyleemphasis{Ideen zu einer Lehre von den Tonvorstellungen}:

\end{itemize}

\sphinxAtStartPar
“The ultimate elements of the tonal imagination are single tones.”
(Wason and Marvin {[}\hyperlink{cite.8_bibliography:id19}{66}{]}, p. 92).

\sphinxAtStartPar
Bearing that in mind, let’s create (or \sphinxstyleemphasis{instantiate}) a tone. To do so, we need to
conceptualize (“vorstellen” in Riemann’s terminology) a \sphinxstyleemphasis{tone location}
(“Tonort”, Mazzola {[}\hyperlink{cite.8_bibliography:id45}{31}{]}, p. 241).
There are many different ways to do this. In fact, the way we specify the location of a tone
defines the tonal space in which it is situated.
The figure below is an adaptation from Lewin {[}\hyperlink{cite.8_bibliography:id2}{26}{]}.
\begin{figure}[htbp]\centering\capstart\begin{tikzpicture}
\node[circle,fill,inner sep=0pt,minimum size=5pt,label={below: $s$}] (s) at (0,0) [below] {};
\node[circle,fill,inner sep=0pt,minimum size=5pt,label={below: $t$}] (t) at (4,1.5) {};
\draw[thick,->,-stealth,shorten <=2pt, shorten >=2pt] (s) -- (t) node[midway,below,sloped] {$i$};
\end{tikzpicture}\caption{Two abstract tones \(s\) and \(t\), and the interval \(i\) between them.}\label{\detokenize{1_fundamentals:id8}}\end{figure}
\sphinxAtStartPar
The fact that music operates with discrete pitches has also been argued to be crucial for its
evolution {[}\hyperlink{cite.8_bibliography:id48}{60}{]}.


\subsection{Frequencies}
\label{\detokenize{1_fundamentals:frequencies}}
\sphinxAtStartPar
Each tone corresponds to some \sphinxstyleemphasis{fundamental frequency} \(f\) in Hertz (Hz),
oscillations per second.
\begin{itemize}
\item {} 
\sphinxAtStartPar
Overtone series

\item {} 
\sphinxAtStartPar
frequency ratios

\item {} 
\sphinxAtStartPar
logarithm: multiplication =\textgreater{} addition

\end{itemize}


\subsection{Euler Space}
\label{\detokenize{1_fundamentals:euler-space}}
\sphinxAtStartPar
One option is to locate a tone \sphinxtitleref{t} as a point \(p=(o, q, t)\) in Euler Space, defined by
a number of octaves \sphinxtitleref{o}, fifths \sphinxtitleref{q}, and thirds \sphinxtitleref{t}. We will use the \sphinxcode{\sphinxupquote{gamuth.Tone}}
class for this

\begin{sphinxVerbatim}[commandchars=\\\{\}]
\PYG{k+kn}{from} \PYG{n+nn}{gamuth} \PYG{k+kn}{import} \PYG{n}{Tone}

\PYG{n}{t} \PYG{o}{=} \PYG{n}{Tone}\PYG{p}{(}\PYG{n}{o}\PYG{o}{=}\PYG{l+m+mi}{0}\PYG{p}{,} \PYG{n}{q}\PYG{o}{=}\PYG{l+m+mi}{0}\PYG{p}{,} \PYG{n}{t}\PYG{o}{=}\PYG{l+m+mi}{0}\PYG{p}{)}
\end{sphinxVerbatim}

\sphinxAtStartPar
From this representation we can derive a variety of others, corrsponding to transformations of
tonal space.


\subsection{Octave equivalence}
\label{\detokenize{1_fundamentals:octave-equivalence}}
\sphinxAtStartPar
Octave equivalance considers all tones to be equivalent that are separated by one or
multiple octaves, e.g C1, C2, C4, C10 etc. More precisely, all tones whose fundamental frequencies
are related by multiples of 2 are octave equivalent.


\subsection{Tonnetz}
\label{\detokenize{1_fundamentals:tonnetz}}
\sphinxAtStartPar
The \sphinxstyleemphasis{Tonnetz} does not contain octaves and thus corresponds to a projection
\begin{equation*}
\begin{split}\pi: (o, q, t) \mapsto (q, t).\end{split}
\end{equation*}

\section{Pitch classes}
\label{\detokenize{1_fundamentals:pitch-classes}}
\sphinxAtStartPar
A very common object in music theory is that of a \sphinxstyleemphasis{pitch class}. Pitch classes
are equivalence classes of tones that incorporate some kind of invariance.
The two most common equivalences are \sphinxstyleemphasis{octave equivalence} and \sphinxstyleemphasis{enharmonic equivalence}.


\subsection{Enharmonic equivalence}
\label{\detokenize{1_fundamentals:enharmonic-equivalence}}
\sphinxAtStartPar
If, in addition to octave equivalence, one further assumes enharmonic equivalence,
all tones separated by 12 fifths on the line of fifths
are considered to be equivalent, e.g. \(\text{A}\sharp\) and \(\text{B}\flat\),
\(\text{F}\sharp\) and \(\text{G}\flat\), \(\text{G}\sharp\), and \(\text{A}\flat\) etc.

\sphinxAtStartPar
The notion of a pitch class usually entails both octave and enharmonic equivalence.
Consequently, there are twelve pitch classes. If not mentioned otherwise, we adopt this convention here.
The twelve pitch classes are usually referred to by their most simple representatives, i.e.
\begin{equation*}
\begin{split}\text{C, C$\sharp$, D, E$\flat$, F, F$\sharp$, G, A$\flat$, A, B$\flat$, B},\end{split}
\end{equation*}
\sphinxAtStartPar
but it is more appropriate to use \sphinxstyleemphasis{integer notation} in which each pitch class is represented
by an integer \(k \in \mathbb{Z}_{12}\).
\begin{equation*}
\begin{split}\mathbb{Z}_{12}=\{0, 1, 2, 3, 4, 5, 6, 7, 8, 9, 10, 11\},\end{split}
\end{equation*}
\sphinxAtStartPar
and usually one sets \(0\equiv \text{C}\). This allows to use \sphinxstyleemphasis{modular arithmetic}
do calculations with pitch classes. In standard music notation, this would be rendered as

\noindent\sphinxincludegraphics{{./_lilypond/4a8a6697e4903f58c19b2a9f28c197772704f0c1/music}.png}

\sphinxAtStartPar
audio option is supported for latex builder



\subsection{Other invariances}
\label{\detokenize{1_fundamentals:other-invariances}}
\sphinxAtStartPar
OPTIC


\subsection{Tuning / Temperament}
\label{\detokenize{1_fundamentals:tuning-temperament}}

\section{Intervals}
\label{\detokenize{1_fundamentals:intervals}}\label{\detokenize{1_fundamentals:id6}}\begin{itemize}
\item {} 
\sphinxAtStartPar
Pitch intervals

\item {} 
\sphinxAtStartPar
Ordered pitch\sphinxhyphen{}class intervals (\sphinxhyphen{}\textgreater{} rather directed)

\item {} 
\sphinxAtStartPar
Unordered pitch\sphinxhyphen{}class intervals

\item {} 
\sphinxAtStartPar
Interval classes

\item {} 
\sphinxAtStartPar
Interval\sphinxhyphen{}class content

\item {} 
\sphinxAtStartPar
Interval\sphinxhyphen{}class vector

\end{itemize}


\subsection{GISs}
\label{\detokenize{1_fundamentals:giss}}\label{\detokenize{1_fundamentals:sec-gis}}
\sphinxAtStartPar
Transformations between representations of tones are actually \sphinxstyleemphasis{transformations of tonal space}.

\sphinxAtStartPar
{[}Diagram of relations between different representations.{]}


\chapter{Scales and Modes}
\label{\detokenize{2_scales_modes:scales-and-modes}}\label{\detokenize{2_scales_modes::doc}}
\sphinxAtStartPar
A vast majority of musical cultures globally have some notion or concept of organization systems
for discrete tones {[}\hyperlink{cite.8_bibliography:id49}{4}{]}. Those can be called \sphinxstyleemphasis{scales} or \sphinxstyleemphasis{modes}.
While this general description holds, the musical reality is incredibly diverse and seems to defy
uniform analytical approaches {[}\hyperlink{cite.8_bibliography:id50}{34}{]}.

\sphinxAtStartPar
In this chapter, we are going to learn some tonal characteristics and organization systems from a range of
musical cultures from different continents. While this will already hint at the global variation,
it is by no means exhaustive. Nor are the descriptions below meant to entirely represent those cultures.
Many factors, especially contextual sociopolitical ones, are out of the scope of this book, and we
restrict the descriptions to what we can model using the approch established in the earlier chapter:
a formal description of tonal relations.

\sphinxAtStartPar
Powers \sphinxstyleemphasis{et al.} {[}\hyperlink{cite.8_bibliography:id23}{49}{]}


\section{Indian classical music}
\label{\detokenize{2_scales_modes:indian-classical-music}}

\section{Turkish Maqam}
\label{\detokenize{2_scales_modes:turkish-maqam}}
\sphinxAtStartPar
\sphinxhref{https://ratioscore.humdrum.org/turkish/}{Makam Dataset} {[}\hyperlink{cite.8_bibliography:id56}{25}{]}


\section{Arab\sphinxhyphen{}Andalusian music}
\label{\detokenize{2_scales_modes:arab-andalusian-music}}
\sphinxAtStartPar
Nuttall \sphinxstyleemphasis{et al.} {[}\hyperlink{cite.8_bibliography:id75}{46}{]}


\section{Persian Music}
\label{\detokenize{2_scales_modes:persian-music}}
\sphinxAtStartPar
FamourZadeh {[}\hyperlink{cite.8_bibliography:id73}{15}{]}, Sanati {[}\hyperlink{cite.8_bibliography:id69}{53}{]}


\section{Western classical music}
\label{\detokenize{2_scales_modes:western-classical-music}}\begin{itemize}
\item {} 
\sphinxAtStartPar
Ancient Greek modes {[}\hyperlink{cite.8_bibliography:id55}{45}{]}

\item {} 
\sphinxAtStartPar
Ecclesiastic modes {[}\hyperlink{cite.8_bibliography:id58}{2}, \hyperlink{cite.8_bibliography:id66}{10}, \hyperlink{cite.8_bibliography:id67}{12}, \hyperlink{cite.8_bibliography:id68}{47}, \hyperlink{cite.8_bibliography:id57}{68}{]}

\item {} 
\sphinxAtStartPar
Major and minor {[}\hyperlink{cite.8_bibliography:id74}{22}{]}

\item {} 
\sphinxAtStartPar
Modes of limited transposition

\item {} 
\sphinxAtStartPar
Georgian liturgic chorales (quartal harmonies) {[}\hyperlink{cite.8_bibliography:id20}{52}{]}

\end{itemize}


\subsection{The diatonic scale}
\label{\detokenize{2_scales_modes:the-diatonic-scale}}
\sphinxAtStartPar
Music in the Western tradition fundamentally builds on
so\sphinxhyphen{}called \sphinxstyleemphasis{diatonic} scales, an arrangement of seven tones
that are named with latin letters from A to G. “Diatonic” can
be roughly translated into “through all tones”. Within this scale,
no tone is privileged, so the diatonic scale can be appropriately
represented by a circle with seven points on it. Mathemacally,
this structure is equivalent to \(\mathbb{Z}_7\).

\sphinxAtStartPar
{[}tikz figure here{]}

\sphinxAtStartPar
Now, if we want to determine the relative relations between the tones,
it is necessary to assign a reference tone that is commonly called the \sphinxstyleemphasis{tonic},
or \sphinxstyleemphasis{finalis} in older music.

\sphinxAtStartPar
For example, if the tone D is the tonic, we can determine all other scale degrees
as distance to this tone. Scale degrees are commonly notated with arabic numbers with a caret:
\begin{equation*}
\begin{split}\text{D}: \hat{1}\\
\text{E}: \hat{2}\\
\text{F}: \hat{3}\\
\text{G}: \hat{4}\\
\text{A}: \hat{5}\\
\text{B}: \hat{6}\\
\text{C}: \hat{7}\\\end{split}
\end{equation*}
\sphinxAtStartPar
Taking these seven notes in scalar order, they can be converted to their \sphinxstyleemphasis{fifth order} via
\begin{equation*}
\begin{split}\phi: t \mapsto 4t \mod 7\end{split}
\end{equation*}
\sphinxAtStartPar
because the octave is divided into 7 steps and there and a fifths consists of 4 steps.
Under this view the diatonic scale is a subsegment
of the \sphinxstyleemphasis{line of fifths} {[}\hyperlink{cite.8_bibliography:id44}{37}, \hyperlink{cite.8_bibliography:id40}{58}{]}
\phantomsection \label{exercise:ex:tonnetz}

\begin{sphinxadmonition}{note}{Exercise 2.1}



\sphinxAtStartPar
How does the fifth order relate to the Euler Space / Tonnetz mentioned earlier?
\end{sphinxadmonition}
\phantomsection \label{2_scales_modes:sol:tonnetz}

\begin{sphinxadmonition}{note}{Solution to Exercise 2.1}



\sphinxAtStartPar
This is the solution
\end{sphinxadmonition}


\subsection{Modes}
\label{\detokenize{2_scales_modes:modes}}
\sphinxAtStartPar
scale plus order plus hierarchy (but order already defined above?)


\subsection{Keys}
\label{\detokenize{2_scales_modes:keys}}

\section{Jazz}
\label{\detokenize{2_scales_modes:jazz}}

\section{Other scales}
\label{\detokenize{2_scales_modes:other-scales}}\begin{itemize}
\item {} 
\sphinxAtStartPar
chromatic

\item {} 
\sphinxAtStartPar
hexatonic

\item {} 
\sphinxAtStartPar
octatonic

\item {} 
\sphinxAtStartPar
whole tone

\item {} 
\sphinxAtStartPar
Messiaen

\end{itemize}

\sphinxAtStartPar
Before we move on to another important musical dimension, time, we have to consider
one of the most famous musical scale systems (at least among music academics): Balinese Pelog and Slendro.


\section{Balinese Pelog and Slendro}
\label{\detokenize{2_scales_modes:balinese-pelog-and-slendro}}
\sphinxAtStartPar
Gamelan

\sphinxAtStartPar
See descriptions of tunings \sphinxhref{http://www.aawmjournal.com/articles/2021b/Vitale\_Sethares\_AAWM\_Vol\_9\_2.pdf}{here}
and data \sphinxhref{http://www.aawmjournal.com/supplemental/2021b/TothGongKebyarSpreadsheets.zip}{here}
{[}\hyperlink{cite.8_bibliography:id33}{65}, \hyperlink{cite.8_bibliography:id34}{67}{]}.


\chapter{Pitch\sphinxhyphen{}Class Set Theory}
\label{\detokenize{3_set_theory:pitch-class-set-theory}}\label{\detokenize{3_set_theory::doc}}
\sphinxAtStartPar
Content adapted from Forte {[}\hyperlink{cite.8_bibliography:id25}{16}{]}, Straus {[}\hyperlink{cite.8_bibliography:id3}{57}{]}.
In this chapter, “pitch class” always entails octave \sphinxstyleemphasis{and} enharmonic equicalenve.


\section{Pitch classes}
\label{\detokenize{3_set_theory:pitch-classes}}

\subsection{Octave equivalence}
\label{\detokenize{3_set_theory:octave-equivalence}}
\sphinxAtStartPar
Octave equivalence maps all pitch classes with the same label/name to one class,
e.g. all C’s on the piano get mapped to pitch class C, all Ab’s get mapped to pitch class Ab,
and all G\#’s get mapped to G\#.

\sphinxAtStartPar
Octave equivalence has perceptual correlates (cite paper) and is frequently employed in composition.
For instance, the double bass sounds an octave lower as a cello but they are notated identically,
or a fugue subject enters in different registers.
The octave is the most fundamental interval with a frequency ratio of 2:1.
Octave equivalence transforms the Euler space to the Tonnetz \(\mathbb{Z}\times\mathbb{Z}\).

\sphinxAtStartPar
{[}diagram{]}


\subsection{Enharmonic equivalence}
\label{\detokenize{3_set_theory:enharmonic-equivalence}}
\sphinxAtStartPar
Enharmonic equivalence describes the identification of pitches like F\#4 and G\#4.
Enharmonic equivalence transforms the Euler Space to a tube, namely \(\mathbb{Z}_{12} \times \mathbb{Z}\)
(the chromatic circle and the octave line).

\sphinxAtStartPar
{[}diagram{]}

\sphinxAtStartPar
Taking octave equivalence and enharmonic equivalence together,
we arrive at the torus that contains all twelve enharmonic (and octave\sphinxhyphen{}related) pitch classes,
and which can be represented by numbers \(k\in \mathbb{Z}_{12}\)

\sphinxAtStartPar
{[}diagram{]}

\sphinxAtStartPar
Both equivalences are indepentend of one another, i.e. we have the following commuting diagram between tonal spaces:

\sphinxAtStartPar
{[}diagram{]}

\sphinxAtStartPar
Assuming enharmonic equivalence has a number of implications:
\begin{itemize}
\item {} 
\sphinxAtStartPar
we cannot distinguish between diatonic and chromatic semitones

\item {} 
\sphinxAtStartPar
we cannot distinguish between augmented fourths and diminished fifths

\item {} 
\sphinxAtStartPar
we cannot distinguish between dominant seventh chords and augmented sixth chords

\item {} 
\sphinxAtStartPar
and more

\end{itemize}

\begin{sphinxadmonition}{note}{Note:}
\sphinxAtStartPar
Describe relation to equal temperament.
\end{sphinxadmonition}

\sphinxAtStartPar
For the remainder of this chapter, pitch classes designate entities for which both equivalences are assumed.
It is thus appropriate to represent them on a circle. Since each pitch class can have infinitely many spellings,
going the reverse direction is a difficult inference problem, for which a number of algorithms have been proposed
(see advanced chapter on {\hyperref[\detokenize{6_advanced:pitch-spelling}]{\sphinxcrossref{\DUrole{std,std-ref}{Pitch Spelling}}}}).


\section{Intervals}
\label{\detokenize{3_set_theory:intervals}}
\sphinxAtStartPar
Generally, intervals describe the distance between two points (see Chapter {\hyperref[\detokenize{1_fundamentals:intervals}]{\sphinxcrossref{\DUrole{std,std-ref}{Intervals}}}}).
But depending on which representation for tones we chose (or, equivalently, which equivalences we assume)
the types of the intervals changes as well.

\begin{sphinxadmonition}{note}{Note:}
\sphinxAtStartPar
Add paragraph about what types are.
\end{sphinxadmonition}


\subsection{Pitch Intervals}
\label{\detokenize{3_set_theory:pitch-intervals}}
\sphinxAtStartPar
Pitch intervals describe the distance between two pitches in semitones. That is, we do assume enharmonic equivalence
but not octave equivalence. Consequently, we can visualize pitch intervals in

\sphinxAtStartPar
Ordered pitch\sphinxhyphen{}class intervals

\sphinxAtStartPar
Unordered pitch\sphinxhyphen{}class intervals

\sphinxAtStartPar
Interval class

\sphinxAtStartPar
Interval class content

\sphinxAtStartPar
Interval class vector


\section{Pitch\sphinxhyphen{}class sets}
\label{\detokenize{3_set_theory:pitch-class-sets}}
\sphinxAtStartPar
Normal form

\sphinxAtStartPar
Transposition

\sphinxAtStartPar
Inversion I

\sphinxAtStartPar
Index number

\sphinxAtStartPar
Inversion II

\sphinxAtStartPar
Set class

\sphinxAtStartPar
Prime form

\sphinxAtStartPar
Segmentation and analysis


\section{Relationships}
\label{\detokenize{3_set_theory:relationships}}
\sphinxAtStartPar
Common tones under transposition

\sphinxAtStartPar
Transpositional symmetry

\sphinxAtStartPar
Common tones under inversion

\sphinxAtStartPar
Inversional symmetry

\sphinxAtStartPar
Z\sphinxhyphen{}relation

\sphinxAtStartPar
Complement relation

\sphinxAtStartPar
Subset and superset relations

\sphinxAtStartPar
Transpositional combination

\sphinxAtStartPar
Contour relations

\sphinxAtStartPar
Composing out

\sphinxAtStartPar
Voice\sphinxhyphen{}leading

\sphinxAtStartPar
Atonal pitch space


\section{Advanced concepts}
\label{\detokenize{3_set_theory:advanced-concepts}}
\sphinxAtStartPar
Tonality

\sphinxAtStartPar
Centricity

\sphinxAtStartPar
Inversional axis

\sphinxAtStartPar
The diatonic collection

\sphinxAtStartPar
The octatonic collection

\sphinxAtStartPar
The whole\sphinxhyphen{}tone collection

\sphinxAtStartPar
The hexatonic collection

\sphinxAtStartPar
Collectional interaction

\sphinxAtStartPar
Interval cycles

\sphinxAtStartPar
Triadic post\sphinxhyphen{}tonality


\section{Twelve\sphinxhyphen{}tone theory}
\label{\detokenize{3_set_theory:twelve-tone-theory}}
\sphinxAtStartPar
Twelve\sphinxhyphen{}tone series

\sphinxAtStartPar
Basic operations

\sphinxAtStartPar
Subset structure

\sphinxAtStartPar
Invariants


\chapter{Time}
\label{\detokenize{4_time:time}}\label{\detokenize{4_time:id1}}\label{\detokenize{4_time::doc}}\begin{itemize}
\item {} 
\sphinxAtStartPar
beats

\item {} 
\sphinxAtStartPar
seconds

\item {} 
\sphinxAtStartPar
onsets

\end{itemize}


\section{Notes}
\label{\detokenize{4_time:notes}}
\sphinxAtStartPar
(Tones + Duration)
blablabla…

\sphinxAtStartPar
Sinve the relations between tones only given by
their location in tonal space (and the interval function)
generalizing the notion of neighbor notes etc. corresponds
to changing what the \sphinxstyleemphasis{lines} in Western notation mean.
Traditionally, two lines separate tones that are a generic third apart.
But there have been other representations.
For instance, the first attempts of Guido separated notes by steps.
Let’s reinterpret the lines as seconds and fifths.
There have also been a number of attempts to develop a fully chromatic
notation system (e.g. Parncutt).


\section{Performance}
\label{\detokenize{4_time:performance}}
\sphinxAtStartPar
Musical time vs. performance time

\sphinxAtStartPar
The simple two\sphinxhyphen{}dimensional model (sound vs. time) from the beginning of the book already
seems quite inadequate at this point where we have engaged with many different aspects of the
sound and pitch side of music. But also in the time domain we will see that the reality is more
complex than this simple picture conveys.


\section{Rhythm}
\label{\detokenize{4_time:rhythm}}
\sphinxAtStartPar
(Duration patterns)


\subsection{African rhythms}
\label{\detokenize{4_time:african-rhythms}}
\sphinxAtStartPar
including afro\sphinxhyphen{}american (Brazil, Cuba, …) Black Atlantic {[}\hyperlink{cite.8_bibliography:id6}{17}{]}.


\subsection{Balkan rhythms}
\label{\detokenize{4_time:balkan-rhythms}}

\subsection{Indian tala}
\label{\detokenize{4_time:indian-tala}}

\section{Meter}
\label{\detokenize{4_time:meter}}
\sphinxAtStartPar
Hierarchy… GTTM
\begin{itemize}
\item {} 
\sphinxAtStartPar
Discussion of long\sphinxhyphen{}short (middle?) ratios {[}\hyperlink{cite.8_bibliography:id8}{28}{]}

\item {} 
\sphinxAtStartPar
isochrony on several levels?

\item {} 
\sphinxAtStartPar
Loud rest

\item {} 
\sphinxAtStartPar
small integer ratios Desain and Honing {[}\hyperlink{cite.8_bibliography:id11}{13}{]}, Ravignani \sphinxstyleemphasis{et al.} {[}\hyperlink{cite.8_bibliography:id9}{50}{]}

\item {} 
\sphinxAtStartPar
Malian music: Polak \sphinxstyleemphasis{et al.} {[}\hyperlink{cite.8_bibliography:id7}{48}{]}

\item {} 
\sphinxAtStartPar
Learning of metrical categories (analogous to relativ pitch): {[}\hyperlink{cite.8_bibliography:id35}{21}, \hyperlink{cite.8_bibliography:id36}{61}{]}

\end{itemize}


\chapter{Harmony}
\label{\detokenize{4_harmony:harmony}}\label{\detokenize{4_harmony::doc}}\begin{itemize}
\item {} 
\sphinxAtStartPar
major\sphinxhyphen{}minor tonality

\item {} 
\sphinxAtStartPar
chords:
\sphinxhyphen{} root on (altered) diatonic scale degree
\sphinxhyphen{} stack of thirds
\sphinxhyphen{} chromatic alterations

\end{itemize}

\sphinxAtStartPar
First note based (Tone\sphinxhyphen{}based)

\sphinxAtStartPar
Then arrive at annotation standard (simple version)
and do simple regex filtering (find all Vs…)

\sphinxAtStartPar
Moss \sphinxstyleemphasis{et al.} {[}\hyperlink{cite.8_bibliography:id54}{39}{]}, Neuwirth \sphinxstyleemphasis{et al.} {[}\hyperlink{cite.8_bibliography:id76}{43}{]}


\section{Transformational theory}
\label{\detokenize{4_harmony:transformational-theory}}
\sphinxAtStartPar
We have already talked about GISs in \hyperref[\detokenize{1_fundamentals:sec-gis}]{Section \ref{\detokenize{1_fundamentals:sec-gis}}}


\chapter{Music Encoding}
\label{\detokenize{encodings:music-encoding}}\label{\detokenize{encodings::doc}}

\section{Audio}
\label{\detokenize{encodings:audio}}\begin{itemize}
\item {} 
\sphinxAtStartPar
WAV

\item {} 
\sphinxAtStartPar
MP3

\end{itemize}


\section{Common Western Music Notation}
\label{\detokenize{encodings:common-western-music-notation}}
\sphinxAtStartPar
Also known as the \sphinxstyleemphasis{score}.

\sphinxAtStartPar
Basic elements
\begin{itemize}
\item {} 
\sphinxAtStartPar
lines and spaces \sphinxhyphen{}\textgreater{} generic intervals

\item {} 
\sphinxAtStartPar
noteheads and beams \sphinxhyphen{}\textgreater{} durations

\item {} 
\sphinxAtStartPar
time signature \textendash{}\textgreater{} meter

\item {} 
\sphinxAtStartPar
key signature

\end{itemize}

\sphinxAtStartPar
lead sheets


\section{Symbolic}
\label{\detokenize{encodings:symbolic}}\begin{itemize}
\item {} 
\sphinxAtStartPar
MIDI

\item {} 
\sphinxAtStartPar
MusicXML

\item {} 
\sphinxAtStartPar
MEI

\item {} 
\sphinxAtStartPar
humdrum **kern

\item {} 
\sphinxAtStartPar
ABC

\item {} 
\sphinxAtStartPar
Volpiano

\item {} 
\sphinxAtStartPar
…

\end{itemize}


\section{Video}
\label{\detokenize{encodings:video}}
\sphinxAtStartPar
Normally, video is not taken into account when talking about music encodings.
But the visual aspect is very important, in particular when the music involves forms of
bodily expression such as dance or celebrations at a festival, as well as light shows.
In certain music genres, especially since the late 20th century, music and video
are inseparable.  (also research on Motion Capture {[}\hyperlink{cite.8_bibliography:id13}{11}{]}).


\chapter{Pitch\sphinxhyphen{}class based music analysis}
\label{\detokenize{5_notes:pitch-class-based-music-analysis}}\label{\detokenize{5_notes::doc}}
\sphinxAtStartPar
Musical pieces are made up of notes. This almost trivial truth is the
starting point for this endeavour: to uncover hidden patterns in
compositions dispersed through the centuries from the late middle ages
to modern times.

\sphinxAtStartPar
First, we will develop an understanding which notes exist and what their
mutual relations are. We will then proceed to use this information to
discover aspects of the structural relations between them and how the
constitute musical spaces (generalized interval systems).

\sphinxAtStartPar
We will see that musical notes in compositions are used far from
randomly and that the order encoded in musical pieces reveals
acquaintances between composers that sometimes transcends separation in
time.

\sphinxAtStartPar
Finally, we will see how the usage of musical notes changes over the
course of history. Observing large\sphinxhyphen{}scale changes in compositional style
we will be able to distinguish historical trends.

\sphinxAtStartPar
…

\sphinxAtStartPar
The study is based on the music as it is notated. Moreover, we assume
octave equivalence. These two assumptions\textendash{}one given by the
representation of notes in the data, the other based on theory and
previous literature\textendash{}leads to the Spiral Array representation of tonal
space.

\sphinxAtStartPar
Tonality is constituted by the ways composers navigate pitch space in
their compositions. This is, of course, hierarchal. Each piece
constitutes a unique instance of tonality which nonetheless shares many
properties with “similar” compositions (similar in style).

\sphinxAtStartPar
But what is pitch space?


\section{Frequency space}
\label{\detokenize{5_notes:frequency-space}}
\sphinxAtStartPar
A frequency \(f\) is measured in Hertz (1/sec). We can represent the
set of fundamental frequencies as
\(F=\{ K \cdot 2^o \cdot 3^q \cdot 5^t\mid o, q, t \in \mathbb Q \}\)
for a fixed \sphinxstyleemphasis{Kammerton} \(K\), and \(o, q, t\in \mathbb Q\). K
has been standardized to \(K=440~\text{Hz}\) in 1939. Hence, each
fundamental frequency can be represented as
\(f=440 \text{ Hz} \cdot 2^o \cdot 3^q \cdot 5^t\). Here, 2, 3, and
5 are arbitrary pairwise coprime integers (See Mazzola, 1985, p. 26) but
this choice relates to the numerators of the fundamental frequency
ratios of the music theoretical basic intervals of two frequencies
\(f\) and \(g\) in just intonation:
\begin{itemize}
\item {} 
\sphinxAtStartPar
octave: \(f:g = 2:1\)

\item {} 
\sphinxAtStartPar
fifth: \(f:g=3:2\)

\item {} 
\sphinxAtStartPar
major third: \(f:g = 5:4\)

\end{itemize}

\sphinxAtStartPar
The fact that \(o, q, t\) are \(\in \mathbb Q\) allows for
unique solution of the equations
{[}\hyperlink{cite.8_bibliography:id46}{29}, \hyperlink{cite.8_bibliography:id45}{31}, \hyperlink{cite.8_bibliography:id4}{32}{]}.


\section{Pitch space}
\label{\detokenize{5_notes:pitch-space}}
\sphinxAtStartPar
\sphinxstyleemphasis{Pitch} (or \sphinxstyleemphasis{pitch height}) is the perceptual correlate to a fundamental
frequency and is measured on a linear scale. It can be calculated as
\(p(f) = 69.0 + 12 \cdot \log\_2(f/440 \text{ Hz}) \), and 12 is the
size of the octave in the unit of \(p\). The formula thus gives the
distance from the reference pitch (69=A4) as a proportion
(\(\log_2(f/g)\)) of the octave (12). In this representation middle
C (C4) gets the pitch number 60.
\begin{equation*}
\begin{split}.. \begin{aligned}
 p(f) & = c_1+c_2\cdot \log_2\left(\frac{440 \text{ Hz}\cdot 2^o \cdot 3^q \cdot 5^t}{440\text{ Hz}}\right)\\
      & = c_1+c_2\cdot \log_2\left(2^o\cdot3^q\cdot5^t\right) \\
            & = c_1+c_2 \cdot (o\cdot\log_2(2) + q\cdot \log_2(3) + t\cdot\log_2(5) ).
.. \end{aligned}\end{split}
\end{equation*}
\sphinxAtStartPar
Equivalently,
\begin{equation*}
\begin{split}\begin{aligned}
  p(f)- c_1 = c_2 \cdot (o\cdot\log_2(2) + q\cdot \log_2(3) + t\cdot\log_2(5) )\end{aligned}\end{split}
\end{equation*}
\sphinxAtStartPar
Accordingly, each frequency \(f\) or equivalently each tone
\(t\) can be conceived as a point
\(x = (p, s, r) \in \mathbb Q^3\). And vice versa, each point is
associated with its fundamental frequency
\(f(x) = f(p, s, r) = 440 \text{Hz} \cdot 2^p \cdot 3^s \cdot 5^r\).
In this representation,
\(p = o \cdot \log_2(2) + q\cdot \log_2(3) + t\cdot \log_2(5)\) is a
linear combination from the basis vectors over
\(\mathbb Q^3 = \mathbb T^3\) with coefficients \(o, q, t\), and
\(\mathbb T^3\) is a module over the ring \(\mathbb Z\) of
integers. Another way to state this is that octave, fifth and major
third are the basis vectors of tonal space.


\section{Transformations of tonal spaces.}
\label{\detokenize{5_notes:transformations-of-tonal-spaces}}
\sphinxAtStartPar
For a fixed (arbitrary) reference tone (e.g. chamber tone A4=440Hz), and
for integers \(o, q, t \in \mathbb Z\) we can represent all tones in
just intonation. The integer lattice
\(\mathbb Z^3 \subseteq \mathbb Q^3\) corresponds to Euler’s
representation {[}\hyperlink{cite.8_bibliography:id5}{14}{]} and we will denote it with
\(\mathbb T^3\), the tonal space incorporating the three dimensions
of the octave \(o\), the pure fifth \(q\) and the pure major
third \(t\).

\sphinxAtStartPar
Usually, descriptions of musical space do not consider octaves and
octave\sphinxhyphen{}related tones are subsumed into equivalence classes (chroma
classes or pitch classes); \(\pi_O: \mathbb T^3 \to \mathbb T^2\) ;
\((o, q, t) \mapsto (q, t)\)

\sphinxAtStartPar
Notes are abstract symbolic representations of tones. They can be
modeled as pairs \(n=(p, d)\in \mathbb T^3 \times \mathbb R\) where
\(p\) encodes \sphinxstyleemphasis{pitch} and \(d\) \sphinxstyleemphasis{duration}. The pitch
dimension can be represented in one of the tonal pitch spaces (TPS).


\section{Generalized Interval Systems}
\label{\detokenize{5_notes:generalized-interval-systems}}\begin{itemize}
\item {} 
\sphinxAtStartPar
(GIS) maps a TPS representation to an appropriate mathematical space

\item {} 
\sphinxAtStartPar
A \sphinxstyleemphasis{Generalized Interval System} (GIS) is an ordered triple
\((S, (G, \circ), \text{int})\), where \(S\) is a musical
space, \((G, \circ)\) is a group, and \(\text{int}\) is an
interval function that maps \(S \times S \to G\). Common
instances for \(G\) are \((\mathbb Z, +)\) (suitable for the
line of fifth) or \((\mathbb{Z}_{12}, +)\) (suitable for the
circle of fifths).

\item {} 
\sphinxAtStartPar
…. many names: Pitch Space, Tonal Space, Tonal Pitch Space, Music
Space, Musical Space….

\end{itemize}

\sphinxAtStartPar
These models of tonal space (line of fifths, circle of fifths, tonnetz,
torus) can serve as support for probability distributions. These in turn
describe the generative process for tonal pieces.

\sphinxAtStartPar
For the scope of this dissertation, the bag\sphinxhyphen{}of\sphinxhyphen{}notes model is assumed.
Meaning, that the grammar \(G\) responsible for the sequential
arrangement of notes is factored out.

\begin{sphinxadmonition}{important}{Important:}
\sphinxAtStartPar
Problem: “Theorizing in the wrong space” {[}\hyperlink{cite.8_bibliography:id10}{69}{]}
\end{sphinxadmonition}

\sphinxAtStartPar
Pitch space encompasses the pitches and their mutual relations, the
intervals. Certain assumptions about pitches transform pitch space.

\sphinxAtStartPar
There are numerous theoretical models of pitch space.

\begin{DUlineblock}{0em}
\item[] \sphinxhyphen{} Euler space
\item[] \sphinxhyphen{} Tonnetz
\item[] \sphinxhyphen{} Line of fifths
\item[] \sphinxhyphen{} Circle of fifths
\item[] \sphinxhyphen{} Spiral Array
\item[] \sphinxhyphen{} MIDI
\end{DUlineblock}


\section{Models of pitch space}
\label{\detokenize{5_notes:models-of-pitch-space}}
\sphinxAtStartPar
Pitches can be expressed as \(2^x3^y5^z\) for
\(x,y,z\in \mathbb Q\) “Fundamental theorem of harmony”
{[}\hyperlink{cite.8_bibliography:id46}{29}, \hyperlink{cite.8_bibliography:id47}{30}, \hyperlink{cite.8_bibliography:id45}{31}{]}.
Pitches thus form a
3\sphinxhyphen{}dimensional space, also called the Euler space that incorporates just
intonation (pure integer ratios of frequencies). Distances between
pitches in this space are called intervals. Music theorists consider a
number of equivalence relations that transform the space. The most
common equivalence relation is octave equivalence that identifies all
pitches that are related by frequency ratios of 2, effectively
projecting the 3\sphinxhyphen{}dimensional Euler space to the plane given by
\(3^y5^z\). This plane is commonly called the Tonnetz and has
numerous historical precursors in the 19th century.

\sphinxAtStartPar
Since the Tonnetz expresses just intonation, one can distinguish, for
instance, between the just third E above C and the Pythagorean third E’
that lies four fifths. The difference between the just and the
Pythagorean third is called the syntonic comma,
\begin{equation*}
\begin{split}\begin{aligned}
    4/5 : (2/3)^4 &= 4/5 : 16/81\\
                                &= 4/5 \cdot 81/16\\
                                &= 81/80 \\
                                \approx 1.0125.\end{aligned}\end{split}
\end{equation*}
\sphinxAtStartPar
Identifying just and Pythoagorean thirds wraps the Tonnetz to a
cylinder, also called the \sphinxstyleemphasis{Spiral Array}
{[}\hyperlink{cite.8_bibliography:id41}{7}, \hyperlink{cite.8_bibliography:id43}{8}{]}

\sphinxAtStartPar
{[}Tikz Picture of Spiral Array{]}

\sphinxAtStartPar
On this cylinder, the line of fifths wraps around in such a way that
every fourth fifth coincides with a third. This also means that all
points on this cylinder lie on this line of fifths. The pitches in this
space are sometimes called tonal pitch classes
{[}\hyperlink{cite.8_bibliography:id44}{37}, \hyperlink{cite.8_bibliography:id40}{58}, \hyperlink{cite.8_bibliography:id31}{59}{]}.
The line of
fifths is sufficient to capture all TPCs but the 2\sphinxhyphen{}dimensional surface
of the cylinder emphasizes triadic relations. Moreover, a segment of six
fifths contains all notes of a major or a (natural) minor scale and
hence all pitches and intervals in a key. The triads within a key form
Moebius strip {[}\hyperlink{cite.8_bibliography:id45}{31}, \hyperlink{cite.8_bibliography:id42}{44}{]}.
Closing this segment to a circle involves only diatonic fifths, but one of them is
diminshed (B\textendash{}F in C major). This pitch set can be mapped to
\(\mathbb Z_7\).

\sphinxAtStartPar
{[}FIGURE: Diatonic chord sequence C\sphinxhyphen{}a\sphinxhyphen{}F\sphinxhyphen{}d\sphinxhyphen{}bo\sphinxhyphen{}G\sphinxhyphen{}e\sphinxhyphen{}C{]}

\sphinxAtStartPar
{[}FIGURE: Moebius strip embedded in \(\mathbb Z_7\){]}

\sphinxAtStartPar
Finally, another importent equivalence relation is that of enharmonic
equivalence. Enharmonic equivalence identifies octaves and augmented
sevenths,
\begin{equation*}
\begin{split}\begin{aligned}
    (1/2)^7 : (2/3)^12 \approx 1.0136.\end{aligned}\end{split}
\end{equation*}
\sphinxAtStartPar
This equivalence relations transforms the cylindrical pitch space to a
torus, and the line wrapped around the cylinder to a circle, the circle
of fifths. The tonal pitch classes are transformed into neutral pitch
classes, or simply pitch classes.

\sphinxAtStartPar
The pitch classes on the circle of fifths can be reorderd to the
chromatic circle by
\begin{equation*}
\begin{split}p\mapsto 7p\mod 12,\end{split}
\end{equation*}
\sphinxAtStartPar
resulting in the order on which the keys within one octave occur in the
piano. Both the chromatic circle and the circle of fifths can be
identified with \(\mathbb Z_{12}\).

\sphinxAtStartPar
Especiall, review line of
fifths, center of gravity and describe it in the language of
distributions. Discuss also, why D is the center and not C!
\begin{itemize}
\item {} 
\sphinxAtStartPar
distinction tonal/spelled vs. (neutral) pcs

\item {} 
\sphinxAtStartPar
explain mapping to \(\mathbb Z_{12}\)

\item {} 
\sphinxAtStartPar
sharp tpcs are mapped to positive numbers, flat tpcs to negative numbers

\item {} 
\sphinxAtStartPar
already by this definition, the white\sphinxhyphen{}note diatonic is more sharp than flat (and not balanced!)

\end{itemize}


\subsection{The bag\sphinxhyphen{}of\sphinxhyphen{}notes model}
\label{\detokenize{5_notes:the-bag-of-notes-model}}
\sphinxAtStartPar
The bag\sphinxhyphen{}of\sphinxhyphen{}notes model conceives pieces simply counts the occurrences of
notes without taking into account the order in which they appear in the
piece.

\sphinxAtStartPar
It is in this sense a much more general model than the theoretically
motivated ones that we have seen in the previous section. This model
does not make any specific assumptions about the relations between notes
other than that their respective frequency is relevant.

\sphinxAtStartPar
In the terminology of probability theory, relative note frequencies
derived from note counts under the bag\sphinxhyphen{}of\sphinxhyphen{}words model correspond to
multinomial distributions.

\sphinxAtStartPar
Take for example the first movement of Alkan’s \sphinxstyleemphasis{Concerto for Solo
Piano}, op. 39, No. 8, in G\(\sharp\) minor. Figure
{[}fig:Alkan\_39\sphinxhyphen{}8\_freqs{]} shows the note counts, weighted by duration
(CHECK) and ranked by frequency.

\sphinxAtStartPar
One can see an almost linear relation between frequency and rank.

\sphinxAtStartPar
Only towards the end of the movement the key signature changes from five
sharps (G sharp minor) to four flats Ab major but even in the sharp
parts of the movement the notated score changes to flats where
convenient.

\sphinxAtStartPar
Since musical pieces can have very different lengths\textendash{}some pieces last
only a few minutes while others may last more than an hour\textendash{}it is useful
to normalize the note counts and to derive the relative frequencies.

\sphinxAtStartPar
Interpret as distributions, show pitchplots… compare to Figure
{[}fig:tonal\_spaces{]}.

\sphinxAtStartPar
These and many more conceivable transformations of tonal space do not
serve the goal of merely reflecting abstract algebraic or geometric
relations. It is important to emphasize that these transformations
reflect rather practical assumptions for performance and instrument
construction (such as dealing with the syntonic comma for keyboard
instuments) or compositional decisions (such as enharmonic equivalence).


\section{Dataset}
\label{\detokenize{5_notes:dataset}}
\begin{DUlineblock}{0em}
\item[] MusicXML files from diverse sources… \sphinxhyphen{} musescore.com
\item[] \sphinxhyphen{} ELVIS
\item[] \sphinxhyphen{} Humdrum
\item[] \sphinxhyphen{} DCML transcriptions
\item[] \sphinxhyphen{} CPDL
\item[] \sphinxhyphen{} other websites…
\end{DUlineblock}

\sphinxAtStartPar
In this chapter we look at the historical development of tonality.
Although the dataset contains ca. 2000 pieces, there are unfortunately
huge gaps in the timeline as can be seen in Figure {[}fig:piece\_dist{]}.
Attributing one year to a piece is not easy, in particular for older
pieces. If available, we use the year of composition, otherwise the year
of publication. Where both dates were unavailable, the middle year of
the composer’s life was chosen to represent the piece. Following this
procedure leads to only 157 years for which we have pieces in the whole
range of 582 years from 1361 to 1942.

\sphinxAtStartPar
If the hypothesis is true that tonality is constituted by the pitch
usage in pieces and that certain compositional assumptions transform
pitch space, then it should be possible to discover aspects of these
assumptions and the structure of pitch space by analyzing the usage of
pitches in musical compositions.

\sphinxAtStartPar
Moreover, comparing different sets of pieces, e.g. from different time
periods or composers, should reveal historical and stylistic
differences.


\chapter{Tonal pitch\sphinxhyphen{}class distributions}
\label{\detokenize{pc_distributions:tonal-pitch-class-distributions}}\label{\detokenize{pc_distributions::doc}}
\sphinxAtStartPar
The tonal pitch\sphinxhyphen{}class distribution of a musical piece is the relative
frequency of each tonal pitch class in that piece. Each piece can thus
be represented as a \(V\)\sphinxhyphen{}dimensional vector, where \(V\) is the
number of different pitch classes in the corpus, and that sums to one.
In this view, pieces are points in a \(V\)\sphinxhyphen{}dimensional vector space.

\sphinxAtStartPar
In this space, pieces that have similar tonal pitch\sphinxhyphen{}class distributions
will be close together whereas pieces with very different tonal
pitch\sphinxhyphen{}class distributions will be more distant.

\sphinxAtStartPar
If all pieces are transposed to the same root, clusters in this space
correspond to different types of distributions that can be interpreted
as modes (take root out). This fact has been used in
{[}\hyperlink{cite.8_bibliography:id74}{22}{]}
and also shown that there are historical developments.

\sphinxAtStartPar
If one does not transpose pieces, pieces that have similar root and mode
(and, accordingly, similar distributions) should cluster together. Since
\(V\) is usually quite large, it is difficult to visualize these
clusters. One can use methods for dimensionality reduction to represent
the data in lower\sphinxhyphen{}dimensional spaces (2D or 3D) in order to visualize
them while at the same time maintaining characteristic properties of the
original space.

\sphinxAtStartPar
One of the most popular and classic methods is \sphinxstylestrong{principal component
analysis} {[}\hyperlink{cite.8_bibliography:id70}{3}{]}, that can be used to project
the data onto a two\sphinxhyphen{}dimensional plane while keeping as much of the
variance in the data as possible. A more recent method for
dimensionality is called \(t\)\sphinxhyphen{}distributed stochastic neighbor
embedding (\(t\)\sphinxhyphen{}SNE) {[}\hyperlink{cite.8_bibliography:id72}{63}{]}. PCA is
better to get a global understanding of the structure of the space and
\(t\)\sphinxhyphen{}SNE is better in illustrating local relationships. Figure
{[}fig:tsne\_pca{]} shows the data reduced to the Euclidean plane by both
methods.

\sphinxAtStartPar
A more modern method is UMAP {[}\hyperlink{cite.8_bibliography:id71}{33}{]}, much faster but conceptually different!

\sphinxAtStartPar
The reduction using \(t\)\sphinxhyphen{}SNE (top panel) shows that there are many
clusters that are relatively homogenuous with respect to their coloring.
The PCA reduction on the bottom panel of Figure {[}fig:tsne\_pca{]} also
shows that pieces with similar coloring are close together but
additionally shows that the colors are ordered along the line of fifths.
This means that pieces in keys that are close on the line of fifths have
similar tonal pitch\sphinxhyphen{}class distributions. Another advantage of PCA is
that the axes, called the principal components, have clear
interpretations. They reflect how much the data varies in this
direction. Applying this interpretation to the right panel of Figure
{[}fig:tsne\_pca{]}, one can see that the first principal component (“PC1”)
roughly represents the “distance to C” or “diatonic” pieces (white or
very light colors) of more chromatic ones (darker shades). This
distinction accounts for 55 percent of the variance in the data. The
second principal component (“PC2”) distinguishes sharp from flat keys
(red vs. blue coloring) which is responsible for 21 percent of the data
variance.

\sphinxAtStartPar
These two principal components together account for 76 percent of the
variance in the data but simplify the space from \(V=35\) dimensions
to just two which seems like a good tradeoff.


\chapter{Historical usage of tonal pitch classes}
\label{\detokenize{history:historical-usage-of-tonal-pitch-classes}}\label{\detokenize{history::doc}}
\sphinxAtStartPar
Apart from counting the number of tonal pitch classes in an individual
musical piece, comparing these distributions between pieces and across
historical time is interesting. In the last section we compared a small
number of pieces manually. This section attempts at quantifying these
intuitions and gaining a picture of the larger view.

\sphinxAtStartPar
The question is, how does the usage of tonal pitch classes change over
time? Can we infer something about tonality from this change? An
immediate caveat that comes to mind is that pieces often feature very
different sets of notes because they are, for instance, in a different
mode (both in the pre\sphinxhyphen{}tonal as well in the tonal sense), or key. It is
therefor a standard preprocessing step in computational musicology to
normalize pitch class distributions by transposing every piece to the
same key in order to make them commensurate. For the same reasone, the
chord symbols in the datasets analyzed in part {[}part:meso{]} where encoded
with relative Roman numerals and not their absolute chord names. But in
order to perform this normalization step, one needs to know the key of a
piece. \sphinxstylestrong{(Well, not really: Harasim et al. 2019)} Moreover, the concept
of “key” does not mean the same thing for all musical styles. Bach’s
B\sphinxhyphen{}minor Mass and Liszt’s B\sphinxhyphen{}minor Sonata share the same nominal key but
differ greatly with respect to their pitch\sphinxhyphen{}class distributions. Since
the underlying tonality has changed, the derivative concept of key has
changed, too. And just identifying B as the most common note in both
pieces as indicative for the key (\sphinxstylestrong{check if that is the case}) is not
a solution either because this procedure would also identify Renaissance
locrian pieces as having the same key without even having touched the
problem of how to infer the mode.

\sphinxAtStartPar
We come back to this issue in later chapters (\sphinxstylestrong{WHERE?}). Maybe it is
appropriate to inspect the absolute pitch distributions of pieces before
delving into the issue of relative pitch classes. This is what this
chapter is about.


\section{Modeling tonal pitch\sphinxhyphen{}class evolution}
\label{\detokenize{history:modeling-tonal-pitch-class-evolution}}
\sphinxAtStartPar
Tonal pitch classes on the line of fifths can be mapped to integers
\(k \in \mathbb Z\). An interval \(I=[a,b]\subseteq\mathbb Z\)
is called a \sphinxstylestrong{line\sphinxhyphen{}of\sphinxhyphen{}fifths segment} and its length is
\(n=|b-a|, a<b\). The distribution of tonal pitch classes at time
\(t\) (in a piece or in historical time) is modeled as a draw from a
Dirichlet distribution:
\begin{equation*}
\begin{split}X^{(t)}\sim \mathrm{Dir}(\mathbf{\alpha}), \mathbf{\alpha}\in\mathbb R^n.\end{split}
\end{equation*}
\sphinxAtStartPar
Importantly, in this model, the dimensions of \(X^{t}\) have no
inherent order. This means that the model knows nothing about the line
of fifths anymore. The ordering of pcs along this line is just for
convenience. The probability of the pitch class \(k = i-a\) at time
\(t\) is given by the \(k+a\)th component of the vector
\(X^{(t)}\), \(p(k | t)=X_{k+a}^{t}\) The diachronic change of
these distributions forms a process
\begin{align*}\!\begin{aligned}
\mathbf{X}=(X^{(1)},\dots,X^{(t)},\dots,X^{(T)})\in \mathbb R^{n\times T},\\
such that :math:`\sum_i X_i^{(t)}=1,\forall t`.\\
\end{aligned}\end{align*}

\section{Variability in tonal pitch\sphinxhyphen{}class usage}
\label{\detokenize{history:variability-in-tonal-pitch-class-usage}}
\sphinxAtStartPar
We count the occurrence of tonal pitch classes in all pieces and trace
the change between them across the historical timeline. Based on
theoretical reasoning {[}\hyperlink{cite.8_bibliography:id63}{18}, \hyperlink{cite.8_bibliography:id40}{58}{]},
we have already seen in section {[}sec:bagofwords{]} that it seems to be the
case that sorting pitch classes along the line of fifths reveals
structural connections between the pitch classes. For that reason we
plot the pitch classes along this axis and also use colors to encode
this relation.

\sphinxAtStartPar
As Figure {[}fig:piece\_dist{]} has shown, the dataset is not uniformly
distributed over time. On one hand, there are some large gaps between
periods, whereas on the other hand some years contain many pieces at the
same time.

\sphinxAtStartPar
For years without data, we take the assumption that “nothing changes”
and keep the values from the last where were data was available. For the
years with many pieces, we add up the pitch class counts, so that they
all contribute to the calculation.

\begin{DUlineblock}{0em}
\item[] 
\item[] …
\item[] A \sphinxstylestrong{rolling mean}, also called a moving average, is calculated over
the whole historical range. It is common that sliding windows are
centered. But because it makes more sense for historical data to only
consider previous events because future events have no impact, the
result of the sliding window takes into account all \(t\) previous
years.
\end{DUlineblock}

\sphinxAtStartPar
For a value \(x_t\) in year \(t\), and window size \(s\),
the rolling mean \(\bar x\) is defined as
\begin{equation*}
\begin{split}\begin{aligned}
    \bar x = \frac{1}{s}\sum_{i=0}^{s-1} x_{t-i}.\end{aligned}\end{split}
\end{equation*}
\sphinxAtStartPar
This definition allows a scalable perspective on historical
developments. Adjusting the windows size allows to all historical
periods in the range of the historical frame under consideration. For
instance, setting \(s=50\) will lead to a curve that at any point
represents the average value of the last 50 years, if years are the unit
of time.

\sphinxAtStartPar
This is done for the tonal pitch\sphinxhyphen{}class distributions of aggregated
pieces and is shown in Figure {[}fig:evolution\_tpcs{]}. It is a complex
plot and we will discuss each part at a time.

\sphinxAtStartPar
The legend above the two subplots show the mapping of tonal pitch
classes to colors. Since tpcs are isomorphic to \(\mathcal Z\), as
mentioned above, it is possible to map flat tpcs to negative numbers,
shown as graded blue colors, and to map sharp tpcs to positive numbers,
shown as graded red colors. The tpc C is mapped to zero which
corresponds to the color white in this plot.

\sphinxAtStartPar
The plot immediately below the legend shows the smoothed distribution of
tonal pitch classes over time, sorted by the associated colors. The two
dashed curves demarcate the white\sphinxhyphen{}note diatonic tonal pitch classes F to
B. It is important to note here that in the bag\sphinxhyphen{}of\sphinxhyphen{}notes model tonal
pitch classes are expressed as multinomial distributions. This means
that there is no inherent order to the pitch classes\textendash{}there is no
structure in the bag. The coloring and sorting is done on theoretical
grounds, but we will soon see that this ordering makes also sense for
the data at hand.

\sphinxAtStartPar
The dark line throughout this plot shows the normalized entropy of the
pitch\sphinxhyphen{}class distributions at any point in time. This line is smoothed by
the same procedure as the individual per\sphinxhyphen{}year pitch\sphinxhyphen{}class distributions
and is thus an adequate measure for the randomness of these
distributions for a given year. Taking into account a 50\sphinxhyphen{}year window
shows that randomness slightly increases over time with some wiggles
along the way. The value of this line is independent of the number of
tonal pitch classes in a given year, since it is normalized by its
maximal value which is given by \(\log(n)\) where \(n\) is the
number of non\sphinxhyphen{}zero tonal pitch classes in that year.

\sphinxAtStartPar
The red line in the bottom plot in Figure {[}fig:evolution\_tpcs{]} shows
the ratio of non\sphinxhyphen{}zero “sharp” tpcs (G, D, A, …) to non\sphinxhyphen{}zero “flat”
notes (F, B\(\flat\), E\(\flat\), etc.), defined as
\(q=s/f\), where \(s\) is the number of sharp tpcs (not unique
but the actual number), and \(f\) is the number of flat notes. If
\(f=0\), the ratio \(q\) is not defined. Since the analysis is
based on the moving average, as well, a piece with no flats (which
implies also F) is excluded. Since the window size is considerably
large, there is no sliding window that contains only pieces with
non\sphinxhyphen{}flat notes so that \(q\) is always defined as can be seen by the
smoothness of the red line. As can be seen, this curve shows
considerable variation. In both subplots, saddle points correspond to
regions where no data is available so no interpretation should be given
for these areas.

\sphinxAtStartPar
\sphinxstylestrong{Bootstrap sample CIs!}

\sphinxAtStartPar
If a musical piece exclusively contains the seven diatonic tpcs, and if
they are furthermore uniformly distributed in this piece, the
sharp\sphinxhyphen{}to\sphinxhyphen{}flat ratio is \(q=|\{G, D, A, E, B\}/|F|=5\). Which is
exactly what we see in the beginning of our timeline.

\sphinxAtStartPar
The reverse statement that diatonic notes are uniformly distributed if
the ratio is \(q=5\) is not necessarily true. In fact, there are
non\sphinxhyphen{}diatonic notes present at the beginning of the timeline, namely
B\(\flat\) in the flat direction, and F\(\sharp\) and
C\(\sharp\) in the sharp direction. A uniform ratio would be then
\(q=|\{G, D, A, E, B, F\sharp, C\sharp\}/|B\flat, F|=3.5\) So we can
rule out uniformity, also because the entropy (the black line in the
upper plot) is not maximal. The question is, whether the non\sphinxhyphen{}randomness
in these distribution tells us something about tonality and its
historical development. We come back to this question later.

\sphinxAtStartPar
The smoothed trends in both subplots show that sharpward tpcs are
generally much more common if not only because all diatonic pitches are
already sharps except F. More precisely, sharp notes occur roughly five
times more often than flat notes until the last quarter of the fifteenth
century. This might be due to the fact that almost only diatonic notes
are being used, with relatively constant but low B\(\flat\)s
\sphinxstylestrong{(transposed modes!)}. On the sharp side of the spectrum,
F\(\sharp\)s occur rarely, as do C\(\sharp\)s which
lends itself to the interpretation that these notes do reflect the
\sphinxstylestrong{musica ficta}. Other accidentals occur vanishingly seldom.

\sphinxAtStartPar
Around 1460 there is a decline in \(q\) that stabilizes around 1530
where the sharps occur only three times as often as flats. This is due
to an increased use of flat notes F and B\(\flat\). Somewhat
surprisingly, F\(\sharp\) and higher sharps are absent in this
period. But for modal music that is a logical consequence. If the
transposed modes are used more often, sharp notes are less likely to
occur.

\sphinxAtStartPar
In the second half of the 16th century, E\(\flat\) appears for the
first time in the corpus in a substantial and stable way. But also
F\(\sharp\) comes back so it is counterbalanced and the ratio
stays roughly the same.

\sphinxAtStartPar
Towards the end of the 16th century, we see a dramatic increase in the
sharp\sphinxhyphen{}flat ratio that continues until the middle of the 17th century and
reaches a more than 7\sphinxhyphen{}fold peak. This is due to the disappearance of
almost all flats below B\(\flat\), while the sharps
C\(\sharp\), G\(\sharp\), and D\(\sharp\) become even
stronger (and never vanish again). In this period, music seems to shift
to the sharp side. While modal music featured the basid diatonic modes
plus downward transposition to the flat side by one, here we see more
and more accidentals.

\sphinxAtStartPar
…going into dominant regions means going sharpwards.

\sphinxAtStartPar
But this peak lasts only shortly. Around 1700 the sharp\sphinxhyphen{}flat ratio has
fallen back to its earlier point around 2.5. But although the ratio is
the same, the tpc usage is quite different. Now many more sharps and
flats are employed than ever before. More importantly, this peak marks
the beginning of the Baroque period. The first Baroque composer in the
corpus is Corelli (also the most frequent one). There are a lot of
pieces from him at the end of the 17th century.

\sphinxAtStartPar
A surge of flats around 1800 brings the ratio down to its lowest point
since ca. 1530 and remains relatively stable throughout the 19th
century. There is a slight rise and decay over the course of this
century. Both sharps and flats increase in this time but more so do the
flats.

\sphinxAtStartPar
In the early 20th century there is the third lowest point where flats
dominate sharps (“renaissance of the Renaissance”? Vaughan Williams,
Finzi, …)


\chapter{Tonal pitch\sphinxhyphen{}class coevolution}
\label{\detokenize{coevolution:tonal-pitch-class-coevolution}}\label{\detokenize{coevolution::doc}}

\section{Modeling tonal pitch\sphinxhyphen{}class coevolution}
\label{\detokenize{coevolution:modeling-tonal-pitch-class-coevolution}}
\sphinxAtStartPar
The change/evolution of each pitch class \(k=i-a\) is given by the
changes in
\(\mathbf{X}_i=(X^{1}_i,\dots,X^{t}_i,\dots,X^{T}_i)^\top\in \mathbb{R}^{T}\).
The pitch\sphinxhyphen{}class coevolution matrix is given by
\begin{equation*}
\begin{split}\mathbf\Sigma=\left(\mathrm{corr}(\mathbf{X}_i, \mathbf{X}_j)\right)_{ij}\in[-1,1]^{n\times n}\end{split}
\end{equation*}
\sphinxAtStartPar
and reflects the similarity of the diachronic change of pitch\sphinxhyphen{}classes.

\sphinxAtStartPar
These upper subplot in Figure {[}fig:evolution\_tpcs{]} have shown the
changes in the usage of each pitch\sphinxhyphen{}class over time. The coloring and
ordering suggests indeed a coevolution but recall that the ordering was
put in manually. The question is whether we can learn something about
the structure from the data by analyzing the coevolution of the tpcs
which is operationalized as the pairwise correlation (the Pearson
correlation coefficient \(\rho\)) (maybe use sample coefficient
\(r\)?) of two pc\sphinxhyphen{}evolution vectors \(p\) and \(q\):
\begin{equation*}
\begin{split}\begin{aligned}
\rho_{p,q} = \frac{\mathrm{cov}(p,q)}{\sigma_p\sigma_q},\end{aligned}\end{split}
\end{equation*}
\sphinxAtStartPar
where \(\mathrm{cov}(p,q)\) is the covariance and \(\sigma\)
the standard deviation. Figure {[}fig:coevolution\_tpcs{]} shows the
pairwise tonal pitch class coevolution values across the entire
timeline.

\sphinxAtStartPar
Interesting observations:
\begin{enumerate}
\sphinxsetlistlabels{\arabic}{enumi}{enumii}{}{.}%
\item {} 
\sphinxAtStartPar
Three regimes are clearly separated: flats (upper left), diatonics
(center), and sharps (lower right)

\item {} 
\sphinxAtStartPar
The chromatic regimes are of roughly the same size, (only visible in
overall plot; the sharps are slightly larger), i.e. the heatmap has
two orthogonal symmetry axes

\item {} 
\sphinxAtStartPar
Moreover, the chromatic notes (flats and sharps) are weakly
positively correlated

\item {} 
\sphinxAtStartPar
F\(\flat\flat\) (and more extreme flats) does not occur in the
entire corpus

\item {} 
\sphinxAtStartPar
The weakest correlations are highly interesting as well: The weakest
correlation is with the chromatic lower neighbor and the tritone
(e.g. A vs. A\(\flat\), E\(\flat\); E vs
E\(\flat\), B\(\flat\); B vs. B\(\flat\), F;
F\(\sharp\) vs. F, C) This is only true for “central” tpcs
(white keys diatonic)

\end{enumerate}

\sphinxAtStartPar
We can use this correlation matrix to plot distances between the pitch
classes. Restricting the relations to the center of the plot, the
diatonic notes plus F\(\sharp\) and B\(\flat\) these
distances actually approximate the line of fifths!


\section{Deciphering pitch\sphinxhyphen{}class coevolution}
\label{\detokenize{coevolution:deciphering-pitch-class-coevolution}}
\sphinxAtStartPar
The last section presented how strong the evolution of pitch classes
correlates with each other. The heatmap in Figure
{[}fig:coevolution\_tpcs{]} indicated an interesting connection to the
ordering of tpcs on the line of fifths. But this ordering was achieved
manually, based on theoretical knowledge. How strong is this connection
based on the available data?

\sphinxAtStartPar
One way to investigate this is to reduce the high\sphinxhyphen{}dimensional space to a
smaller one. A common method to achieve this is \sphinxstylestrong{principal component
analysis} (PCA). PCA analyzes the variance in the data and projects the
data to a lower\sphinxhyphen{}dimensional space while maximizing the retained
variance.

\sphinxAtStartPar
Subsequently, one can inspect the individual principal components
individually and interpret the variance within and between them.

\sphinxAtStartPar
The results are very interesting:
\begin{enumerate}
\sphinxsetlistlabels{\arabic}{enumi}{enumii}{}{.}%
\item {} 
\sphinxAtStartPar
Roughly, 64\% of the variance is explained with the diatonic\sphinxhyphen{}chromatic
distinction (PC1)

\item {} 
\sphinxAtStartPar
About 22\% is explained by the sharp\sphinxhyphen{}flat distinction (PC2). Note also
that C is on the zero\sphinxhyphen{}line for PC2 (does this really mean
something?).

\item {} 
\sphinxAtStartPar
Another 6\% of variance is explained by the third principal component.
It roughly corresponds to the numbers of accidentals and follows,
approximately, a zig\sphinxhyphen{}zag pattern for the 5 regions
\(\flat\flat\), \(\flat\), ”, \(\sharp\), and
\(\sharp\sharp\).

\item {} 
\sphinxAtStartPar
PC4 is not easy to interpret, but it still captures a difference
between, flat, diatonic, and sharp tpcs. Indeed, it seems that this
component captures enharmonic equivalence! The tpcs C, G, D, A, was
well as their enharmonic sharp and flat equivalents are all separated
from the other notes. The same goes for F and E\# (but not Gbb).

\end{enumerate}

\sphinxAtStartPar
It seems that the PCA reduction was not only able to capture meaningful
dimensions, but also a meaningful relation between them, namely the
hierarchical one depicted in Figure {[}fig:pca\_hierarchy{]}.

\sphinxAtStartPar
The variance explained by each of the components can be interpreted as
the weight or importance of these dimensions for the data. The two most
important principal components are PC1 and PC2, together contributing
approximately 86\% of variance to the data. Figure {[}fig:PCA\_2dim{]} shows
how these two dimensions interact. Largely speaking, diatonic and
chromatic tpcs can be separated by a vertical line (not exactly),
whereas sharpward and flatward tpcs can be separated by a horizontal
line, with C, the only tpc that is neither flatwards nor sharpwards,
being exactly on the axis. Moreover, the three respectively most extreme
tpcs, Fbb\textendash{}Gbb and Ax\textendash{}Bx, are located close to the origin of the PCA
transformed plot. This means that they do not contribute much to the
variance in the data. These are also precisely the ones outside of the
enharmonic equivalence shown in PC4.


\section{TPC coevolution per historical period}
\label{\detokenize{coevolution:tpc-coevolution-per-historical-period}}
\sphinxAtStartPar
This “global view” can be broken down to compare how the tpc
correlations change over time. The next figure shows the correlations
for 50\sphinxhyphen{}year periods
\begin{enumerate}
\sphinxsetlistlabels{\arabic}{enumi}{enumii}{}{.}%
\item {} 
\sphinxAtStartPar
1500\sphinxhyphen{}1550: Two clusters emerge

\item {} 
\sphinxAtStartPar
1550\sphinxhyphen{}1600: Clear separation between recta and ficta.

\item {} 
\sphinxAtStartPar
1600\sphinxhyphen{}1650: Dahlhaus situates the origin of harmonic tonality in the
early 17th century (Untersuchungen, p. 14), namely (following Fétis)
in Monteverdi’s Cruda Amarilli SV 94, mm. 9\sphinxhyphen{}19, 24\sphinxhyphen{}30. Without
diminishing Monteverdi’s influence we can see here that the first
half of the 17 century was indeed a time of change, at least with
respect to the conjunct usage of tones. But note also that the most
prominent composer in that epoch in the dataset is Gesualdo who is
well\sphinxhyphen{}known for his unusual harmonies.

\item {} 
\sphinxAtStartPar
1650\sphinxhyphen{}1700: Confusion

\item {} 
\sphinxAtStartPar
1650\sphinxhyphen{}1800 The separation between flat, diatonic, and sharp tpcs
stabilizes. This is the closest to the overall picture above
(although not as centered). The closest distribution to the overall
distribution (check!) is the one in the late 18th century. It
coincides with the common\sphinxhyphen{}practice period. Since we see that tpc
behavior is different before and after, the CPT should not be taken
as a synonym to tonal music. This affects large portions of empirical
research of tonality presupposing two modes with clear and stable
patterns. Review also Harasim et al. (2019)

\item {} 
\sphinxAtStartPar
1800\sphinxhyphen{}1900: Strong correlation between all accidentals vanishes. The
diagonal line is very clear. In this time, all pitch classes exhibit
the greatest independence historically speaking.

\item {} 
\sphinxAtStartPar
1900\sphinxhyphen{}…: Looks like a mix of CPT and Extended

\end{enumerate}

\sphinxAtStartPar
Dahlhaus situates the origin of harmonic tonality in the early 17th
century (Untersuchungen, p. 14), namely (following Fétis) in
Monteverdi’s *Cruda Amarilli* SV 94, mm. 9\sphinxhyphen{}19, 24\sphinxhyphen{}30. Without
diminishing Monteverdi’s influence we can see here that the first half
of the 17 century was indeed a time of change, at least with respect to
the conjunct usage of tones. But note also that the most prominent
composer in that epoch in the dataset is Gesualdo who is well\sphinxhyphen{}known for
his unusual harmonies.

\sphinxAtStartPar
Turn argument around: Use inter\sphinxhyphen{}pc correlations to show importance of
fifth structure! What about thirds?


\chapter{Diatonicism — Chromaticism — Enharmonicism}
\label{\detokenize{diatonicism_chromaticism:diatonicism-chromaticism-enharmonicism}}\label{\detokenize{diatonicism_chromaticism::doc}}
\sphinxAtStartPar
“When we think about harmony, we automatically think about chords. In
fact, we are so fixated on chords that we sometimes forget they tell
only part of the story” Tymoczko {[}\hyperlink{cite.8_bibliography:id59}{62}{]}, p. 154.

\sphinxAtStartPar
The development of tonality can also be described as a change in two
dimensions: key\sphinxhyphen{}distance and separatedness (tonal closure/unity).
\begin{itemize}
\item {} 
\sphinxAtStartPar
Baroque: Keys are relatively close to each other but changes occur
frequently, tonicizations are commonplace

\item {} 
\sphinxAtStartPar
Classic: Keys are relatively close to each other and key sections are
larger and relatively homogenuous

\item {} 
\sphinxAtStartPar
early Romantic: Keys are further apart and key sections are larger
and relatively homogenuous

\item {} 
\sphinxAtStartPar
late Romantic: Keys are further apart but changes occur frequently

\end{itemize}

\sphinxAtStartPar
Here a tabular overview of this hypothesis:


\begin{savenotes}\sphinxattablestart
\centering
\begin{tabulary}{\linewidth}[t]{|T|T|T|T|}
\hline
&&&\\
\hline&
\sphinxAtStartPar
small
&
\sphinxAtStartPar
large
&\\
\hline&
\sphinxAtStartPar
small
&
\sphinxAtStartPar
Baroque
&
\sphinxAtStartPar
Late Romantic
\\
\hline&
\sphinxAtStartPar
large
&
\sphinxAtStartPar
Classic
&
\sphinxAtStartPar
Early Romantic
\\
\hline
\end{tabulary}
\par
\sphinxattableend\end{savenotes}

\sphinxAtStartPar
Table: Stages of Tonality.


\section{Expansion of tonal material}
\label{\detokenize{diatonicism_chromaticism:expansion-of-tonal-material}}
\sphinxAtStartPar
Based on {[}\hyperlink{cite.8_bibliography:id63}{18}{]}: (see MGG “Diatonik \textendash{}
Chromatik \textendash{} Enharmonik”)
\begin{itemize}
\item {} 
\sphinxAtStartPar
same diatonic region on LoF: relative keys/scales \sphinxhyphen{} although
theoretically, LoF is equivalent to \(\mathbb{Z}\), composers use
only a relatively small subset of it

\item {} 
\sphinxAtStartPar
individual intervals can be associated with a regime on the
fifth\sphinxhyphen{}width space: m2 (5Q) is diatonic, wheras A1 (7Q) is chromatic, and
A7\(\approx\)P8 (12Q).e

\item {} 
\sphinxAtStartPar
compare with “pitch class circulation” {[}\hyperlink{cite.8_bibliography:id59}{62}{]}, p. 158ff.

\item {} 
\sphinxAtStartPar
fifth width measures Diatonicism \sphinxhyphen{}\textgreater{} Chromaticism \sphinxhyphen{}\textgreater{} Enharmonicism, {[}\hyperlink{cite.8_bibliography:id63}{18}{]}, p. 243

\end{itemize}

\sphinxAtStartPar
{[}Insert image from SysMus tutorial{]}
\begin{itemize}
\item {} 
\sphinxAtStartPar
Analyze also the variance of fifth\sphinxhyphen{}widths, not only the means!

\end{itemize}

\sphinxAtStartPar
How can enharmonic exchange (Verwechslung) and enharmonic equivalence
(Umdeutung) distinguished? The former implies a reinterpretation of
tonal pitch classes, i.e. a transition to a different location in tonal
space, whereas the former is only motivated by notational constraints
(parsimony) and tonal/diatonic relations remain constant.

\sphinxAtStartPar
For example, in Debussy’s \sphinxstyleemphasis{Claire de lune}. It is in D\(\flat\)
major with a middle segment in C\(\sharp\) minor which is
enharmonically equivalent to D\(\flat\) minor but only has four
sharps instead of eight flats.

\sphinxAtStartPar
First, enharmonic equivalence should only be invoked to render notation
easier, not more difficult. This means, that the number of accidentals
has to be reduced by the transformation.

\sphinxAtStartPar
Second, the key in question should be in a direct relation with the
preceding and/or consequent key. In the case of the reinterpretation of
the German sixth chord as a dominant chord effects a key shift by a
semiton, which is far away in tonal space (LoF). In the Debussy example,
the keys are only \(R\) related after applying the equivalence.


\section{Expansion of local harmonic content}
\label{\detokenize{diatonicism_chromaticism:expansion-of-local-harmonic-content}}
\sphinxAtStartPar
Fifth width per measure in a piece.

\sphinxAtStartPar
A couple of examples


\subsection{Over time}
\label{\detokenize{diatonicism_chromaticism:over-time}}
\sphinxAtStartPar
The change in fifth widths is differently on a global (piece) and a
local (measure) scale. Globally, pieces cross the boundary to
chromaticism quite early (which can already happen with ficta), and even
to enharmonicism (because modulations to distant keys takes place). At
the same time looking on a local harmonic scale we see that chromatic
(“dissonant”) harmonies are rare on average (mode, mean, median) but are
increasing historically (with an interesting wavelike pattern \sphinxhyphen{} what
does it mean?). Locally, pieces do not cross the enharmonic threshold
(on average)


\section{Enharmonic spectrum}
\label{\detokenize{diatonicism_chromaticism:enharmonic-spectrum}}
\sphinxAtStartPar
In the extreme case, for each note, a random tonal representative of the
neutral pitch class is sampled uniformly. =\textgreater{} unpredictable because
infinite possibilities.

\sphinxAtStartPar
In practice, only a few representatives are likely candidates: not
uniform prior on representatives but concentrated (has \sphinxstyleemphasis{a lot} to do
with surrounding notes\textendash{}context\textendash{}but this is not possible on the
bag\sphinxhyphen{}of\sphinxhyphen{}notes model).

\sphinxAtStartPar
Anyway, \sphinxstyleemphasis{if} absolute enharmonicism would prevail, the prior on the
representatives would be flat (but this does not even happen in 12\sphinxhyphen{}tone
music, show some examples). In “moderate” enharmonicism, some candidates
would be preferred.

\sphinxAtStartPar
I can measure hoe many representatives of a pitch class occur in a
piece.

\sphinxAtStartPar
=\textgreater{} \sphinxstylestrong{enharmonic pitch\sphinxhyphen{}class entropy} is a measure of enharmonicism
(works obviously only with spelled pitch classes)

\sphinxAtStartPar
But: Even in a Bach piece (or older), e.g. F\(\sharp\) and
G\(\flat\) can co\sphinxhyphen{}occur. Because they occur in different contexts
(different keys/tonal centers), they are \sphinxstylestrong{not} enharmonically
equivalent. In the bag\sphinxhyphen{}of\sphinxhyphen{}notes model we need to factor in the fact of
how likely it is to belong to a tonal center:

\sphinxAtStartPar
Which I already can estimate because of the mixture/topic model!

\sphinxAtStartPar
Thus, \sphinxstylestrong{enharmonicism} can be operationalized as the pitch\sphinxhyphen{}class
entropy, weighted by the likelihood to belong to different
tonalities/clusters/keys/tone fields.

\sphinxAtStartPar
=\textgreater{} maybe inverse weight, because: higher weights of F\# and Gb in a Bach
piece should trigger a new mixture component whereas in an extended
tonal piece it might just adjust the parameters of existing components
(variance)

\sphinxAtStartPar
{[}By the way, enharmonic distance is 12n (in fifths){]}

\sphinxAtStartPar
But maybe also: actually inverse because in tonal music, enharmonic
notes are outliers whereas in enharmonic music they get a lot of
probability mass.

\sphinxAtStartPar
But then: How to distinguish enharmonically equivalent tonal regions
from random enharmonicity?

\sphinxAtStartPar
—\textgreater{} entropy might help

\sphinxAtStartPar
If entropy is low, they should be outliers. If it is high, enharmonicism
can be assumed.

\sphinxAtStartPar
Entropy is \sphinxstylestrong{highest} when all representatives are equally likely
(ideally, 12\sphinxhyphen{}tone music).

\sphinxAtStartPar
Thus: the higher the enharmonic pitch\sphinxhyphen{}class entropy the higher is
enharmonicism.

\sphinxAtStartPar
\sphinxstylestrong{Hypothesis:} EPCE increases over time (and is maximal with 12\sphinxhyphen{}tone
compositions)


\bigskip\hrule\bigskip


\sphinxAtStartPar
Segmentation can be either achieved in a fixed manner by bars, groups of
bars, or segmentation sign posts such as key\sphinxhyphen{}signature changes or double
bars.

\sphinxAtStartPar
A data\sphinxhyphen{}driven segmentation could use the notes themselves.

\sphinxAtStartPar
Then, segment length \(l\) would be informative about tonal
stability / rate of tonality changes
\begin{itemize}
\item {} 
\sphinxAtStartPar
given the optimal number of clusters from the mixture model, apply
key\sphinxhyphen{}scape algorithm. It should give rise to a much clearer
segmentation with \(K<<24\) components.

\item {} 
\sphinxAtStartPar
Again, use Information Theory to determine best segmentation. (lowest
entropy, per segment?, KL divergence with component distributions?)

\item {} 
\sphinxAtStartPar
How can these differences be measured? (not always binary)

\end{itemize}

\sphinxAtStartPar
Laws can govern \sphinxstyleemphasis{primary parameters} which allow for syntactic relations
between discrete units (such as pitch, or rhythm), and \sphinxstyleemphasis{secondary
parameters} which discribe continous dimensions such as timbre,
dynamics, etc.

\sphinxAtStartPar
\sphinxstylestrong{Tonal Centers}
\begin{itemize}
\item {} 
\sphinxAtStartPar
Number of Tonal Centers

\item {} 
\sphinxAtStartPar
Distance of Tonal Centers

\item {} 
\sphinxAtStartPar
Divergence on the Tonnetz

\end{itemize}


\chapter{Topic Modeling with Latent Dirichlet Allocation (LDA)}
\label{\detokenize{topic_modeling:topic-modeling-with-latent-dirichlet-allocation-lda}}\label{\detokenize{topic_modeling::doc}}
\begin{sphinxadmonition}{note}{Note:}
\sphinxAtStartPar
Rework this chapter based on the pedagogical introduction
in {[}\hyperlink{cite.8_bibliography:id15}{40}{]}.
\end{sphinxadmonition}

\sphinxAtStartPar
\sphinxstylestrong{Topic Models: What are corpora, documents, topics? “Distributional
hypothesis” (Harris, 1954; Firth, 1957).}

\sphinxAtStartPar
In general, topic models describe the generative process of how
documents (viewed as bags of words) have been created. A document is
defined as a distribution over topics and a topic is defined as a
distribution over words. To generate a new document, one first chooses a
distribution over topics, and for each word in the document choose a
topic from this distribution. The word is then sampled from the
distribution over words of this topic.

\sphinxAtStartPar
A generative model for documents is based on simple probabilistic
sampling rules that describe how words in documents might be
generated on the basis of latent (random) variables. When fitting a
generative model, the goal is to find the best set of latent
variables that can explain the observed data (i.e., observed words
in documents), assuming that the model actually generated the data.
(Steyvers \& Griffiths, 2007)


\section{Background}
\label{\detokenize{topic_modeling:background}}\begin{itemize}
\item {} 
\sphinxAtStartPar
LDA in general (short review of relevant papers), numerous
extensions of the basic model

\item {} 
\sphinxAtStartPar
application to music, review model of
{[}\hyperlink{cite.8_bibliography:id61}{23}, \hyperlink{cite.8_bibliography:id62}{24}{]}

\end{itemize}

\sphinxAtStartPar
A \sphinxstylestrong{corpus} \(\mathcal C\) is a set of \(M\) pieces. For each
piece, the \sphinxstylestrong{distribution of topics} \(\theta\) is drawn from a
Dirichlet distribution with fixed corpus parameter \(\alpha.\)

\sphinxAtStartPar
A collection (multiset) of \sphinxstylestrong{notes}
\(\boldsymbol{u}_n = \{u_{n1},\ldots,u_{nL}\}\) defines a
\sphinxstylestrong{segment}. The number of unique notes in a corpus is the \sphinxstylestrong{vocabulary
size} \(V\). Each segment \({u}_n\) (e.g. beat, slice, bar,
section, …) is assigned a unique \sphinxstylestrong{topic label} \(z\) (key,
tonality, mode, …). A \sphinxstylestrong{piece}
\(\mathcal P = \{\boldsymbol{u}_1, \ldots, u_N\}\) consists of
\(N\) segments with associated topic labels. A piece can have at
most \(N\) topics, if \(N\leq V\), otherwise at most \(V\)
topics. \sphinxstylestrong{Topics} \(\beta\) are defined as distributions over
notes.
Since there are \(K\) topics and \(V\) distinct notes,
\(\beta\) can be represented as a \(V \times K\) matrix where
\(\beta_{ij}\) encodes the probability of note \(i\) in topic
\(j\).


\subsection{Definitions and Assumptions}
\label{\detokenize{topic_modeling:definitions-and-assumptions}}\begin{enumerate}
\sphinxsetlistlabels{\arabic}{enumi}{enumii}{}{.}%
\item {} 
\sphinxAtStartPar
A \sphinxstyleemphasis{note} \(u \in \mathbb Z_{12}\).

\item {} 
\sphinxAtStartPar
A \sphinxstyleemphasis{segment} \(\mathbf{u}_n = \{u_{n1}, \ldots{}, u_{nL}\}\). In a
bag\sphinxhyphen{}of\sphinxhyphen{}notes (BoN) model, a segment can also be represented by a
12\sphinxhyphen{}dimensional count vector \(x_n\), where \(x_n^j\) counts
the number of times note \(j\) occurs.

\item {} 
\sphinxAtStartPar
A \sphinxstyleemphasis{piece} \(s\) is a sequence of \(N\) segments:
\(s=\{\mathbf u_1, ..., \mathbf u_N \}\). Again, in a BoN model a
piece can be represented as a sequence of count vectors
\(X=(x_1, ..., x_N)\).

\item {} 
\sphinxAtStartPar
A \sphinxstyleemphasis{corpus} is a collection of \(M\) pieces,
\(\mathcal S = \{s_1, ..., s_M\}\).

\item {} 
\sphinxAtStartPar
Finally, a \sphinxstyleemphasis{topic} \(z\) is a probability distribution over the
12 pitch classes. In their model, a topic models the concept of key
and each segment is assumed to have precisely one topic/key. Thus,
the sequence of topics in a given piece is modeled as
\(\mathbf z = (z_1, ..., z_N)\).

\item {} 
\sphinxAtStartPar
They fix the number of topics to \(K=24\), based on prior music
theory knowledge.

\end{enumerate}


\section{A generative model for a musical piece}
\label{\detokenize{topic_modeling:a-generative-model-for-a-musical-piece}}
\sphinxAtStartPar
Bag of notes model… multinomial distribution… no order/structure
among the classes (tpcs)
\begin{itemize}
\item {} 
\sphinxAtStartPar
Formalization of LDA as a probabilistic graphical model (PGM)

\item {} 
\sphinxAtStartPar
PGMs are generative models. Toy example to generate pieces.

\end{itemize}
\begin{enumerate}
\sphinxsetlistlabels{\arabic}{enumi}{enumii}{}{.}%
\item {} 
\sphinxAtStartPar
For each piece \(s_m\), \(m=1, ..., M\), draw a
\(K\)\sphinxhyphen{}dimensional topic weight vector \(\theta\) from a
Dirichlet distribution
\(\left(\theta \sim \mathrm{Dir}(\alpha)\right)\) to determine
which keys are likely to occur:
\begin{equation*}
\begin{split}\begin{aligned}
    p(\theta \mid \alpha) = \frac{\Gamma\left(\sum_i \alpha_i\right)}{\prod_i \Gamma \left(\alpha_i\right)}\prod_i \theta^{\alpha_i - 1}.
    \end{aligned}\end{split}
\end{equation*}
\sphinxAtStartPar
The corpus\sphinxhyphen{}level parameter \(\alpha\) determines which topics are
likely to co\sphinxhyphen{}occur in pieces.

\item {} 
\sphinxAtStartPar
For each segment \(\mathbf u_n\), \(n=1, ...N\), in the
piece, choose topic \(z_n \in \{1, ..., K\}\) from the
multinomial distribution \(p(z_n=k \mid \theta) = \theta_k\).

\item {} 
\sphinxAtStartPar
For each note \(u_{nl}\) in \(\mathbb u_n\),
\(l=1, ..., L\), choose a pitch\sphinxhyphen{}class from the multinomial
distribution \(p(u_{nl} = i \mid z_n=k, \beta)=\beta_{ij}\),
where \(\beta\) is a \(V \times K\) matrix encoding each
topic as a distribution over \(V=12\) pitch classes.

\end{enumerate}

\sphinxAtStartPar
This generative process defines a joint probability distribution over
observed and latent random variables for each piece in the corpus:
\begin{equation*}
\begin{split}\begin{aligned}
    p(\theta, \mathbf z, s \mid \alpha, \beta) = p(\theta \mid \alpha)\prod_{n=1}^N p(z_n \mid \theta) \prod_{l=1}^L p(u_{nl} \mid z_n, \beta).\end{aligned}\end{split}
\end{equation*}
\sphinxAtStartPar
In this model, a piece is a bag\sphinxhyphen{}of\sphinxhyphen{}segments, and segments are
bags\sphinxhyphen{}of\sphinxhyphen{}notes.


\subsection{Inference and Learning}
\label{\detokenize{topic_modeling:inference-and-learning}}
\sphinxAtStartPar
The model is fully specified by the corpus\sphinxhyphen{}level Dirichlet parameter
\(\alpha\) and the key\sphinxhyphen{}profile matrix \(\beta\). Under the
assumption that they are known, key\sphinxhyphen{}profiles for segments or pieces can
be inferred by computing the posterior distribution
\begin{equation*}
\begin{split}\begin{aligned}
    p(\theta, \mathbf z \mid \alpha, \beta, s) = \frac{p(\theta, \mathbf z, s \mid \alpha, \beta)}{p(s\mid \alpha, \beta)},\end{aligned}\end{split}
\end{equation*}
\sphinxAtStartPar
according to Bayes’ rule.

\sphinxAtStartPar
The denominator in the last equation is called the \sphinxstyleemphasis{marginal
distribution} or \sphinxstyleemphasis{likelihood} of a piece. The learning problem for the
present setting is to maximize the log\sphinxhyphen{}likelihood of all pieces in the
corpus (“Which combination of \(\alpha\) and \(\beta\) make it
most likely that these pieces were generated?”). Thus, we want to
maximize
\begin{equation*}
\begin{split}\begin{aligned}
\mathcal L(\alpha, \beta) = \int d\theta p(\theta \mid \alpha) \prod_{n=1}^N \sum_{z_n=1}^K p(z_n \mid \theta) \prod_{l=1}^L p(u_{nl}\mid z_k, \beta).\end{aligned}\end{split}
\end{equation*}
\sphinxAtStartPar
The simplest learning algorithm for this task is the expectation
maximization (EM) algorithm. Since this is not tractable, it has to be
approximated. They use variational approximation. I use Gibbs sampling.
Gibbs sampling can be understood as a generalization of the EM
algorithm. Instead of maximizing at each of its two steps (E and M),
Gibbs sampling uses the conditional distributions and samples from them.
\begin{itemize}
\item {} 
\sphinxAtStartPar
Learn model parameters from corpus given a number \(K\) of topics
via Gibbs sampling.

\item {} 
\sphinxAtStartPar
Train\sphinxhyphen{}test split not self\sphinxhyphen{}evident. Possibilities:
\begin{itemize}
\item {} 
\sphinxAtStartPar
Train on whole corpus

\item {} 
\sphinxAtStartPar
Lern topics for different periods separately

\end{itemize}

\end{itemize}


\section{Extension of the LDA music model}
\label{\detokenize{topic_modeling:extension-of-the-lda-music-model}}
\sphinxAtStartPar
Describe potential adaptions of the LDA model. In particular the
difference to the pitch class representation and the interpretation of
topics as tone fields, not keys
\begin{itemize}
\item {} 
\sphinxAtStartPar
Model notes in \(\mathbb Z\) instead of \(\mathbb Z_{12}\)
(line of fifths instead of circle of fifths).

\item {} 
\sphinxAtStartPar
Use different segmentations:
\begin{itemize}
\item {} 
\sphinxAtStartPar
Slices (onsets)

\item {} 
\sphinxAtStartPar
Beats

\item {} 
\sphinxAtStartPar
Bars

\item {} 
\sphinxAtStartPar
Key\sphinxhyphen{}regions (as defined by accidentals)

\item {} 
\sphinxAtStartPar
entire piece

\end{itemize}

\item {} 
\sphinxAtStartPar
Allow for more topics. Hypothesis: chromatic passages, hexatonic,
octatonic, pentatonic, and variants of several keys will show up as
topics.

\item {} 
\sphinxAtStartPar
Take note\sphinxhyphen{}order into account: Griffiths, Steyvers, Blei \& Tenenbaum
(2005), and Andrews \& Vigliocco (2010)

\item {} 
\sphinxAtStartPar
Dynamic Topic Modeling \sphinxhyphen{} Changes of topics over time: Blei \& Lafferty
(2006)

\end{itemize}

\sphinxAtStartPar
Other Features / Random Variables
\begin{itemize}
\item {} 
\sphinxAtStartPar
Length in notes

\item {} 
\sphinxAtStartPar
Pitch\sphinxhyphen{}class distribution in \(\mathbb Z\) and in
\(\mathbb Z_{12}\)

\item {} 
\sphinxAtStartPar
Number of key changes

\item {} 
\sphinxAtStartPar
Chromaticism

\item {} 
\sphinxAtStartPar
Meter / Meter change

\end{itemize}

\sphinxAtStartPar
General Notes:
\begin{itemize}
\item {} 
\sphinxAtStartPar
Interpretation: Because the musical vocabulary is quite small when
notes are the equivalent of words, it is not sufficient to just look
at the most frequent notes in a topic in order to interpret it but
rather to inspect the whole distribution over notes.

\item {} 
\sphinxAtStartPar
Similarity between documents (pieces) and subcorpora \textendash{}\textgreater{} Clustering

\end{itemize}


\section{Topics inferred from the corpus}
\label{\detokenize{topic_modeling:topics-inferred-from-the-corpus}}
\sphinxAtStartPar
Figure {[}fig:topic\_dists{]} shows the note distributions for the
\(K=7\) inferred topics.

\sphinxAtStartPar
The relative weights of the topics in the overall corpus can be seen in
Figure {[}fig:topics\_weights{]}. The most common topic is topic 0
(“(transposed) diatonic”) and the least common is topic 4 (“far\sphinxhyphen{}flats”)

\sphinxAtStartPar
Besides calculating the overall importance of topics in the corpus, one
can also look at the relative topic weights within individual pieces.
Figure {[}fig:raw\_topics{]} shows this distribution. Analogous to the
pitch\sphinxhyphen{}class evolution from Figure {[}{]}, pieces that are assigned to the
same year are accumulated (\sphinxstylestrong{Describe procedure!}).

\sphinxAtStartPar
It is obvious how the lack of data in earlier periods affects the
pattern we see. Nonetheless, it can be seen, that earlier pieces rarely
contain the certain topics which only occur later.


\section{Number of topics reveals hierarchy of tonality}
\label{\detokenize{topic_modeling:number-of-topics-reveals-hierarchy-of-tonality}}
\sphinxAtStartPar
“Vertical”: Different values of \(K\in [2,12]\) indicate
hierarchical nature of tonality.
\begin{enumerate}
\sphinxsetlistlabels{\arabic}{enumi}{enumii}{}{.}%
\item {} 
\sphinxAtStartPar
Compare topic distributions for different values of \(K\)

\item {} 
\sphinxAtStartPar
Relate topics from different \(K\)\sphinxhyphen{}stages with each other: coarse
to fine, correlations between some topics should increase

\end{enumerate}

\sphinxAtStartPar
All matrices where based on documents. In the classical LDA setting, a
corpus is a bag of documents. We are in particular interested in
historical developments, so the chronological order is important.
Moreover, we do not a piece for each year and for some years we have
many pieces. The first step is to re\sphinxhyphen{}assign each piece its “display
year” (composition, publication, of composer half\sphinxhyphen{}life). Then we average
all pieces in the same year. We now have at most one topic distribution
per year in the corpus.

\sphinxAtStartPar
…

\sphinxAtStartPar
But there are still years for which we do not have data, in particular
in the earlier periods. Pragmatically, if we do not have a topic
distribution for a given year, we take the one from the previous year.
To that end, we create a time index ranging from the earliest to the
latest date in the corpus.

\sphinxAtStartPar
We then iterate over all years and use the inferred topic distributions
if there is one for that particular year. If not, we use the same as in
the year before.


\section{Historical inferences for each value of \protect\(K\protect\)}
\label{\detokenize{topic_modeling:historical-inferences-for-each-value-of-k}}

\subsection{Sliding windows reveal trends}
\label{\detokenize{topic_modeling:sliding-windows-reveal-trends}}
\sphinxAtStartPar
Figure {[}fig:raw\_topics{]} takes a very fine\sphinxhyphen{}grained view on the evolution
of the topic distribution because each year is a single data point. In
order to see larger trends, we can zoom out and look at smoothed
versions of the same data. We inspect rolling averages with a window
size of 30, 50, and a hundred years to see generational, epochal and
secular trends.

\sphinxAtStartPar
In these figures we can observe a) the relative topic importances
(weights) over time, and b) identify breaking points and local extrema.

\sphinxAtStartPar
Moreover, the entropy of these distributions is informative!

\sphinxAtStartPar
Todo: PCA shows relation between topics and documents in (reduced)
note\sphinxhyphen{}space


\section{Topic coevolution}
\label{\detokenize{topic_modeling:topic-coevolution}}
\sphinxAtStartPar
topic correlations motivate hierarchical clustering


\section{Pitch class \textendash{} topic coevolution}
\label{\detokenize{topic_modeling:pitch-class-topic-coevolution}}
\sphinxAtStartPar
Overall Figure {[}fig:tpc\_topic\_coevolution{]} almost looks like a block
matrix.

\sphinxAtStartPar
Locally, the fifths order is preserved, especially the diatonic, as seen
by the row clusters! Regarding the topics, we see two major clusters.
The left most one is “chromatic notes” and the right one is “diatonic
plus”. The diatonic cluster contains the diatonic notes without F, which
get clustered with F but without B (note also that when B is included,
Bb is most negatively correlated, and, when F is included, F\# is most
negatively correlated). The tritone is the condition to separate these
as well as the chromatic semitone. Then it gets extended by sharps up to
G\# which makes sense because of the dominant of a minor. The last topic
to join this cluster extends it into the flat direction. We have already
noted that tonal music is generally sharpwards oriented so it makes
sense that the evolution of flat notes is weaker correlated with
diatonic notes than sharp ones.

\sphinxAtStartPar
The second cluster…


\subsection{Beyond the bag\sphinxhyphen{}of\sphinxhyphen{}notes model: the Hidden Markov Topic Model (HMTM)}
\label{\detokenize{topic_modeling:beyond-the-bag-of-notes-model-the-hidden-markov-topic-model-hmtm}}
\sphinxAtStartPar
Improving the bag\sphinxhyphen{}of\sphinxhyphen{}notes model with a Hidden Markov Topic Model


\subsection{Discussion}
\label{\detokenize{topic_modeling:discussion}}\begin{itemize}
\item {} 
\sphinxAtStartPar
Result 1: Historically, ever larger portions of pitch space are explored

\item {} 
\sphinxAtStartPar
There is a trend from diatonic \textgreater{} chromatic \textgreater{} enharmonic pieces, but it
is not monotonic. In the 19th century, there are diatonic Lieder
(composer) and Alkan, who has the greatest tonality range
(diatonic\sphinxhyphen{}enharmonic).

\end{itemize}


\chapter{Segmentation}
\label{\detokenize{5_segmentation:segmentation}}\label{\detokenize{5_segmentation::doc}}

\section{Free segmentation}
\label{\detokenize{5_segmentation:free-segmentation}}
\sphinxAtStartPar
(pc set analysis)
\begin{itemize}
\item {} 
\sphinxAtStartPar
Straus {[}\hyperlink{cite.8_bibliography:id3}{57}{]}

\item {} 
\sphinxAtStartPar
Hanninen {[}\hyperlink{cite.8_bibliography:id52}{19}{]}, Hanninen {[}\hyperlink{cite.8_bibliography:id53}{20}{]}

\end{itemize}

\sphinxAtStartPar
relating segments creates specific graph structure


\section{Contiguous segmentation}
\label{\detokenize{5_segmentation:contiguous-segmentation}}
\sphinxAtStartPar
chord labels, Roman numerals, etc.

\sphinxAtStartPar
graph: chain


\section{Hierarchical segmentation}
\label{\detokenize{5_segmentation:hierarchical-segmentation}}
\sphinxAtStartPar
CFGs, Schenker

\sphinxAtStartPar
graph: tree


\section{Exhaustive segmentation}
\label{\detokenize{5_segmentation:exhaustive-segmentation}}
\sphinxAtStartPar
Keyscapes Sapp {[}\hyperlink{cite.8_bibliography:id37}{54}{]}, Sapp {[}\hyperlink{cite.8_bibliography:id38}{55}{]},
Pitchscapes Lieck and Rohrmeier {[}\hyperlink{cite.8_bibliography:id32}{27}{]},
Wavescapes Viaccoz \sphinxstyleemphasis{et al.} {[}\hyperlink{cite.8_bibliography:id51}{64}{]}

\sphinxAtStartPar
See also Müller (form)


\chapter{Advanced topics}
\label{\detokenize{6_advanced:advanced-topics}}\label{\detokenize{6_advanced::doc}}

\section{Pitch Spelling}
\label{\detokenize{6_advanced:pitch-spelling}}\label{\detokenize{6_advanced:id1}}
\sphinxAtStartPar
Meredith {[}\hyperlink{cite.8_bibliography:id27}{35}{]}, Meredith and Wiggins {[}\hyperlink{cite.8_bibliography:id26}{36}{]}
Cambouropoulos {[}\hyperlink{cite.8_bibliography:id28}{6}{]}, Chew and Chen {[}\hyperlink{cite.8_bibliography:id29}{9}{]}
Stoddard \sphinxstyleemphasis{et al.} {[}\hyperlink{cite.8_bibliography:id30}{56}{]}
Temperley {[}\hyperlink{cite.8_bibliography:id31}{59}{]}


\section{Style (classification)}
\label{\detokenize{6_advanced:style-classification}}

\subsection{Feature clustering}
\label{\detokenize{6_advanced:feature-clustering}}
\sphinxAtStartPar
(k\sphinxhyphen{}means, PCA, …)


\subsection{Hierarchical clustering}
\label{\detokenize{6_advanced:hierarchical-clustering}}

\section{History}
\label{\detokenize{6_advanced:history}}
\sphinxAtStartPar
(regression, GPs)
\begin{itemize}
\item {} 
\sphinxAtStartPar
trends (maybe with a non note\sphinxhyphen{}based dataset e.g. metadata)

\end{itemize}


\section{Performance}
\label{\detokenize{6_advanced:performance}}\begin{itemize}
\item {} 
\sphinxAtStartPar
Spotify API to compare different recordings

\end{itemize}


\section{Modeling musical sequences}
\label{\detokenize{6_advanced:modeling-musical-sequences}}

\subsection{Hierarchical theories}
\label{\detokenize{6_advanced:hierarchical-theories}}
\sphinxAtStartPar
describe central Schenkerian concepts in terms
of tones, intervals, and underlying tonal spaces.
E.g., a neighbor note is the next note (upper or lower)
in a tonal space that has a notion of neighborhood, e.g.
the diatonic or chromatic spaces. But in this generalized sense,
a neighbor can be a semitone, a whole tone, a third, or a fifth
apart. What the neighbor actually is, depends on the underlying
assumed tonal space. Accordingly, the \sphinxstyleemphasis{Bassbrechung} is an upper
neighbor on the circle or line of fifths, while the common neighbor note
only exists in diatonic spaces.


\subsection{Formal models for music sequences}
\label{\detokenize{6_advanced:formal-models-for-music-sequences}}
\sphinxAtStartPar
See overview in {[}\hyperlink{cite.8_bibliography:id17}{51}{]}


\subsubsection{Regular Expressions}
\label{\detokenize{6_advanced:regular-expressions}}
\sphinxAtStartPar
(chord symbols, rhythms)


\subsubsection{n\sphinxhyphen{}gram models}
\label{\detokenize{6_advanced:n-gram-models}}
\sphinxAtStartPar
(melody, rhythms)


\subsubsection{Hidden Markov models}
\label{\detokenize{6_advanced:hidden-markov-models}}
\sphinxAtStartPar
(harmony)


\subsubsection{Probabilistic Context\sphinxhyphen{}Free Grammars}
\label{\detokenize{6_advanced:probabilistic-context-free-grammars}}
\sphinxAtStartPar
(form; choro) {[}\hyperlink{cite.8_bibliography:id14}{38}, \hyperlink{cite.8_bibliography:id16}{41}{]}


\subsubsection{More advanced models}
\label{\detokenize{6_advanced:more-advanced-models}}
\sphinxAtStartPar
Neural nets


\section{Music analysis with the discrete Fourier transform}
\label{\detokenize{6_advanced:music-analysis-with-the-discrete-fourier-transform}}

\section{Topic modeling}
\label{\detokenize{6_advanced:topic-modeling}}
\cleardoublepage
\begingroup
\renewcommand\chapter[1]{\endgroup}
\phantomsection


\chapter{Bibliography}
\label{\detokenize{8_bibliography:bibliography}}\label{\detokenize{8_bibliography::doc}}
\sphinxAtStartPar



\chapter{Developers}
\label{\detokenize{index:developers}}

\section{API \sphinxhyphen{} gamuth}
\label{\detokenize{api:module-gamuth}}\label{\detokenize{api:api-gamuth}}\label{\detokenize{api:api}}\label{\detokenize{api::doc}}\index{module@\spxentry{module}!gamuth@\spxentry{gamuth}}\index{gamuth@\spxentry{gamuth}!module@\spxentry{module}}

\begin{fulllineitems}
\pysiglinewithargsret{\sphinxbfcode{\sphinxupquote{class\DUrole{w}{  }}}\sphinxcode{\sphinxupquote{gamuth.}}\sphinxbfcode{\sphinxupquote{Interval}}}{\emph{\DUrole{n}{source}}, \emph{\DUrole{n}{target}}}{}
\sphinxAtStartPar
Class for an interval between two tones \sphinxtitleref{s} (source) and \sphinxtitleref{t} (target).


\begin{fulllineitems}
\pysiglinewithargsret{\sphinxbfcode{\sphinxupquote{get\_euclidean\_distance}}}{\emph{\DUrole{n}{precision}\DUrole{o}{=}\DUrole{default_value}{2}}}{}
\sphinxAtStartPar
Calculates the Euclidean distance between two tones
with coordinates in Euler space.
\begin{quote}\begin{description}
\item[{Parameters}] \leavevmode
\sphinxAtStartPar
\sphinxstyleliteralstrong{\sphinxupquote{precision}} (\sphinxstyleliteralemphasis{\sphinxupquote{int}}) \textendash{} Rounding precision.

\item[{Returns}] \leavevmode
\sphinxAtStartPar
The Euclidean distance between two tones \sphinxtitleref{s} (source) and \sphinxtitleref{t} (target).

\item[{Return type}] \leavevmode
\sphinxAtStartPar
float

\end{description}\end{quote}
\subsubsection*{Example}

\begin{sphinxVerbatim}[commandchars=\\\{\}]
\PYG{g+gp}{\PYGZgt{}\PYGZgt{}\PYGZgt{} }\PYG{n}{s} \PYG{o}{=} \PYG{n}{Tone}\PYG{p}{(}\PYG{l+m+mi}{0}\PYG{p}{,}\PYG{l+m+mi}{0}\PYG{p}{,}\PYG{l+m+mi}{0}\PYG{p}{)} \PYG{c+c1}{\PYGZsh{} C\PYGZus{}0}
\PYG{g+gp}{\PYGZgt{}\PYGZgt{}\PYGZgt{} }\PYG{n}{t} \PYG{o}{=} \PYG{n}{Tone}\PYG{p}{(}\PYG{l+m+mi}{1}\PYG{p}{,}\PYG{l+m+mi}{2}\PYG{p}{,}\PYG{l+m+mi}{1}\PYG{p}{)} \PYG{c+c1}{\PYGZsh{} D\PYGZsq{}1}
\PYG{g+gp}{\PYGZgt{}\PYGZgt{}\PYGZgt{} }\PYG{n}{i} \PYG{o}{=} \PYG{n}{Interval}\PYG{p}{(}\PYG{n}{s}\PYG{p}{,}\PYG{n}{t}\PYG{p}{)}
\PYG{g+gp}{\PYGZgt{}\PYGZgt{}\PYGZgt{} }\PYG{n}{i}\PYG{o}{.}\PYG{n}{get\PYGZus{}euclidean\PYGZus{}distance}\PYG{p}{(}\PYG{p}{)}
\PYG{g+go}{2.45}
\end{sphinxVerbatim}

\end{fulllineitems}



\begin{fulllineitems}
\pysiglinewithargsret{\sphinxbfcode{\sphinxupquote{get\_generic\_interval}}}{\emph{\DUrole{n}{directed}\DUrole{o}{=}\DUrole{default_value}{True}}, \emph{\DUrole{n}{octaves}\DUrole{o}{=}\DUrole{default_value}{True}}}{}
\sphinxAtStartPar
Generic interval (directed) between two tones.
\begin{quote}\begin{description}
\item[{Parameters}] \leavevmode\begin{itemize}
\item {} 
\sphinxAtStartPar
\sphinxstyleliteralstrong{\sphinxupquote{directed}} (\sphinxstyleliteralemphasis{\sphinxupquote{bool}}) \textendash{} Affects whether the returned interval is directed or not.

\item {} 
\sphinxAtStartPar
\sphinxstyleliteralstrong{\sphinxupquote{octaves}} (\sphinxstyleliteralemphasis{\sphinxupquote{bool}}) \textendash{} returns generic interval class if \sphinxtitleref{False}.

\end{itemize}

\item[{Returns}] \leavevmode
\sphinxAtStartPar
(Directed) generic interval from \sphinxtitleref{s} to \sphinxtitleref{t}.

\item[{Return type}] \leavevmode
\sphinxAtStartPar
int

\end{description}\end{quote}
\subsubsection*{Example}

\begin{sphinxVerbatim}[commandchars=\\\{\}]
\PYG{g+gp}{\PYGZgt{}\PYGZgt{}\PYGZgt{} }\PYG{n}{db} \PYG{o}{=} \PYG{n}{Tone}\PYG{p}{(}\PYG{l+m+mi}{0}\PYG{p}{,}\PYG{o}{\PYGZhy{}}\PYG{l+m+mi}{1}\PYG{p}{,}\PYG{o}{\PYGZhy{}}\PYG{l+m+mi}{1}\PYG{p}{)} \PYG{c+c1}{\PYGZsh{} Db,0}
\PYG{g+gp}{\PYGZgt{}\PYGZgt{}\PYGZgt{} }\PYG{n}{b} \PYG{o}{=} \PYG{n}{Tone}\PYG{p}{(}\PYG{l+m+mi}{0}\PYG{p}{,}\PYG{l+m+mi}{1}\PYG{p}{,}\PYG{l+m+mi}{1}\PYG{p}{)} \PYG{c+c1}{\PYGZsh{} B\PYGZsq{}0}
\PYG{g+gp}{\PYGZgt{}\PYGZgt{}\PYGZgt{} }\PYG{n}{i1} \PYG{o}{=} \PYG{n}{Interval}\PYG{p}{(}\PYG{n}{db}\PYG{p}{,} \PYG{n}{b}\PYG{p}{)} \PYG{c+c1}{\PYGZsh{} the interval between Db0 and B1 is an ascending thirteenth}
\PYG{g+gp}{\PYGZgt{}\PYGZgt{}\PYGZgt{} }\PYG{n}{i1}\PYG{o}{.}\PYG{n}{generic\PYGZus{}interval}\PYG{p}{(}\PYG{p}{)}
\PYG{g+go}{13}
\end{sphinxVerbatim}

\begin{sphinxVerbatim}[commandchars=\\\{\}]
\PYG{g+gp}{\PYGZgt{}\PYGZgt{}\PYGZgt{} }\PYG{n}{i2} \PYG{o}{=} \PYG{n}{Interval}\PYG{p}{(}\PYG{n}{b}\PYG{p}{,} \PYG{n}{db}\PYG{p}{)} \PYG{c+c1}{\PYGZsh{} the interval between B1 and Db0 is a descending thirteenth}
\PYG{g+gp}{\PYGZgt{}\PYGZgt{}\PYGZgt{} }\PYG{n}{i2}\PYG{o}{.}\PYG{n}{generic\PYGZus{}interval}\PYG{p}{(}\PYG{p}{)}
\PYG{g+go}{\PYGZhy{}13}
\end{sphinxVerbatim}

\begin{sphinxVerbatim}[commandchars=\\\{\}]
\PYG{g+gp}{\PYGZgt{}\PYGZgt{}\PYGZgt{} }\PYG{n}{i3} \PYG{o}{=} \PYG{n}{Interval}\PYG{p}{(}\PYG{n}{b}\PYG{p}{,} \PYG{n}{db}\PYG{p}{)} \PYG{c+c1}{\PYGZsh{} the interval between B1 and Db0 is a descending thirteenth}
\PYG{g+gp}{\PYGZgt{}\PYGZgt{}\PYGZgt{} }\PYG{n}{i3}\PYG{o}{.}\PYG{n}{generic\PYGZus{}interval}\PYG{p}{(}\PYG{n}{directed}\PYG{o}{=}\PYG{k+kc}{False}\PYG{p}{)}
\PYG{g+go}{13}
\end{sphinxVerbatim}

\end{fulllineitems}



\begin{fulllineitems}
\pysiglinewithargsret{\sphinxbfcode{\sphinxupquote{get\_specific\_interval}}}{\emph{\DUrole{n}{directed}\DUrole{o}{=}\DUrole{default_value}{True}}, \emph{\DUrole{n}{octaves}\DUrole{o}{=}\DUrole{default_value}{True}}}{}
\sphinxAtStartPar
Specific interval (directed) between two tones.
\begin{quote}\begin{description}
\item[{Parameters}] \leavevmode\begin{itemize}
\item {} 
\sphinxAtStartPar
\sphinxstyleliteralstrong{\sphinxupquote{directed}} (\sphinxstyleliteralemphasis{\sphinxupquote{bool}}) \textendash{} Affects whether the returned interval is directed or not.

\item {} 
\sphinxAtStartPar
\sphinxstyleliteralstrong{\sphinxupquote{octaves}} (\sphinxstyleliteralemphasis{\sphinxupquote{bool}}) \textendash{} returns specific interval class if \sphinxtitleref{False}.

\end{itemize}

\item[{Returns}] \leavevmode
\sphinxAtStartPar
(Directed) specific interval from \sphinxtitleref{s} to \sphinxtitleref{t}.

\item[{Return type}] \leavevmode
\sphinxAtStartPar
int

\end{description}\end{quote}
\subsubsection*{Example}

\begin{sphinxVerbatim}[commandchars=\\\{\}]
\PYG{g+gp}{\PYGZgt{}\PYGZgt{}\PYGZgt{} }\PYG{n}{fs} \PYG{o}{=} \PYG{n}{Tone}\PYG{p}{(}\PYG{l+m+mi}{0}\PYG{p}{,}\PYG{l+m+mi}{2}\PYG{p}{,}\PYG{l+m+mi}{1}\PYG{p}{)} \PYG{c+c1}{\PYGZsh{} F\PYGZsh{}\PYGZsq{}0}
\PYG{g+gp}{\PYGZgt{}\PYGZgt{}\PYGZgt{} }\PYG{n}{db} \PYG{o}{=} \PYG{n}{Tone}\PYG{p}{(}\PYG{l+m+mi}{0}\PYG{p}{,}\PYG{o}{\PYGZhy{}}\PYG{l+m+mi}{1}\PYG{p}{,}\PYG{o}{\PYGZhy{}}\PYG{l+m+mi}{1}\PYG{p}{)} \PYG{c+c1}{\PYGZsh{} Db,0}
\PYG{g+gp}{\PYGZgt{}\PYGZgt{}\PYGZgt{} }\PYG{n}{i1} \PYG{o}{=} \PYG{n}{Interval}\PYG{p}{(}\PYG{n}{fs}\PYG{p}{,} \PYG{n}{db}\PYG{p}{)}
\PYG{g+gp}{\PYGZgt{}\PYGZgt{}\PYGZgt{} }\PYG{n}{i1}\PYG{o}{.}\PYG{n}{specific\PYGZus{}interval}\PYG{p}{(}\PYG{p}{)}
\PYG{g+go}{17}
\end{sphinxVerbatim}

\begin{sphinxVerbatim}[commandchars=\\\{\}]
\PYG{g+gp}{\PYGZgt{}\PYGZgt{}\PYGZgt{} }\PYG{n}{i1}\PYG{o}{.}\PYG{n}{specific\PYGZus{}interval}\PYG{p}{(}\PYG{n}{octaves}\PYG{o}{=}\PYG{k+kc}{False}\PYG{p}{)}
\PYG{g+go}{5}
\end{sphinxVerbatim}

\end{fulllineitems}


\end{fulllineitems}



\begin{fulllineitems}
\pysigline{\sphinxbfcode{\sphinxupquote{class\DUrole{w}{  }}}\sphinxcode{\sphinxupquote{gamuth.}}\sphinxbfcode{\sphinxupquote{PitchClass}}}
\sphinxAtStartPar
Pitch class instance in \(\mathbb{Z}_{12}\).

\end{fulllineitems}



\begin{fulllineitems}
\pysiglinewithargsret{\sphinxbfcode{\sphinxupquote{class\DUrole{w}{  }}}\sphinxcode{\sphinxupquote{gamuth.}}\sphinxbfcode{\sphinxupquote{PitchClassSet}}}{\emph{\DUrole{n}{st}}}{}
\sphinxAtStartPar
Pitch class sets
\# For multiple constructors see: \sphinxurl{https://pythonconquerstheuniverse.wordpress.com/2010/03/17/multiple-constructors-in-a-python-class/}


\begin{fulllineitems}
\pysiglinewithargsret{\sphinxbfcode{\sphinxupquote{interval\_class\_vector}}}{}{}
\sphinxAtStartPar
Interval\sphinxhyphen{}class vector for given pitch\sphinxhyphen{}class set
\begin{quote}\begin{description}
\item[{Returns}] \leavevmode
\sphinxAtStartPar
interval\sphinxhyphen{}class vector

\item[{Return type}] \leavevmode
\sphinxAtStartPar
list

\end{description}\end{quote}

\end{fulllineitems}



\begin{fulllineitems}
\pysiglinewithargsret{\sphinxbfcode{\sphinxupquote{invert}}}{\emph{\DUrole{n}{t}\DUrole{o}{=}\DUrole{default_value}{0}}}{}
\sphinxAtStartPar
Invert pitch\sphinxhyphen{}class set. If the inversion pc is not specified, it is set to 0 by default.
\begin{quote}\begin{description}
\item[{Parameters}] \leavevmode
\sphinxAtStartPar
\sphinxstyleliteralstrong{\sphinxupquote{t}} (\sphinxstyleliteralemphasis{\sphinxupquote{int}}) \textendash{} inversion pitch class (default: 0)

\item[{Returns}] \leavevmode
\sphinxAtStartPar
inverted pitch\sphinxhyphen{}class set

\item[{Return type}] \leavevmode
\sphinxAtStartPar
set

\end{description}\end{quote}

\end{fulllineitems}



\begin{fulllineitems}
\pysiglinewithargsret{\sphinxbfcode{\sphinxupquote{normal\_form}}}{}{}
\sphinxAtStartPar
Normal form of pitch\sphinxhyphen{}class set
\begin{quote}\begin{description}
\item[{Returns}] \leavevmode
\sphinxAtStartPar
Normal form

\item[{Return type}] \leavevmode
\sphinxAtStartPar
set

\end{description}\end{quote}

\end{fulllineitems}



\begin{fulllineitems}
\pysiglinewithargsret{\sphinxbfcode{\sphinxupquote{transpose}}}{\emph{\DUrole{n}{t}}}{}
\sphinxAtStartPar
Transposition by \sphinxtitleref{t} semitones.
\begin{quote}\begin{description}
\item[{Parameters}] \leavevmode
\sphinxAtStartPar
\sphinxstyleliteralstrong{\sphinxupquote{t}} (\sphinxstyleliteralemphasis{\sphinxupquote{int}}) \textendash{} number of semitones to transpose up

\item[{Returns}] \leavevmode
\sphinxAtStartPar
transposed pitch\sphinxhyphen{}class set

\item[{Return type}] \leavevmode
\sphinxAtStartPar
set

\end{description}\end{quote}

\end{fulllineitems}


\end{fulllineitems}



\begin{fulllineitems}
\pysiglinewithargsret{\sphinxbfcode{\sphinxupquote{class\DUrole{w}{  }}}\sphinxcode{\sphinxupquote{gamuth.}}\sphinxbfcode{\sphinxupquote{Tone}}}{\emph{\DUrole{n}{octave}\DUrole{o}{=}\DUrole{default_value}{None}}, \emph{\DUrole{n}{fifth}\DUrole{o}{=}\DUrole{default_value}{None}}, \emph{\DUrole{n}{third}\DUrole{o}{=}\DUrole{default_value}{None}}, \emph{\DUrole{n}{name}\DUrole{o}{=}\DUrole{default_value}{None}}}{}
\sphinxAtStartPar
Class for tones.


\begin{fulllineitems}
\pysiglinewithargsret{\sphinxbfcode{\sphinxupquote{get\_accidentals}}}{}{}
\sphinxAtStartPar
Gets the accidentals of the tone (flats (\(\flat\)) or sharps (\(\sharp\)))..
\begin{quote}\begin{description}
\item[{Parameters}] \leavevmode
\sphinxAtStartPar
\sphinxstyleliteralstrong{\sphinxupquote{None}} \textendash{} 

\item[{Returns}] \leavevmode
\sphinxAtStartPar
The accidentals of the tone.

\item[{Return type}] \leavevmode
\sphinxAtStartPar
str

\end{description}\end{quote}
\subsubsection*{Example}

\begin{sphinxVerbatim}[commandchars=\\\{\}]
\PYG{g+gp}{\PYGZgt{}\PYGZgt{}\PYGZgt{} }\PYG{n}{t} \PYG{o}{=} \PYG{n}{Tone}\PYG{p}{(}\PYG{l+m+mi}{0}\PYG{p}{,}\PYG{l+m+mi}{7}\PYG{p}{,}\PYG{l+m+mi}{0}\PYG{p}{)} \PYG{c+c1}{\PYGZsh{} C sharp}
\PYG{g+gp}{\PYGZgt{}\PYGZgt{}\PYGZgt{} }\PYG{n}{t}\PYG{o}{.}\PYG{n}{get\PYGZus{}accidentals}\PYG{p}{(}\PYG{p}{)}
\PYG{g+go}{`\PYGZsh{}`}
\end{sphinxVerbatim}

\end{fulllineitems}



\begin{fulllineitems}
\pysiglinewithargsret{\sphinxbfcode{\sphinxupquote{get\_frequency}}}{\emph{\DUrole{n}{chamber\_tone}\DUrole{o}{=}\DUrole{default_value}{440.0}}, \emph{\DUrole{n}{precision}\DUrole{o}{=}\DUrole{default_value}{2}}}{}
\sphinxAtStartPar
Get the frequency of the tone.
\begin{quote}\begin{description}
\item[{Parameters}] \leavevmode\begin{itemize}
\item {} 
\sphinxAtStartPar
\sphinxstyleliteralstrong{\sphinxupquote{chamber\_tone}} (\sphinxstyleliteralemphasis{\sphinxupquote{float}}) \textendash{} The frequency in Hz of the chamber tone. Default: 440.0 (A)

\item {} 
\sphinxAtStartPar
\sphinxstyleliteralstrong{\sphinxupquote{precision}} (\sphinxstyleliteralemphasis{\sphinxupquote{int}}) \textendash{} Rounding precision.

\end{itemize}

\item[{Returns}] \leavevmode
\sphinxAtStartPar
The frequency of the tone in Hertz (Hz).

\item[{Return type}] \leavevmode
\sphinxAtStartPar
float

\end{description}\end{quote}
\subsubsection*{Example}

\begin{sphinxVerbatim}[commandchars=\\\{\}]
\PYG{g+gp}{\PYGZgt{}\PYGZgt{}\PYGZgt{} }\PYG{n}{t} \PYG{o}{=} \PYG{n}{Tone}\PYG{p}{(}\PYG{l+m+mi}{0}\PYG{p}{,}\PYG{l+m+mi}{0}\PYG{p}{,}\PYG{l+m+mi}{0}\PYG{p}{)}
\PYG{g+gp}{\PYGZgt{}\PYGZgt{}\PYGZgt{} }\PYG{n}{t}\PYG{o}{.}\PYG{n}{get\PYGZus{}frequency}\PYG{p}{(}\PYG{n}{precision}\PYG{o}{=}\PYG{l+m+mi}{3}\PYG{p}{)}
\PYG{g+go}{261.626}
\end{sphinxVerbatim}

\end{fulllineitems}



\begin{fulllineitems}
\pysiglinewithargsret{\sphinxbfcode{\sphinxupquote{get\_label}}}{}{}
\sphinxAtStartPar
Gets the complete label of the tone, consisting of its note name, syntonic position, and octave.
\begin{quote}\begin{description}
\item[{Parameters}] \leavevmode
\sphinxAtStartPar
\sphinxstyleliteralstrong{\sphinxupquote{None}} \textendash{} 

\item[{Returns}] \leavevmode
\sphinxAtStartPar
The accidentals of the tone.

\item[{Return type}] \leavevmode
\sphinxAtStartPar
str

\end{description}\end{quote}
\subsubsection*{Example}

\begin{sphinxVerbatim}[commandchars=\\\{\}]
\PYG{g+gp}{\PYGZgt{}\PYGZgt{}\PYGZgt{} }\PYG{n}{c} \PYG{o}{=} \PYG{n}{Tone}\PYG{p}{(}\PYG{l+m+mi}{0}\PYG{p}{,}\PYG{l+m+mi}{0}\PYG{p}{,}\PYG{l+m+mi}{0}\PYG{p}{)}
\PYG{g+gp}{\PYGZgt{}\PYGZgt{}\PYGZgt{} }\PYG{n}{ab} \PYG{o}{=} \PYG{n}{Tone}\PYG{p}{(}\PYG{l+m+mi}{0}\PYG{p}{,}\PYG{l+m+mi}{1}\PYG{p}{,}\PYG{o}{\PYGZhy{}}\PYG{l+m+mi}{1}\PYG{p}{)}
\PYG{g+gp}{\PYGZgt{}\PYGZgt{}\PYGZgt{} }\PYG{n}{c}\PYG{o}{.}\PYG{n}{get\PYGZus{}label}\PYG{p}{(}\PYG{p}{)}\PYG{p}{,} \PYG{n}{ab}\PYG{o}{.}\PYG{n}{get\PYGZus{}label}\PYG{p}{(}\PYG{p}{)}
\PYG{g+go}{`C\PYGZus{}0` `Ab,1`}
\end{sphinxVerbatim}

\end{fulllineitems}



\begin{fulllineitems}
\pysiglinewithargsret{\sphinxbfcode{\sphinxupquote{get\_midi\_pitch}}}{}{}
\sphinxAtStartPar
Get the MIDI pitch of the tone.
\begin{quote}\begin{description}
\item[{Parameters}] \leavevmode
\sphinxAtStartPar
\sphinxstyleliteralstrong{\sphinxupquote{None}} \textendash{} 

\item[{Returns}] \leavevmode
\sphinxAtStartPar
The MIDI pitch of the tone if it is in MIDI pitch range (0\textendash{}128)

\item[{Return type}] \leavevmode
\sphinxAtStartPar
int

\end{description}\end{quote}
\subsubsection*{Example}

\begin{sphinxVerbatim}[commandchars=\\\{\}]
\PYG{g+gp}{\PYGZgt{}\PYGZgt{}\PYGZgt{} }\PYG{n}{t} \PYG{o}{=} \PYG{n}{Tone}\PYG{p}{(}\PYG{l+m+mi}{0}\PYG{p}{,}\PYG{l+m+mi}{0}\PYG{p}{,}\PYG{l+m+mi}{0}\PYG{p}{)}
\PYG{g+gp}{\PYGZgt{}\PYGZgt{}\PYGZgt{} }\PYG{n}{t}\PYG{o}{.}\PYG{n}{get\PYGZus{}midi\PYGZus{}pitch}\PYG{p}{(}\PYG{p}{)}
\PYG{g+go}{60}
\end{sphinxVerbatim}

\end{fulllineitems}



\begin{fulllineitems}
\pysiglinewithargsret{\sphinxbfcode{\sphinxupquote{get\_pitch\_class}}}{\emph{\DUrole{n}{start}\DUrole{o}{=}\DUrole{default_value}{0}}, \emph{\DUrole{n}{order}\DUrole{o}{=}\DUrole{default_value}{\textquotesingle{}chromatic\textquotesingle{}}}}{}
\sphinxAtStartPar
Get the pitch\sphinxhyphen{}class number on the circle of fifths or the chromatic circle.
\begin{quote}\begin{description}
\item[{Parameters}] \leavevmode\begin{itemize}
\item {} 
\sphinxAtStartPar
\sphinxstyleliteralstrong{\sphinxupquote{start}} (\sphinxstyleliteralemphasis{\sphinxupquote{int}}) \textendash{} Pitch\sphinxhyphen{}class number that gets mapped to C (default: 0).

\item {} 
\sphinxAtStartPar
\sphinxstyleliteralstrong{\sphinxupquote{order}} (\sphinxstyleliteralemphasis{\sphinxupquote{str}}) \textendash{} Return pitch\sphinxhyphen{}class number on the chromatic circle (default) or the circle of fifths.

\end{itemize}

\item[{Returns}] \leavevmode
\sphinxAtStartPar
The pitch class of the tone on the circle of fifths or the chromatic circle.

\item[{Return type}] \leavevmode
\sphinxAtStartPar
int

\end{description}\end{quote}
\subsubsection*{Example}

\begin{sphinxVerbatim}[commandchars=\\\{\}]
\PYG{g+gp}{\PYGZgt{}\PYGZgt{}\PYGZgt{} }\PYG{n}{t} \PYG{o}{=} \PYG{n}{Tone}\PYG{p}{(}\PYG{l+m+mi}{0}\PYG{p}{,}\PYG{l+m+mi}{7}\PYG{p}{,}\PYG{l+m+mi}{0}\PYG{p}{)} \PYG{c+c1}{\PYGZsh{} C sharp}
\PYG{g+gp}{\PYGZgt{}\PYGZgt{}\PYGZgt{} }\PYG{n}{t}\PYG{o}{.}\PYG{n}{get\PYGZus{}pitch\PYGZus{}class}\PYG{p}{(}\PYG{n}{order}\PYG{o}{=}\PYG{l+s+s2}{\PYGZdq{}}\PYG{l+s+s2}{chromatic}\PYG{l+s+s2}{\PYGZdq{}}\PYG{p}{)}
\PYG{g+go}{1}
\end{sphinxVerbatim}

\begin{sphinxVerbatim}[commandchars=\\\{\}]
\PYG{g+gp}{\PYGZgt{}\PYGZgt{}\PYGZgt{} }\PYG{n}{t} \PYG{o}{=} \PYG{n}{Tone}\PYG{p}{(}\PYG{l+m+mi}{0}\PYG{p}{,}\PYG{l+m+mi}{7}\PYG{p}{,}\PYG{l+m+mi}{0}\PYG{p}{)} \PYG{c+c1}{\PYGZsh{} C sharp}
\PYG{g+gp}{\PYGZgt{}\PYGZgt{}\PYGZgt{} }\PYG{n}{t}\PYG{o}{.}\PYG{n}{get\PYGZus{}pitch\PYGZus{}class}\PYG{p}{(}\PYG{n}{order}\PYG{o}{=}\PYG{l+s+s2}{\PYGZdq{}}\PYG{l+s+s2}{fifths}\PYG{l+s+s2}{\PYGZdq{}}\PYG{p}{)}
\PYG{g+go}{7}
\end{sphinxVerbatim}

\end{fulllineitems}



\begin{fulllineitems}
\pysiglinewithargsret{\sphinxbfcode{\sphinxupquote{get\_step}}}{}{}
\sphinxAtStartPar
Gets the diatonic letter name (C, D, E, F, G, A, or B) of the tone \sphinxstyleemphasis{without} accidentals.
\begin{quote}\begin{description}
\item[{Parameters}] \leavevmode
\sphinxAtStartPar
\sphinxstyleliteralstrong{\sphinxupquote{None}} \textendash{} 

\item[{Returns}] \leavevmode
\sphinxAtStartPar
The diatonic step of the tone.

\item[{Return type}] \leavevmode
\sphinxAtStartPar
str

\end{description}\end{quote}
\subsubsection*{Example}

\begin{sphinxVerbatim}[commandchars=\\\{\}]
\PYG{g+gp}{\PYGZgt{}\PYGZgt{}\PYGZgt{} }\PYG{n}{t} \PYG{o}{=} \PYG{n}{Tone}\PYG{p}{(}\PYG{l+m+mi}{0}\PYG{p}{,}\PYG{l+m+mi}{7}\PYG{p}{,}\PYG{l+m+mi}{0}\PYG{p}{)} \PYG{c+c1}{\PYGZsh{} C sharp}
\PYG{g+gp}{\PYGZgt{}\PYGZgt{}\PYGZgt{} }\PYG{n}{t}\PYG{o}{.}\PYG{n}{get\PYGZus{}step}\PYG{p}{(}\PYG{p}{)}
\PYG{g+go}{`C`}
\end{sphinxVerbatim}

\end{fulllineitems}



\begin{fulllineitems}
\pysiglinewithargsret{\sphinxbfcode{\sphinxupquote{get\_syntonic}}}{}{}
\sphinxAtStartPar
Gets the value of the syntonic level in Euler space.
Tones on the same syntonic line as central C are marked with \sphinxtitleref{\_},
and those above or below this line with \sphinxtitleref{‘} or \sphinxtitleref{,}, respectively.
\begin{quote}\begin{description}
\item[{Parameters}] \leavevmode
\sphinxAtStartPar
\sphinxstyleliteralstrong{\sphinxupquote{None}} \textendash{} 

\item[{Returns}] \leavevmode
\sphinxAtStartPar
The number of thirds above or below the central C.

\item[{Return type}] \leavevmode
\sphinxAtStartPar
int

\end{description}\end{quote}
\subsubsection*{Example}

\begin{sphinxVerbatim}[commandchars=\\\{\}]
\PYG{g+gp}{\PYGZgt{}\PYGZgt{}\PYGZgt{} }\PYG{n}{e1} \PYG{o}{=} \PYG{n}{Tone}\PYG{p}{(}\PYG{l+m+mi}{0}\PYG{p}{,}\PYG{l+m+mi}{4}\PYG{p}{,}\PYG{l+m+mi}{0}\PYG{p}{)} \PYG{c+c1}{\PYGZsh{} Pythagorean major third above C}
\PYG{g+gp}{\PYGZgt{}\PYGZgt{}\PYGZgt{} }\PYG{n}{e2} \PYG{o}{=} \PYG{n}{Tone}\PYG{p}{(}\PYG{l+m+mi}{0}\PYG{p}{,}\PYG{l+m+mi}{0}\PYG{p}{,}\PYG{l+m+mi}{1}\PYG{p}{)} \PYG{c+c1}{\PYGZsh{} Just major third above C}
\PYG{g+gp}{\PYGZgt{}\PYGZgt{}\PYGZgt{} }\PYG{n}{e3} \PYG{o}{=} \PYG{n}{Tone}\PYG{p}{(}\PYG{l+m+mi}{0}\PYG{p}{,}\PYG{l+m+mi}{8}\PYG{p}{,}\PYG{o}{\PYGZhy{}}\PYG{l+m+mi}{1}\PYG{p}{)} \PYG{c+c1}{\PYGZsh{} Just major third below G sharp}
\PYG{g+gp}{\PYGZgt{}\PYGZgt{}\PYGZgt{} }\PYG{n}{e1}\PYG{o}{.}\PYG{n}{get\PYGZus{}syntonic}\PYG{p}{(}\PYG{p}{)}\PYG{p}{,} \PYG{n}{e2}\PYG{o}{.}\PYG{n}{get\PYGZus{}syntonic}\PYG{p}{(}\PYG{p}{)}\PYG{p}{,} \PYG{n}{e3}\PYG{o}{.}\PYG{n}{get\PYGZus{}syntonic}\PYG{p}{(}\PYG{p}{)}
\PYG{g+go}{`\PYGZsq{}` `\PYGZus{}` `,`}
\end{sphinxVerbatim}

\end{fulllineitems}


\end{fulllineitems}


\begin{sphinxthebibliography}{10}
\bibitem[1]{8_bibliography:id12}
\sphinxAtStartPar
Edward Aldwell, Carl Schachter, and Allen Cadwallader. \sphinxstyleemphasis{Harmony and Voice Leading}. Cengage Learning, 4th edition, 2010.
\bibitem[2]{8_bibliography:id58}
\sphinxAtStartPar
Hernandez Benavides Miguel Angel and Galpin Ixent. Exploring the musicalization of texts in Gregorian Chant using Data Analytics. Technical Report, Universitaria de Bogotá Jorge Tadeo Lozano, 2020.
\bibitem[3]{8_bibliography:id70}
\sphinxAtStartPar
Christopher M Bishop. \sphinxstyleemphasis{Pattern Recognition and Machine Learning}. Springer, 2006.
\bibitem[4]{8_bibliography:id49}
\sphinxAtStartPar
S. Brown and J. Jordania. Universals in the world\textquotesingle{}s musics. \sphinxstyleemphasis{Psychology of Music}, 41(2):229\textendash{}248, 2013. \sphinxhref{https://doi.org/10.1177/0305735611425896}{doi:10.1177/0305735611425896}.
\bibitem[5]{8_bibliography:id18}
\sphinxAtStartPar
Allen Cadwallader and David Gagné. \sphinxstyleemphasis{Analysis of Tonal Music. A Schenkerian Approach}. Oxford University Press, 1998.
\bibitem[6]{8_bibliography:id28}
\sphinxAtStartPar
Emilios Cambouropoulos. Pitch Spelling: A Computational Model. \sphinxstyleemphasis{Music Perception: An Interdisciplinary Journal}, 20(4):411\textendash{}429, June 2003. \sphinxhref{https://doi.org/10.1525/mp.2003.20.4.411}{doi:10.1525/mp.2003.20.4.411}.
\bibitem[7]{8_bibliography:id41}
\sphinxAtStartPar
Elaine Chew. \sphinxstyleemphasis{Towards a Mathematical Model of Tonality}. PhD thesis, Massachusetts Institute of Technology, 2000.
\bibitem[8]{8_bibliography:id43}
\sphinxAtStartPar
Elaine Chew. \sphinxstyleemphasis{Mathematical and Computational Modeling of Tonality}. Volume 204. Springer, 2014. ISBN 978\sphinxhyphen{}1\sphinxhyphen{}4614\sphinxhyphen{}9474\sphinxhyphen{}4. \sphinxhref{https://doi.org/10.1007/978-1-4614-9475-1}{doi:10.1007/978\sphinxhyphen{}1\sphinxhyphen{}4614\sphinxhyphen{}9475\sphinxhyphen{}1}.
\bibitem[9]{8_bibliography:id29}
\sphinxAtStartPar
Elaine Chew and Yun\sphinxhyphen{}Ching Chen. Real\sphinxhyphen{}Time Pitch Spelling Using the Spiral Array. \sphinxstyleemphasis{Computer Music Journal}, 29(2):61\textendash{}76, June 2005. \sphinxhref{https://doi.org/10.1162/0148926054094378}{doi:10.1162/0148926054094378}.
\bibitem[10]{8_bibliography:id66}
\sphinxAtStartPar
Bas Cornelissen, Willem Zuidema, and John Ashley Burgoyne. Mode classification and natural units in plainchant. In \sphinxstyleemphasis{Proceedings of the 21st International Society for Music Information Retrieval Conference}, 869\textendash{}875. Montreal, Canada, October 2020. ISMIR. \sphinxhref{https://doi.org/10.5281/zenodo.4245572}{doi:10.5281/zenodo.4245572}.
\bibitem[11]{8_bibliography:id13}
\sphinxAtStartPar
Anne Danielsen, Mari Romarheim Haugen, and Alexander Refsum Jensenius. Moving to the Beat: Studying Entrainment to Micro\sphinxhyphen{}Rhythmic Changes in Pulse by Motion Capture. \sphinxstyleemphasis{Timing \& Time Perception}, 3(1\sphinxhyphen{}2):133\textendash{}154, May 2015. \sphinxhref{https://doi.org/10.1163/22134468-00002043}{doi:10.1163/22134468\sphinxhyphen{}00002043}.
\bibitem[12]{8_bibliography:id67}
\sphinxAtStartPar
Dimitrios S. Delviniotis. New Method of Byzantine Music (BM) Intervals\textquotesingle{} Measuring and Its Application in the Fourth Mode. \$A\$ New Approach of the Music Intervals\textquotesingle{} Definition. \sphinxstyleemphasis{Series Musicologica Balcanica}, 1(2):209\textendash{}232, February 2021. \sphinxhref{https://doi.org/10.26262/smb.v1i2.7941}{doi:10.26262/smb.v1i2.7941}.
\bibitem[13]{8_bibliography:id11}
\sphinxAtStartPar
Peter Desain and Henkjan Honing. The Formation of Rhythmic Categories and Metric Priming. \sphinxstyleemphasis{Perception}, 32(3):341\textendash{}365, March 2003. \sphinxhref{https://doi.org/10.1068/p3370}{doi:10.1068/p3370}.
\bibitem[14]{8_bibliography:id5}
\sphinxAtStartPar
Leonhard Euler. \sphinxstyleemphasis{Tentamen Novae Theoriae Musicae Ex Certissimis Harmoniae Principiis Dilucide Expositae}. Ex Typographia Academiae Scientiarum, St. Petersburg, 1739.
\bibitem[15]{8_bibliography:id73}
\sphinxAtStartPar
Vedad FamourZadeh. La musique persane, Formalisation algébrique des structures. Technical Report, IRCAM, 2005.
\bibitem[16]{8_bibliography:id25}
\sphinxAtStartPar
Allen Forte. \sphinxstyleemphasis{The Structure of Atonal Music}. Yale University Press, New Haven and London, 1977.
\bibitem[17]{8_bibliography:id6}
\sphinxAtStartPar
Paul Gilroy. \sphinxstyleemphasis{The Black Atlantic: Modernity and Double\sphinxhyphen{}Consciousness}. Harvard University Press, 1995.
\bibitem[18]{8_bibliography:id63}
\sphinxAtStartPar
Zsolt Gárdonyi and Hubert Nordhoff. \sphinxstyleemphasis{Harmonik}. Möseler Verlag, Wolfenbüttel, 2002.
\bibitem[19]{8_bibliography:id52}
\sphinxAtStartPar
Dora A Hanninen. Orientations, Criteria, Segments: A General Theory of Segmentation for Music Analysis. \sphinxstyleemphasis{Journal of Music Theory}, 45(2):345\textendash{}433, 2001.
\bibitem[20]{8_bibliography:id53}
\sphinxAtStartPar
Dora A. Hanninen. \sphinxstyleemphasis{A Theory of Music Analysis: On Segmentation and Associative Organization}. University of Rochester Press, Rochester, NY, first edition, December 2012.
\bibitem[21]{8_bibliography:id35}
\sphinxAtStartPar
Erin E. Hannon and Sandra E. Trehub. Tuning in to musical rhythms: infants learn more readily than adults. \sphinxstyleemphasis{Proceedings of the National Academy of Sciences of the United States of America}, 102(35):12639\textendash{}12643, 2005. \sphinxhref{https://doi.org/10.1073/pnas.0504254102}{doi:10.1073/pnas.0504254102}.
\bibitem[22]{8_bibliography:id74}
\sphinxAtStartPar
Daniel Harasim, Fabian C. Moss, Matthias Ramirez, and Martin Rohrmeier. Exploring the foundations of tonality: statistical cognitive modeling of modes in the history of Western classical music. \sphinxstyleemphasis{Humanities and Social Sciences Communications}, 8(1):1\textendash{}11, January 2021. \sphinxhref{https://doi.org/10.1057/s41599-020-00678-6}{doi:10.1057/s41599\sphinxhyphen{}020\sphinxhyphen{}00678\sphinxhyphen{}6}.
\bibitem[23]{8_bibliography:id61}
\sphinxAtStartPar
Diane J Hu and Lawrence K Saul. A probabilistic topic model for unsupervised learning of musical key\sphinxhyphen{}profiles. In \sphinxstyleemphasis{Proceedings of the 10th International Society for Music Information Retrieval Conference (ISMIR 2009)}, 441\textendash{}446. Kobe, Japan, October 2009. \sphinxhref{https://doi.org/10.1.1.174.7644}{doi:10.1.1.174.7644}.
\bibitem[24]{8_bibliography:id62}
\sphinxAtStartPar
Diane J Hu and Lawrence K Saul. A Probabilistic Topic Model for Music Analysis. In Y. Bengio, D. Schuurmans, J. D. Lafferty, C. K. I. Williams, and A. Culotta, editors, \sphinxstyleemphasis{Proceedings of the 22nd Conference on Neural Information Processing Systems 22 (NIPS 2009)}, 1\textendash{}4. Vancouver, Canada, 2009.
\bibitem[25]{8_bibliography:id56}
\sphinxAtStartPar
M Kemal Karaosmanoğlu. A Turkish Makam Music Symbolic Database for Music Information Retrieval: SymbTr. In \sphinxstyleemphasis{ISMIR}, 6. 2012.
\bibitem[26]{8_bibliography:id2}
\sphinxAtStartPar
David Lewin. \sphinxstyleemphasis{Generalized Musical Intervals and Transformations}. Oxford University Press, Oxford, 1987.
\bibitem[27]{8_bibliography:id32}
\sphinxAtStartPar
Robert Lieck and Martin Rohrmeier. Modelling Hierarchical Key Structure with Pitch Scapes. In \sphinxstyleemphasis{Proceedings of the 21st International Society for Music Information Retrieval Conference (ISMIR 2020)}, 811\textendash{}818. Montreal, Canada, 2020. \sphinxhref{https://doi.org/10.5281/zenodo.4245558}{doi:10.5281/zenodo.4245558}.
\bibitem[28]{8_bibliography:id8}
\sphinxAtStartPar
Justin London. \sphinxstyleemphasis{Hearing in Time: Psychological Aspects of Musical Meter}. Oxford University Press, Oxford, second edition, 2012.
\bibitem[29]{8_bibliography:id46}
\sphinxAtStartPar
Christopher Longuet\sphinxhyphen{}Higgins. The Three Dimensions of Harmony. In \sphinxstyleemphasis{Mental Processes: Studies in Cognitive Science}, pages 59\textendash{}63. MIT Press, Cambridge, MA, 1987.
\bibitem[30]{8_bibliography:id47}
\sphinxAtStartPar
Christopher Longuet\sphinxhyphen{}Higgins. Two Letters to a Musical Friend. In \sphinxstyleemphasis{Mental Processes: Studies in Cognitive Science}, pages 64\textendash{}81. MIT Press, Cambridge, MA, 1987.
\bibitem[31]{8_bibliography:id45}
\sphinxAtStartPar
Guerino Mazzola. \sphinxstyleemphasis{Geometrie der Töne: Elemente der Mathematischen Musiktheorie}. Birkhäuser, Basel, 1990.
\bibitem[32]{8_bibliography:id4}
\sphinxAtStartPar
Guerino Mazzola. The Euler Space. In \sphinxstyleemphasis{Basic Music Technology}, pages 51\textendash{}61. Springer, 2018. \sphinxhref{https://doi.org/10.1007/978-3-030-00982-3\_6}{doi:10.1007/978\sphinxhyphen{}3\sphinxhyphen{}030\sphinxhyphen{}00982\sphinxhyphen{}3\_6}.
\bibitem[33]{8_bibliography:id71}
\sphinxAtStartPar
Leland McInnes, John Healy, and James Melville. UMAP: Uniform Manifold Approximation and Projection for Dimension Reduction. \sphinxstyleemphasis{arXiv:1802.03426 {[}cs, stat{]}}, September 2020. \sphinxhref{https://arxiv.org/abs/1802.03426}{arXiv:1802.03426}.
\bibitem[34]{8_bibliography:id50}
\sphinxAtStartPar
Samuel Mehr, Manvir Singh, Dean Knox, Daniel Ketter, Daniel Pickens\sphinxhyphen{}Jones, S. Atwood, Christopher Lucas, Nori Jacoby, Alena Egner, Erin J. Hopkins, Rhea M. Howard, Joshua Hartshorne, Mariela Jennings, Jan Simson, Constance Bainbridge, Steven Pinker, Timothy J. O\textquotesingle{}Donnell, Max Krasnow, and Luke Glowacki. Universality and diversity in human song. \sphinxstyleemphasis{Science}, 366:eaax0868, 2019. \sphinxhref{https://doi.org/10.31234/OSF.IO/EMQ8R}{doi:10.31234/OSF.IO/EMQ8R}.
\bibitem[35]{8_bibliography:id27}
\sphinxAtStartPar
David Meredith. The ps13 pitch spelling algorithm. \sphinxstyleemphasis{Journal of New Music Research}, 35(2):121\textendash{}159, June 2006. \sphinxhref{https://doi.org/10.1080/09298210600834961}{doi:10.1080/09298210600834961}.
\bibitem[36]{8_bibliography:id26}
\sphinxAtStartPar
David Meredith and Geraint A Wiggins. Comparing Pitch Spelling Algorithms. \sphinxstyleemphasis{Proceedings of the International Symposium on Music Information Retrieval (ISMIR)}, pages 280\textendash{}287, 2005. \sphinxhref{https://doi.org/10.1007/978-3-540-31807-1\_14}{doi:10.1007/978\sphinxhyphen{}3\sphinxhyphen{}540\sphinxhyphen{}31807\sphinxhyphen{}1\_14}.
\bibitem[37]{8_bibliography:id44}
\sphinxAtStartPar
Fabian C Moss, Markus Neuwirth, and Martin Rohrmeier. The line of fifths and the co\sphinxhyphen{}evolution of tonal pitch classes. \sphinxstyleemphasis{Journal of Mathematics and Music}, accepted.
\bibitem[38]{8_bibliography:id14}
\sphinxAtStartPar
Fabian C. Moss. Choro Songbook Corpus. Zenodo, 2020. \sphinxhref{https://doi.org/10.5281/zenodo.3881347}{doi:10.5281/zenodo.3881347}.
\bibitem[39]{8_bibliography:id54}
\sphinxAtStartPar
Fabian C. Moss, Markus Neuwirth, Daniel Harasim, and Martin Rohrmeier. Statistical characteristics of tonal harmony: a corpus study of Beethoven\textquotesingle{}s string quartets. \sphinxstyleemphasis{PLoS ONE}, 14(6):e0217242, 2019. \sphinxhref{https://doi.org/10.1371/journal.pone.0217242}{doi:10.1371/journal.pone.0217242}.
\bibitem[40]{8_bibliography:id15}
\sphinxAtStartPar
Fabian C. Moss and Martin Rohrmeier. Discovering Tonal Profiles with Latent Dirichlet Allocation. \sphinxstyleemphasis{Music \& Science}, 4:20592043211048827, January 2021. \sphinxhref{https://doi.org/10.1177/20592043211048827}{doi:10.1177/20592043211048827}.
\bibitem[41]{8_bibliography:id16}
\sphinxAtStartPar
Fabian C. Moss, Willian Fernandes Souza, and Martin Rohrmeier. Harmony and form in Brazilian Choro: A corpus\sphinxhyphen{}driven approach to musical style analysis. \sphinxstyleemphasis{Journal of New Music Research}, 49(5):416\textendash{}437, 2020. \sphinxhref{https://doi.org/10.1080/09298215.2020.1797109}{doi:10.1080/09298215.2020.1797109}.
\bibitem[42]{8_bibliography:id24}
\sphinxAtStartPar
Meinhard Müller. \sphinxstyleemphasis{Fundamentals of Music Processing}. Springer, 2015.
\bibitem[43]{8_bibliography:id76}
\sphinxAtStartPar
Markus Neuwirth, Daniel Harasim, Fabian C. Moss, and Martin Rohrmeier. The Annotated Beethoven Corpus (ABC): A Dataset of Harmonic Analyses of All Beethoven String Quartets. \sphinxstyleemphasis{Frontiers in Digital Humanities}, 5(July):1\textendash{}5, 2018. \sphinxhref{https://doi.org/10.3389/fdigh.2018.00016}{doi:10.3389/fdigh.2018.00016}.
\bibitem[44]{8_bibliography:id42}
\sphinxAtStartPar
Thomas Noll. Die Vernunft in der Tradition: Neue mathematische Untersuchungen zu den alten Begriffen der Diatonizität. \sphinxstyleemphasis{Zeitschrift der Gesellschaft für Musiktheorie}, 2016.
\bibitem[45]{8_bibliography:id55}
\sphinxAtStartPar
Edward Nowacki. \sphinxstyleemphasis{Latin and Greek Music Theory}. Eastman Studies in Music. University of Rochester Press, Rochester, 2020.
\bibitem[46]{8_bibliography:id75}
\sphinxAtStartPar
Thomas Nuttall, Miguel G. Casado, Andres Ferraro, Darrell Conklin, and Rafael Caro Repetto. A computational exploration of melodic patterns in Arab\sphinxhyphen{}Andalusian music. \sphinxstyleemphasis{Journal of Mathematics and Music}, 0(0):1\textendash{}13, May 2021. \sphinxhref{https://doi.org/10.1080/17459737.2021.1917010}{doi:10.1080/17459737.2021.1917010}.
\bibitem[47]{8_bibliography:id68}
\sphinxAtStartPar
Richard Parncutt. Pitch\sphinxhyphen{}class prevalence in plainchant, scale\sphinxhyphen{}degree consonance, and the origin of the rising leading tone. \sphinxstyleemphasis{Journal of New Music Research}, 48(5):434\textendash{}448, October 2019. \sphinxhref{https://doi.org/10.1080/09298215.2019.1642360}{doi:10.1080/09298215.2019.1642360}.
\bibitem[48]{8_bibliography:id7}
\sphinxAtStartPar
Rainer Polak, Nori Jacoby, and Justin London. Both isochronous and non\sphinxhyphen{}isochronous metrical subdivision afford precise and stable ensemble entrainment: A corpus study of Malian jembe drumming. \sphinxstyleemphasis{Frontiers in Neuroscience}, 2016. \sphinxhref{https://doi.org/10.3389/fnins.2016.00285}{doi:10.3389/fnins.2016.00285}.
\bibitem[49]{8_bibliography:id23}
\sphinxAtStartPar
Harold S. Powers, Frans Wiering, James Porter, James Cowdery, Richard Widdess, Ruth Davis, Marc Perlman, Stephen Jones, and Allan Marett. Mode. In \sphinxstyleemphasis{Grove Music Online}. Oxford University Press, Oxford, 2001. \sphinxhref{https://doi.org/10.1093/gmo/9781561592630.article.43718}{doi:10.1093/gmo/9781561592630.article.43718}.
\bibitem[50]{8_bibliography:id9}
\sphinxAtStartPar
Andrea Ravignani, Bill Thompson, Massimo Lumaca, and Manon Grube. Why Do Durations in Musical Rhythms Conform to Small Integer Ratios? \sphinxstyleemphasis{Frontiers in Computational Neuroscience}, 12:1\textendash{}9, November 2018. \sphinxhref{https://doi.org/10.3389/fncom.2018.00086}{doi:10.3389/fncom.2018.00086}.
\bibitem[51]{8_bibliography:id17}
\sphinxAtStartPar
Martin Rohrmeier and Thore Graepel. Comparing Feature\sphinxhyphen{}Based Models of Harmony. In \sphinxstyleemphasis{Proceedings of the 9th International Symposium on Computer Music Modeling and Retrieval CMMR 2012}, 357\textendash{}370. London, UK, 2012. Springer.
\bibitem[52]{8_bibliography:id20}
\sphinxAtStartPar
Sebastian Rosenzweig, Frank Scherbaum, David Shugliashvili, Vlora Arifi\sphinxhyphen{}Müller, and Meinard Müller. Erkomaishvili Dataset: A Curated Corpus of Traditional Georgian Vocal Music for Computational Musicology. \sphinxstyleemphasis{Transactions of the International Society for Music Information Retrieval}, 3(1):31\textendash{}41, 2020. \sphinxhref{https://doi.org/10.5334/tismir.44}{doi:10.5334/tismir.44}.
\bibitem[53]{8_bibliography:id69}
\sphinxAtStartPar
Farshad Sanati. \sphinxstyleemphasis{An Investigation on the Value of Intervals in Persian Music}. Master Thesis, University of Jyväskylä, 2020.
\bibitem[54]{8_bibliography:id37}
\sphinxAtStartPar
Craig Stuart Sapp. Harmonic Visualizations of Tonal Music. In \sphinxstyleemphasis{The International Computer Music Conference (ICMC)}, 423\textendash{}430. 2001.
\bibitem[55]{8_bibliography:id38}
\sphinxAtStartPar
Craig Stuart Sapp. Visual hierarchical key analysis. \sphinxstyleemphasis{Computers in Entertainment}, 3(4):3, 2005. \sphinxhref{https://doi.org/10.1145/1095534.1095544}{doi:10.1145/1095534.1095544}.
\bibitem[56]{8_bibliography:id30}
\sphinxAtStartPar
Joshua Stoddard, Christopher Raphael, and Paul E Utgoff. Well\sphinxhyphen{}tempered spelling: A key\sphinxhyphen{}invariant pitch spelling algorithm. In \sphinxstyleemphasis{Proceedings of the 5th International Conference on Music Information Retrieval (ISMIR 2004)}. Barcelona, Spain, 2004.
\bibitem[57]{8_bibliography:id3}
\sphinxAtStartPar
Joseph Nathan Straus. \sphinxstyleemphasis{Introduction to post\sphinxhyphen{}tonal theory}. Pearson Prentice Hall, New York, 3rd edition, 2005.
\bibitem[58]{8_bibliography:id40}
\sphinxAtStartPar
David Temperley. The Line of Fifths. \sphinxstyleemphasis{Music Analysis}, 19(3):289\textendash{}319, 2000. \sphinxhref{https://doi.org/10.2307/854457}{doi:10.2307/854457}.
\bibitem[59]{8_bibliography:id31}
\sphinxAtStartPar
David Temperley. \sphinxstyleemphasis{The Cognition of Basic Musical Structures}. MIT Press, 2001.
\bibitem[60]{8_bibliography:id48}
\sphinxAtStartPar
Gary Tomlinson. \sphinxstyleemphasis{A Million Years of Music}. Princeton University Press, 2018.
\bibitem[61]{8_bibliography:id36}
\sphinxAtStartPar
Sandra E. Trehub, E. Glenn Schellenberg, and Stuart B. Kamenetsky. Infants\textquotesingle{} and adults\textquotesingle{} perception of scale structure. \sphinxstyleemphasis{Journal of Experimental Psychology: Human Perception and Performance}, 25(4):965\textendash{}975, 1999. \sphinxhref{https://doi.org/10.1037/0096-1523.25.4.965}{doi:10.1037/0096\sphinxhyphen{}1523.25.4.965}.
\bibitem[62]{8_bibliography:id59}
\sphinxAtStartPar
Dmitri Tymoczko. \sphinxstyleemphasis{A Geometry of Music: Harmony and Counterpoint in the Extended Common Practice}. Oxford University Press, Oxford, 2011.
\bibitem[63]{8_bibliography:id72}
\sphinxAtStartPar
Laurens Van Der Maaten and Geoffrey Hinton. Visualizing Data using t\sphinxhyphen{}SNE. \sphinxstyleemphasis{Journal of Machine Learning Research}, 9(November):2579\textendash{}2605, 2008.
\bibitem[64]{8_bibliography:id51}
\sphinxAtStartPar
Cédric Viaccoz, Daniel Harasim, Fabian C. Moss, and Martin Rohrmeier. Wavescapes: A visual hierarchical analysis of tonality using the discrete Fourier transform. \sphinxstyleemphasis{Musicae Scientiae}, pages 10298649211034906, January 2022. \sphinxhref{https://doi.org/10.1177/10298649211034906}{doi:10.1177/10298649211034906}.
\bibitem[65]{8_bibliography:id33}
\sphinxAtStartPar
Wayne Vitale and William Sethares. Balinese Gamelan Tuning: The Toth Archives. \sphinxstyleemphasis{Analytical Approaches to World Music}, 9(2):1\textendash{}39, 2021.
\bibitem[66]{8_bibliography:id19}
\sphinxAtStartPar
Robert W Wason and Elizabeth West Marvin. Riemann\textquotesingle{}s "Ideen zu Einer \textquotesingle{}Lehre von den Tonvorstellungen\textquotesingle{}": An Annotated Translation. \sphinxstyleemphasis{Journal of Music Theory}, 36(1):69\textendash{}79, 1992.
\bibitem[67]{8_bibliography:id34}
\sphinxAtStartPar
Gerrit Wendt and Rolf Bader. Analysis and Perception of Javanese Gamelan Tunings. In Rolf Bader, editor, \sphinxstyleemphasis{Computational Phonogram Archiving}, Current Research in Systematic Musicology, pages 129\textendash{}142. Springer International Publishing, Cham, 2019. \sphinxhref{https://doi.org/10.1007/978-3-030-02695-0\_6}{doi:10.1007/978\sphinxhyphen{}3\sphinxhyphen{}030\sphinxhyphen{}02695\sphinxhyphen{}0\_6}.
\bibitem[68]{8_bibliography:id57}
\sphinxAtStartPar
Frans Wiering. \sphinxstyleemphasis{The Language of the Modes: Studies in the History of Polyphonic Modality}. Routledge, New York and London, 2001.
\bibitem[69]{8_bibliography:id10}
\sphinxAtStartPar
Geraint A. Wiggins. Music, mind and mathematics: theory, reality and formality. \sphinxstyleemphasis{Journal of Mathematics and Music}, 6(2):111\textendash{}123, 2012. \sphinxhref{https://doi.org/10.1080/17459737.2012.694710}{doi:10.1080/17459737.2012.694710}.
\bibitem[70]{8_bibliography:id39}
\sphinxAtStartPar
Jason Yust. \sphinxstyleemphasis{Organized Time: Rhythm, Tonality, and Form}. Oxford Studies in Music Theory. Oxford University Press, Oxford, New York, 2018.
\end{sphinxthebibliography}


\renewcommand{\indexname}{Python Module Index}
\begin{sphinxtheindex}
\let\bigletter\sphinxstyleindexlettergroup
\bigletter{g}
\item\relax\sphinxstyleindexentry{gamuth}\sphinxstyleindexpageref{api:\detokenize{module-gamuth}}
\end{sphinxtheindex}

\renewcommand{\indexname}{Index}
\printindex
\end{document}