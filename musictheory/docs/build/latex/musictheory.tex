%% Generated by Sphinx.
\def\sphinxdocclass{report}
\documentclass[letterpaper,10pt,english]{sphinxmanual}
\ifdefined\pdfpxdimen
   \let\sphinxpxdimen\pdfpxdimen\else\newdimen\sphinxpxdimen
\fi \sphinxpxdimen=.75bp\relax

\PassOptionsToPackage{warn}{textcomp}
\usepackage[utf8]{inputenc}
\ifdefined\DeclareUnicodeCharacter
% support both utf8 and utf8x syntaxes
  \ifdefined\DeclareUnicodeCharacterAsOptional
    \def\sphinxDUC#1{\DeclareUnicodeCharacter{"#1}}
  \else
    \let\sphinxDUC\DeclareUnicodeCharacter
  \fi
  \sphinxDUC{00A0}{\nobreakspace}
  \sphinxDUC{2500}{\sphinxunichar{2500}}
  \sphinxDUC{2502}{\sphinxunichar{2502}}
  \sphinxDUC{2514}{\sphinxunichar{2514}}
  \sphinxDUC{251C}{\sphinxunichar{251C}}
  \sphinxDUC{2572}{\textbackslash}
\fi
\usepackage{cmap}
\usepackage[T1]{fontenc}
\usepackage{amsmath,amssymb,amstext}
\usepackage{babel}



\usepackage{times}
\expandafter\ifx\csname T@LGR\endcsname\relax
\else
% LGR was declared as font encoding
  \substitutefont{LGR}{\rmdefault}{cmr}
  \substitutefont{LGR}{\sfdefault}{cmss}
  \substitutefont{LGR}{\ttdefault}{cmtt}
\fi
\expandafter\ifx\csname T@X2\endcsname\relax
  \expandafter\ifx\csname T@T2A\endcsname\relax
  \else
  % T2A was declared as font encoding
    \substitutefont{T2A}{\rmdefault}{cmr}
    \substitutefont{T2A}{\sfdefault}{cmss}
    \substitutefont{T2A}{\ttdefault}{cmtt}
  \fi
\else
% X2 was declared as font encoding
  \substitutefont{X2}{\rmdefault}{cmr}
  \substitutefont{X2}{\sfdefault}{cmss}
  \substitutefont{X2}{\ttdefault}{cmtt}
\fi


\usepackage[Bjarne]{fncychap}
\usepackage{sphinx}

\fvset{fontsize=\small}
\usepackage{geometry}

% Include hyperref last.
\usepackage{hyperref}
% Fix anchor placement for figures with captions.
\usepackage{hypcap}% it must be loaded after hyperref.
% Set up styles of URL: it should be placed after hyperref.
\urlstyle{same}

\usepackage{sphinxmessages}
\setcounter{tocdepth}{2}



\title{musictheory}
\date{Apr 22, 2020}
\release{0.0.1}
\author{Fabian C.\@{} Moss}
\newcommand{\sphinxlogo}{\vbox{}}
\renewcommand{\releasename}{Release}
\makeindex
\begin{document}

\pagestyle{empty}
\sphinxmaketitle
\pagestyle{plain}
\sphinxtableofcontents
\pagestyle{normal}
\phantomsection\label{\detokenize{index::doc}}


Welcome!

This is not a pedagogical resource for basic music theory concepts
but an in-depth introduction into the structures of Western music,
built axiomatically from tones and their relations.
The logo, a \sphinxhref{https://en.wikipedia.org/wiki/Sixty-fourth\_note}{hemidemisemiquaver} (or a sixty-fourth note),
symbolically reflects this level of difficulty.

This project has been inspired by this great book:
\begin{itemize}
\item {} 
Lewin, D. (2007). \sphinxstyleemphasis{Generalized Intervals and Transformations}.
Oxford: Oxford University Press.

\end{itemize}

I recently also discovered \sphinxhref{https://pedrokroger.net/mfgan/}{Music for Geeks and Nerds} by Pedro Kroger
which looks very interesting.

Since this is ongoing work, I can give no guarantee for completeness or accuracy.
Feel free to \sphinxhref{mailto:fabianmoss@gmail.com}{contact me} with your questions and suggestions!


\chapter{Introduction}
\label{\detokenize{0_intro:introduction}}\label{\detokenize{0_intro::doc}}

\section{What is CM?}
\label{\detokenize{0_intro:what-is-cm}}\begin{itemize}
\item {} 
Description and relation to related fields
\textendash{} musicology
\textendash{} music theory
\textendash{} music information retrieval
\textendash{} music cognition
\textendash{} music psychology

\item {} 
modeling (!!!) very important chapter, maybe deserves its own paragraph. Modeling as a form of question/hypothesis guided research as opposed to wild tool application (‘’We did machine learning!’’)

\item {} 
How to do CM? conferences, journals, twitter, GitHub repositories, libraries

\item {} 
which language to use? Matlab, R, Python, Julia…

\end{itemize}


\section{General notes}
\label{\detokenize{0_intro:general-notes}}\begin{itemize}
\item {} 
Overarching goal is to have an ACCESSIBLE introduction for musicologists with elementary understanding of a programming language such as Python or R. Requirement should be a sound understanding of how functions, loops, and conditionals work.

\item {} 
Every chapter must have:
\textendash{} a very clear focus on one musicological question
\textendash{} one particular method to answer this question
\textendash{} a range of exercises (not always involving programming, also listening and composing)
\textendash{} and a list of relevant references

\item {} 
the algorithms/methods used in each chapter should be one of the most basic instances of a class of methods. The point is not to have the best classifier, the best dimensionality reduction, the best regression model etc. but rather to understand the class of problems that we are dealing with. Thinking in these abstract problem classes helps to recognize and understand the nature of other problems more easily.

\end{itemize}


\section{Notes (old)}
\label{\detokenize{0_intro:notes-old}}
General remark: Create excercises with listening, composing and analyzing tasks.
\begin{itemize}
\item {} 
Sounds in the external world

\item {} 
Perception, constraints (e.g. audible range)

\item {} 
discretization
- Musical Universals (3-7 note scales)
- allows symbolic representation

\item {} 
scales (independent from tuning/temperament): collections of pitches

\item {} 
members of scale: notes
- neighborness
- Schenkerian terms: neighbor notes
- pitch classes
- pitch class sets

\item {} 
intervals
- counterpoint
- consonance / dissonance
- interval classes
- interval class vectors

\item {} 
special pitch class sets: chords
- Triads
- Euler space
- tonnetz
- seventh chords

\item {} 
notes in time: durations, rhythm
- Schenkerian terms: passing notes
- cognitive framework: meter
- metrical hierarchies

\item {} 
visualisations (pitch-time plots)
- pianoroll
- MIDI
- modern Western notation
- different keys (not only treble and bass)

\end{itemize}


\chapter{Fundamentals}
\label{\detokenize{1_fundamentals:fundamentals}}\label{\detokenize{1_fundamentals::doc}}

\section{Representations and Formats}
\label{\detokenize{1_fundamentals:representations-and-formats}}
(This part could e.g. feature (algorithmic) composition exercises)


\subsection{Scores}
\label{\detokenize{1_fundamentals:scores}}\begin{itemize}
\item {} 
MEI

\item {} 
music21

\item {} 
MusicXML

\item {} 
**kern

\item {} 
MIDI

\end{itemize}


\subsection{Chords}
\label{\detokenize{1_fundamentals:chords}}
(symbolic)


\section{Tones, pitches, pitch classes}
\label{\detokenize{1_fundamentals:tones-pitches-pitch-classes}}

\subsection{Tones}
\label{\detokenize{1_fundamentals:tones}}

\subsection{pitches / pitch classes / pitch class sets}
\label{\detokenize{1_fundamentals:pitches-pitch-classes-pitch-class-sets}}

\subsection{GISs}
\label{\detokenize{1_fundamentals:giss}}

\subsection{Euler Space / Tonnetz / Tonal Space}
\label{\detokenize{1_fundamentals:euler-space-tonnetz-tonal-space}}

\subsection{Frequency / Temperament}
\label{\detokenize{1_fundamentals:frequency-temperament}}

\section{Time}
\label{\detokenize{1_fundamentals:time}}

\subsection{Notes}
\label{\detokenize{1_fundamentals:notes}}
(Tones + Duration)


\subsection{Rhythm}
\label{\detokenize{1_fundamentals:rhythm}}
(Duration patterns)


\subsection{Meter}
\label{\detokenize{1_fundamentals:meter}}
(Hierarchy)


\subsection{Musical time vs. performance time}
\label{\detokenize{1_fundamentals:musical-time-vs-performance-time}}

\chapter{Modeling musical sequences}
\label{\detokenize{2_sequences:modeling-musical-sequences}}\label{\detokenize{2_sequences::doc}}

\section{Regular Expressions}
\label{\detokenize{2_sequences:regular-expressions}}
(chord symbols, rhythms)


\section{n-gram models}
\label{\detokenize{2_sequences:n-gram-models}}
(melody, rhythms)


\section{Hidden Markov models}
\label{\detokenize{2_sequences:hidden-markov-models}}
(harmony)


\section{Probabilistic Context-Free Grammars}
\label{\detokenize{2_sequences:probabilistic-context-free-grammars}}
(form; choro)


\section{More advanced models}
\label{\detokenize{2_sequences:more-advanced-models}}
Neural nets


\chapter{Advanced applications}
\label{\detokenize{3_advanced:advanced-applications}}\label{\detokenize{3_advanced::doc}}

\section{Style (classification)}
\label{\detokenize{3_advanced:style-classification}}

\subsection{Feature clustering}
\label{\detokenize{3_advanced:feature-clustering}}
(k-means, PCA, …)


\subsection{Hierarchical clustering}
\label{\detokenize{3_advanced:hierarchical-clustering}}

\section{History}
\label{\detokenize{3_advanced:history}}
(regression, GPs)
\begin{itemize}
\item {} 
trends (maybe with a non note-based dataset e.g. metadata)

\end{itemize}


\section{Performance}
\label{\detokenize{3_advanced:performance}}\begin{itemize}
\item {} 
Spotify API to compare different recordings

\end{itemize}


\chapter{Developers}
\label{\detokenize{index:developers}}

\section{Installation}
\label{\detokenize{install:installation}}\label{\detokenize{install::doc}}
\begin{sphinxadmonition}{warning}{Warning:}
These instructions do not work yet.
\end{sphinxadmonition}

To install \sphinxcode{\sphinxupquote{musictheory}} type the following in your terminal:

\begin{sphinxVerbatim}[commandchars=\\\{\}]
\PYG{n}{pip} \PYG{n}{install} \PYG{n}{musictheory}
\end{sphinxVerbatim}


\section{API}
\label{\detokenize{api:api}}\label{\detokenize{api::doc}}
The entire \sphinxcode{\sphinxupquote{musictheory}} API.


\subsection{Tone}
\label{\detokenize{api:tone}}\index{Tone (class in main)@\spxentry{Tone}\spxextra{class in main}}

\begin{fulllineitems}
\phantomsection\label{\detokenize{api:main.Tone}}\pysiglinewithargsret{\sphinxbfcode{\sphinxupquote{class }}\sphinxcode{\sphinxupquote{main.}}\sphinxbfcode{\sphinxupquote{Tone}}}{\emph{octave=None}, \emph{fifth=None}, \emph{third=None}, \emph{name=None}}{}
Class for tones.
\index{get\_accidentals() (main.Tone method)@\spxentry{get\_accidentals()}\spxextra{main.Tone method}}

\begin{fulllineitems}
\phantomsection\label{\detokenize{api:main.Tone.get_accidentals}}\pysiglinewithargsret{\sphinxbfcode{\sphinxupquote{get\_accidentals}}}{}{}
Gets the accidentals of the tone (flats (\sphinxtitleref{b}) or sharps (\sphinxtitleref{\#}))..
\begin{quote}\begin{description}
\item[{Parameters}] \leavevmode
\sphinxstyleliteralstrong{\sphinxupquote{None}} \textendash{} 

\item[{Returns}] \leavevmode
The accidentals of the tone.

\item[{Return type}] \leavevmode
str

\end{description}\end{quote}
\subsubsection*{Example}

\begin{sphinxVerbatim}[commandchars=\\\{\}]
\PYG{g+gp}{\PYGZgt{}\PYGZgt{}\PYGZgt{} }\PYG{n}{t} \PYG{o}{=} \PYG{n}{Tone}\PYG{p}{(}\PYG{l+m+mi}{0}\PYG{p}{,}\PYG{l+m+mi}{7}\PYG{p}{,}\PYG{l+m+mi}{0}\PYG{p}{)} \PYG{c+c1}{\PYGZsh{} C sharp}
\PYG{g+gp}{\PYGZgt{}\PYGZgt{}\PYGZgt{} }\PYG{n}{t}\PYG{o}{.}\PYG{n}{get\PYGZus{}accidentals}\PYG{p}{(}\PYG{p}{)}
\PYG{g+go}{{}`\PYGZsh{}{}`}
\end{sphinxVerbatim}

\end{fulllineitems}

\index{get\_frequency() (main.Tone method)@\spxentry{get\_frequency()}\spxextra{main.Tone method}}

\begin{fulllineitems}
\phantomsection\label{\detokenize{api:main.Tone.get_frequency}}\pysiglinewithargsret{\sphinxbfcode{\sphinxupquote{get\_frequency}}}{\emph{chamber\_tone=440.0}, \emph{precision=2}}{}
Get the frequency of the tone.
\begin{quote}\begin{description}
\item[{Parameters}] \leavevmode\begin{itemize}
\item {} 
\sphinxstyleliteralstrong{\sphinxupquote{chamber\_tone}} (\sphinxstyleliteralemphasis{\sphinxupquote{float}}) \textendash{} The frequency in Hz of the chamber tone. Default: 440.0 (A)

\item {} 
\sphinxstyleliteralstrong{\sphinxupquote{precision}} (\sphinxstyleliteralemphasis{\sphinxupquote{int}}) \textendash{} Rounding precision.

\end{itemize}

\item[{Returns}] \leavevmode
The frequency of the tone in Hertz (Hz).

\item[{Return type}] \leavevmode
float

\end{description}\end{quote}
\subsubsection*{Example}

\begin{sphinxVerbatim}[commandchars=\\\{\}]
\PYG{g+gp}{\PYGZgt{}\PYGZgt{}\PYGZgt{} }\PYG{n}{t} \PYG{o}{=} \PYG{n}{Tone}\PYG{p}{(}\PYG{l+m+mi}{0}\PYG{p}{,}\PYG{l+m+mi}{0}\PYG{p}{,}\PYG{l+m+mi}{0}\PYG{p}{)}
\PYG{g+gp}{\PYGZgt{}\PYGZgt{}\PYGZgt{} }\PYG{n}{t}\PYG{o}{.}\PYG{n}{get\PYGZus{}frequency}\PYG{p}{(}\PYG{n}{precision}\PYG{o}{=}\PYG{l+m+mi}{3}\PYG{p}{)}
\PYG{g+go}{261.626}
\end{sphinxVerbatim}

\end{fulllineitems}

\index{get\_label() (main.Tone method)@\spxentry{get\_label()}\spxextra{main.Tone method}}

\begin{fulllineitems}
\phantomsection\label{\detokenize{api:main.Tone.get_label}}\pysiglinewithargsret{\sphinxbfcode{\sphinxupquote{get\_label}}}{}{}
Gets the complete label of the tone, consisting of its note name, syntonic position, and octave.
\begin{quote}\begin{description}
\item[{Parameters}] \leavevmode
\sphinxstyleliteralstrong{\sphinxupquote{None}} \textendash{} 

\item[{Returns}] \leavevmode
The accidentals of the tone.

\item[{Return type}] \leavevmode
str

\end{description}\end{quote}
\subsubsection*{Example}

\begin{sphinxVerbatim}[commandchars=\\\{\}]
\PYG{g+gp}{\PYGZgt{}\PYGZgt{}\PYGZgt{} }\PYG{n}{c} \PYG{o}{=} \PYG{n}{Tone}\PYG{p}{(}\PYG{l+m+mi}{0}\PYG{p}{,}\PYG{l+m+mi}{0}\PYG{p}{,}\PYG{l+m+mi}{0}\PYG{p}{)}
\PYG{g+gp}{\PYGZgt{}\PYGZgt{}\PYGZgt{} }\PYG{n}{ab} \PYG{o}{=} \PYG{n}{Tone}\PYG{p}{(}\PYG{l+m+mi}{0}\PYG{p}{,}\PYG{l+m+mi}{1}\PYG{p}{,}\PYG{o}{\PYGZhy{}}\PYG{l+m+mi}{1}\PYG{p}{)}
\PYG{g+gp}{\PYGZgt{}\PYGZgt{}\PYGZgt{} }\PYG{n}{c}\PYG{o}{.}\PYG{n}{get\PYGZus{}label}\PYG{p}{(}\PYG{p}{)}\PYG{p}{,} \PYG{n}{ab}\PYG{o}{.}\PYG{n}{get\PYGZus{}label}\PYG{p}{(}\PYG{p}{)}
\PYG{g+go}{{}`C\PYGZus{}0{}` {}`Ab,1{}`}
\end{sphinxVerbatim}

\end{fulllineitems}

\index{get\_midi\_pitch() (main.Tone method)@\spxentry{get\_midi\_pitch()}\spxextra{main.Tone method}}

\begin{fulllineitems}
\phantomsection\label{\detokenize{api:main.Tone.get_midi_pitch}}\pysiglinewithargsret{\sphinxbfcode{\sphinxupquote{get\_midi\_pitch}}}{}{}
Get the MIDI pitch of the tone.
\begin{quote}\begin{description}
\item[{Parameters}] \leavevmode
\sphinxstyleliteralstrong{\sphinxupquote{None}} \textendash{} 

\item[{Returns}] \leavevmode
The MIDI pitch of the tone if it is in MIDI pitch range (0\textendash{}128)

\item[{Return type}] \leavevmode
int

\end{description}\end{quote}
\subsubsection*{Example}

\begin{sphinxVerbatim}[commandchars=\\\{\}]
\PYG{g+gp}{\PYGZgt{}\PYGZgt{}\PYGZgt{} }\PYG{n}{t} \PYG{o}{=} \PYG{n}{Tone}\PYG{p}{(}\PYG{l+m+mi}{0}\PYG{p}{,}\PYG{l+m+mi}{0}\PYG{p}{,}\PYG{l+m+mi}{0}\PYG{p}{)}
\PYG{g+gp}{\PYGZgt{}\PYGZgt{}\PYGZgt{} }\PYG{n}{t}\PYG{o}{.}\PYG{n}{get\PYGZus{}midi\PYGZus{}pitch}\PYG{p}{(}\PYG{p}{)}
\PYG{g+go}{60}
\end{sphinxVerbatim}

\end{fulllineitems}

\index{get\_pitch\_class() (main.Tone method)@\spxentry{get\_pitch\_class()}\spxextra{main.Tone method}}

\begin{fulllineitems}
\phantomsection\label{\detokenize{api:main.Tone.get_pitch_class}}\pysiglinewithargsret{\sphinxbfcode{\sphinxupquote{get\_pitch\_class}}}{\emph{start=0}, \emph{order='chromatic'}}{}
Get the pitch-class number on the circle of fifths or the chromatic circle.
\begin{quote}\begin{description}
\item[{Parameters}] \leavevmode\begin{itemize}
\item {} 
\sphinxstyleliteralstrong{\sphinxupquote{start}} (\sphinxstyleliteralemphasis{\sphinxupquote{int}}) \textendash{} Pitch-class number that gets mapped to C (default: 0).

\item {} 
\sphinxstyleliteralstrong{\sphinxupquote{order}} (\sphinxstyleliteralemphasis{\sphinxupquote{str}}) \textendash{} Return pitch-class number on the chromatic circle (default) or the circle of fifths.

\end{itemize}

\item[{Returns}] \leavevmode
The pitch class of the tone on the circle of fifths or the chromatic circle.

\item[{Return type}] \leavevmode
int

\end{description}\end{quote}
\subsubsection*{Example}

\begin{sphinxVerbatim}[commandchars=\\\{\}]
\PYG{g+gp}{\PYGZgt{}\PYGZgt{}\PYGZgt{} }\PYG{n}{t} \PYG{o}{=} \PYG{n}{Tone}\PYG{p}{(}\PYG{l+m+mi}{0}\PYG{p}{,}\PYG{l+m+mi}{7}\PYG{p}{,}\PYG{l+m+mi}{0}\PYG{p}{)} \PYG{c+c1}{\PYGZsh{} C sharp}
\PYG{g+gp}{\PYGZgt{}\PYGZgt{}\PYGZgt{} }\PYG{n}{t}\PYG{o}{.}\PYG{n}{get\PYGZus{}pitch\PYGZus{}class}\PYG{p}{(}\PYG{n}{order}\PYG{o}{=}\PYG{l+s+s2}{\PYGZdq{}}\PYG{l+s+s2}{chromatic}\PYG{l+s+s2}{\PYGZdq{}}\PYG{p}{)}
\PYG{g+go}{1}
\end{sphinxVerbatim}

\begin{sphinxVerbatim}[commandchars=\\\{\}]
\PYG{g+gp}{\PYGZgt{}\PYGZgt{}\PYGZgt{} }\PYG{n}{t} \PYG{o}{=} \PYG{n}{Tone}\PYG{p}{(}\PYG{l+m+mi}{0}\PYG{p}{,}\PYG{l+m+mi}{7}\PYG{p}{,}\PYG{l+m+mi}{0}\PYG{p}{)} \PYG{c+c1}{\PYGZsh{} C sharp}
\PYG{g+gp}{\PYGZgt{}\PYGZgt{}\PYGZgt{} }\PYG{n}{t}\PYG{o}{.}\PYG{n}{get\PYGZus{}pitch\PYGZus{}class}\PYG{p}{(}\PYG{n}{order}\PYG{o}{=}\PYG{l+s+s2}{\PYGZdq{}}\PYG{l+s+s2}{fifths}\PYG{l+s+s2}{\PYGZdq{}}\PYG{p}{)}
\PYG{g+go}{7}
\end{sphinxVerbatim}

\end{fulllineitems}

\index{get\_step() (main.Tone method)@\spxentry{get\_step()}\spxextra{main.Tone method}}

\begin{fulllineitems}
\phantomsection\label{\detokenize{api:main.Tone.get_step}}\pysiglinewithargsret{\sphinxbfcode{\sphinxupquote{get\_step}}}{}{}
Gets the diatonic letter name (C, D, E, F, G, A, or B) of the tone \sphinxstyleemphasis{without} accidentals.
\begin{quote}\begin{description}
\item[{Parameters}] \leavevmode
\sphinxstyleliteralstrong{\sphinxupquote{None}} \textendash{} 

\item[{Returns}] \leavevmode
The diatonic step of the tone.

\item[{Return type}] \leavevmode
str

\end{description}\end{quote}
\subsubsection*{Example}

\begin{sphinxVerbatim}[commandchars=\\\{\}]
\PYG{g+gp}{\PYGZgt{}\PYGZgt{}\PYGZgt{} }\PYG{n}{t} \PYG{o}{=} \PYG{n}{Tone}\PYG{p}{(}\PYG{l+m+mi}{0}\PYG{p}{,}\PYG{l+m+mi}{7}\PYG{p}{,}\PYG{l+m+mi}{0}\PYG{p}{)} \PYG{c+c1}{\PYGZsh{} C sharp}
\PYG{g+gp}{\PYGZgt{}\PYGZgt{}\PYGZgt{} }\PYG{n}{t}\PYG{o}{.}\PYG{n}{get\PYGZus{}step}\PYG{p}{(}\PYG{p}{)}
\PYG{g+go}{{}`C{}`}
\end{sphinxVerbatim}

\end{fulllineitems}

\index{get\_syntonic() (main.Tone method)@\spxentry{get\_syntonic()}\spxextra{main.Tone method}}

\begin{fulllineitems}
\phantomsection\label{\detokenize{api:main.Tone.get_syntonic}}\pysiglinewithargsret{\sphinxbfcode{\sphinxupquote{get\_syntonic}}}{}{}
Gets the value of the syntonic level in Euler space.
Tones on the same syntonic line as central C are marked with \sphinxtitleref{\_},
and those above or below this line with \sphinxtitleref{‘} or \sphinxtitleref{,}, respectively.
\begin{quote}\begin{description}
\item[{Parameters}] \leavevmode
\sphinxstyleliteralstrong{\sphinxupquote{None}} \textendash{} 

\item[{Returns}] \leavevmode
The number of thirds above or below the central C.

\item[{Return type}] \leavevmode
int

\end{description}\end{quote}
\subsubsection*{Example}

\begin{sphinxVerbatim}[commandchars=\\\{\}]
\PYG{g+gp}{\PYGZgt{}\PYGZgt{}\PYGZgt{} }\PYG{n}{e1} \PYG{o}{=} \PYG{n}{Tone}\PYG{p}{(}\PYG{l+m+mi}{0}\PYG{p}{,}\PYG{l+m+mi}{4}\PYG{p}{,}\PYG{l+m+mi}{0}\PYG{p}{)} \PYG{c+c1}{\PYGZsh{} Pythagorean major third above C}
\PYG{g+gp}{\PYGZgt{}\PYGZgt{}\PYGZgt{} }\PYG{n}{e2} \PYG{o}{=} \PYG{n}{Tone}\PYG{p}{(}\PYG{l+m+mi}{0}\PYG{p}{,}\PYG{l+m+mi}{0}\PYG{p}{,}\PYG{l+m+mi}{1}\PYG{p}{)} \PYG{c+c1}{\PYGZsh{} Just major third above C}
\PYG{g+gp}{\PYGZgt{}\PYGZgt{}\PYGZgt{} }\PYG{n}{e3} \PYG{o}{=} \PYG{n}{Tone}\PYG{p}{(}\PYG{l+m+mi}{0}\PYG{p}{,}\PYG{l+m+mi}{8}\PYG{p}{,}\PYG{o}{\PYGZhy{}}\PYG{l+m+mi}{1}\PYG{p}{)} \PYG{c+c1}{\PYGZsh{} Just major third below G sharp}
\PYG{g+gp}{\PYGZgt{}\PYGZgt{}\PYGZgt{} }\PYG{n}{e1}\PYG{o}{.}\PYG{n}{get\PYGZus{}syntonic}\PYG{p}{(}\PYG{p}{)}\PYG{p}{,} \PYG{n}{e2}\PYG{o}{.}\PYG{n}{get\PYGZus{}syntonic}\PYG{p}{(}\PYG{p}{)}\PYG{p}{,} \PYG{n}{e3}\PYG{o}{.}\PYG{n}{get\PYGZus{}syntonic}\PYG{p}{(}\PYG{p}{)}
\PYG{g+go}{{}`\PYGZsq{}{}` {}`\PYGZus{}{}` {}`,{}`}
\end{sphinxVerbatim}

\end{fulllineitems}


\end{fulllineitems}



\subsection{Interval}
\label{\detokenize{api:interval}}\index{Interval (class in main)@\spxentry{Interval}\spxextra{class in main}}

\begin{fulllineitems}
\phantomsection\label{\detokenize{api:main.Interval}}\pysiglinewithargsret{\sphinxbfcode{\sphinxupquote{class }}\sphinxcode{\sphinxupquote{main.}}\sphinxbfcode{\sphinxupquote{Interval}}}{\emph{source}, \emph{target}}{}
Class for an interval between two tones \sphinxtitleref{s} (source) and \sphinxtitleref{t} (target).
\index{get\_euclidean\_distance() (main.Interval method)@\spxentry{get\_euclidean\_distance()}\spxextra{main.Interval method}}

\begin{fulllineitems}
\phantomsection\label{\detokenize{api:main.Interval.get_euclidean_distance}}\pysiglinewithargsret{\sphinxbfcode{\sphinxupquote{get\_euclidean\_distance}}}{\emph{precision=2}}{}
Calculates the Euclidean distance between two tones
with coordinates in Euler space.
\begin{quote}\begin{description}
\item[{Parameters}] \leavevmode
\sphinxstyleliteralstrong{\sphinxupquote{precision}} (\sphinxstyleliteralemphasis{\sphinxupquote{int}}) \textendash{} Rounding precision.

\item[{Returns}] \leavevmode
The Euclidean distance between two tones \sphinxtitleref{s} (source) and \sphinxtitleref{t} (target).

\item[{Return type}] \leavevmode
float

\end{description}\end{quote}
\subsubsection*{Example}

\begin{sphinxVerbatim}[commandchars=\\\{\}]
\PYG{g+gp}{\PYGZgt{}\PYGZgt{}\PYGZgt{} }\PYG{n}{s} \PYG{o}{=} \PYG{n}{Tone}\PYG{p}{(}\PYG{l+m+mi}{0}\PYG{p}{,}\PYG{l+m+mi}{0}\PYG{p}{,}\PYG{l+m+mi}{0}\PYG{p}{)} \PYG{c+c1}{\PYGZsh{} C\PYGZus{}0}
\PYG{g+gp}{\PYGZgt{}\PYGZgt{}\PYGZgt{} }\PYG{n}{t} \PYG{o}{=} \PYG{n}{Tone}\PYG{p}{(}\PYG{l+m+mi}{1}\PYG{p}{,}\PYG{l+m+mi}{2}\PYG{p}{,}\PYG{l+m+mi}{1}\PYG{p}{)} \PYG{c+c1}{\PYGZsh{} D\PYGZsq{}1}
\PYG{g+gp}{\PYGZgt{}\PYGZgt{}\PYGZgt{} }\PYG{n}{i} \PYG{o}{=} \PYG{n}{Interval}\PYG{p}{(}\PYG{n}{s}\PYG{p}{,}\PYG{n}{t}\PYG{p}{)}
\PYG{g+gp}{\PYGZgt{}\PYGZgt{}\PYGZgt{} }\PYG{n}{i}\PYG{o}{.}\PYG{n}{get\PYGZus{}euclidean\PYGZus{}distance}\PYG{p}{(}\PYG{p}{)}
\PYG{g+go}{2.45}
\end{sphinxVerbatim}

\end{fulllineitems}

\index{get\_generic\_interval() (main.Interval method)@\spxentry{get\_generic\_interval()}\spxextra{main.Interval method}}

\begin{fulllineitems}
\phantomsection\label{\detokenize{api:main.Interval.get_generic_interval}}\pysiglinewithargsret{\sphinxbfcode{\sphinxupquote{get\_generic\_interval}}}{\emph{directed=True}}{}
Generic interval (directed) between two tones.
\begin{quote}\begin{description}
\item[{Parameters}] \leavevmode
\sphinxstyleliteralstrong{\sphinxupquote{directed}} (\sphinxstyleliteralemphasis{\sphinxupquote{bool}}) \textendash{} Affects whether the returned interval is directed or not.

\item[{Returns}] \leavevmode
(Directed) generic interval from \sphinxtitleref{s} to \sphinxtitleref{t}.

\item[{Return type}] \leavevmode
int

\end{description}\end{quote}
\subsubsection*{Example}

\begin{sphinxVerbatim}[commandchars=\\\{\}]
\PYG{g+gp}{\PYGZgt{}\PYGZgt{}\PYGZgt{} }\PYG{n}{db} \PYG{o}{=} \PYG{n}{Tone}\PYG{p}{(}\PYG{l+m+mi}{0}\PYG{p}{,}\PYG{o}{\PYGZhy{}}\PYG{l+m+mi}{1}\PYG{p}{,}\PYG{o}{\PYGZhy{}}\PYG{l+m+mi}{1}\PYG{p}{)} \PYG{c+c1}{\PYGZsh{} Db,0}
\PYG{g+gp}{\PYGZgt{}\PYGZgt{}\PYGZgt{} }\PYG{n}{b} \PYG{o}{=} \PYG{n}{Tone}\PYG{p}{(}\PYG{l+m+mi}{0}\PYG{p}{,}\PYG{l+m+mi}{1}\PYG{p}{,}\PYG{l+m+mi}{1}\PYG{p}{)} \PYG{c+c1}{\PYGZsh{} B\PYGZsq{}0}
\PYG{g+gp}{\PYGZgt{}\PYGZgt{}\PYGZgt{} }\PYG{n}{i1} \PYG{o}{=} \PYG{n}{Interval}\PYG{p}{(}\PYG{n}{db}\PYG{p}{,} \PYG{n}{b}\PYG{p}{)} \PYG{c+c1}{\PYGZsh{} the interval between Db0 and B1 is an ascending thirteenth}
\PYG{g+gp}{\PYGZgt{}\PYGZgt{}\PYGZgt{} }\PYG{n}{i1}\PYG{o}{.}\PYG{n}{generic\PYGZus{}interval}\PYG{p}{(}\PYG{p}{)}
\PYG{g+go}{13}
\end{sphinxVerbatim}

\begin{sphinxVerbatim}[commandchars=\\\{\}]
\PYG{g+gp}{\PYGZgt{}\PYGZgt{}\PYGZgt{} }\PYG{n}{i2} \PYG{o}{=} \PYG{n}{Interval}\PYG{p}{(}\PYG{n}{b}\PYG{p}{,} \PYG{n}{db}\PYG{p}{)} \PYG{c+c1}{\PYGZsh{} the interval between B1 and Db0 is a descending thirteenth}
\PYG{g+gp}{\PYGZgt{}\PYGZgt{}\PYGZgt{} }\PYG{n}{i2}\PYG{o}{.}\PYG{n}{generic\PYGZus{}interval}\PYG{p}{(}\PYG{p}{)}
\PYG{g+go}{\PYGZhy{}13}
\end{sphinxVerbatim}

\begin{sphinxVerbatim}[commandchars=\\\{\}]
\PYG{g+gp}{\PYGZgt{}\PYGZgt{}\PYGZgt{} }\PYG{n}{i3} \PYG{o}{=} \PYG{n}{Interval}\PYG{p}{(}\PYG{n}{b}\PYG{p}{,} \PYG{n}{db}\PYG{p}{)} \PYG{c+c1}{\PYGZsh{} the interval between B1 and Db0 is a descending thirteenth}
\PYG{g+gp}{\PYGZgt{}\PYGZgt{}\PYGZgt{} }\PYG{n}{i3}\PYG{o}{.}\PYG{n}{generic\PYGZus{}interval}\PYG{p}{(}\PYG{n}{directed}\PYG{o}{=}\PYG{k+kc}{False}\PYG{p}{)}
\PYG{g+go}{13}
\end{sphinxVerbatim}

\end{fulllineitems}


\end{fulllineitems}



\chapter{Indices and tables}
\label{\detokenize{index:indices-and-tables}}\begin{itemize}
\item {} 
\DUrole{xref,std,std-ref}{genindex}

\item {} 
\DUrole{xref,std,std-ref}{modindex}

\item {} 
\DUrole{xref,std,std-ref}{search}

\end{itemize}



\renewcommand{\indexname}{Index}
\printindex
\end{document}