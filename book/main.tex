\documentclass{book}
\usepackage[utf8]{inputenc}
\usepackage[english]{babel}

\usepackage[
   backend=biber,
   style=apa,
   citestyle=authoryear-comp,
   uniquename=false,
	 uniquelist=false,
	 sorting=ynt,
	 % maxcitenames=2
   natbib=true
 ]{biblatex}
\addbibresource{references.bib}

\title{Computational Music Analysis}
\author{Fabian C. Moss}

\begin{document}

\maketitle 

\tableofcontents

\chapter{What is Computational Music Analysis?}

\citep{horton_compositionality_2001}

\chapter{Representations, Formats, and Programs}

\section{Notes}

pitches, pitch classes, pitch class sets, GISs, 

\section{Chords}

RomanText, Music21, ABC, \*\*kern

\section{Scores}

MIDI, MusicXML, MEI

\chapter{Music Models}

\section{Regular Expressions}

(Chord symbols)

\section{$n$-gram Models}

(Melody)

\section{Hidden Markov Models}

(Functional Harmony)

\section{Probabilistic Context-Free Grammars}

(Harmony, Form)

\chapter{Style}

- Zipf's law (style, idiom, intra-opus patterns) \citep{Meyer1989}
- feature clustering (k-means, PCA, ...)

\chapter{History}

- trends (maybe with a non note-based dataset e.g. metadata)

\chapter{Performance}

- Spotify API to compare different recordings


%%%%% TAIL %%%%%%%

\printbibliography

\end{document}